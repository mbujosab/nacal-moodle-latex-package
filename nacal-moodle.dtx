% \iffalse meta-comment
% 
% Copyright (C) 2023 by Marcos Bujosa <mbujosab@ucm.es>
% -----------------------------------------------------
% 
% This file may be distributed and/or modified under the
% conditions of the LaTeX Project Public License, either
% version 1.3 of this license or (at your option) any later
% version. The latest version of this license is in:
% 
% http://www.latex-project.org/lppl.txt
% 
% and version 1.3 or later is part of all distributions of
% LaTeX version 2005/12/01 or later.
% 
% \fi
%
% \iffalse
%<package>\NeedsTeXFormat{LaTeX2e}[2005/12/01]
%<package>\ProvidesPackage{nacal-moodle}
%<package>    [2023/01/08 v1.0 Notación Asociativa para un Curso de Álgebra en Moodle]
%<package>\RequirePackage{moodle}
%<package>\RequirePackageWithOptions{nacal}
%<package>\DeclareOption{esp}{\html@def\eng#1#2{#1}}
%<package>\DeclareOption{eng}{\html@def\eng#1#2{#2}}
%<package>\DeclareOption*{\PackageWarning{nacal-moodle}{Unknown ‘\CurrentOption’}}
%<package>\ExecuteOptions{esp}
%<package>\ProcessOptions\relax
% 
%<*driver>
\documentclass{ltxdoc}
\usepackage[dvipsnames,svgnames]{xcolor}
\usepackage{nacal-moodle}
\usepackage[utf8]{inputenc}
\usepackage[T1]{fontenc}
\usepackage[spanish]{babel}
\usepackage[colorlinks=true,linkcolor=DarkGreen,urlcolor=Maroon]{hyperref}
%\usepackage{fullpage}
\usepackage[twoside, a4paper, hmargin=2.8cm, vmargin=2.2cm, includeheadfoot]{geometry} 
\EnableCrossrefs         
\CodelineIndex
\RecordChanges
%%\OnlyDescription
\usepackage{nacal}
\setcounter{tocdepth}{4}
\begin{document}
  \DocInput{nacal-moodle.dtx}
\end{document}
%</driver>
% \fi
%
% \CheckSum{0}
% 
% \CharacterTable
%  {Upper-case    \A\B\C\D\E\F\G\H\I\J\K\L\M\N\O\P\Q\R\S\T\U\V\W\X\Y\Z
%   Lower-case    \a\b\c\d\e\f\g\h\i\j\k\l\m\n\o\p\q\r\s\t\u\v\w\x\y\z
%   Digits        \0\1\2\3\4\5\6\7\8\9
%   Exclamation   \!     Double quote  \"     Hash (number) \#
%   Dollar        \$     Percent       \%     Ampersand     \&
%   Acute accent  \'     Left paren    \(     Right paren   \)
%   Asterisk      \*     Plus          \+     Comma         \,
%   Minus         \-     Point         \.     Solidus       \/
%   Colon         \:     Semicolon     \;     Less than     \<
%   Equals        \=     Greater than  \>     Question mark \?
%   Commercial at \@     Left bracket  \[     Backslash     \\
%   Right bracket \]     Circumflex    \^     Underscore    \_
%   Grave accent  \`     Left brace    \{     Vertical bar  \|
%   Right brace   \}     Tilde         \~}
%
% \changes{v1.0}{2022/12/04}{Versión inicial}
%
% \GetFileInfo{nacal-moodle.sty}
%
% \DoNotIndex{\left,\right }
% \DoNotIndex{\\!,\\,}
% \DoNotIndex{\scriptscriptstyle,\times}
% \DoNotIndex{\html@def}
%
% 
% \title{El paquete \textsf{nacal-moodle}\thanks{Este documento 
%   corresponde a \textsf{nacal-moodle}~\fileversion, fecha \filedate.}}
% \author{Marcos Bujosa \\ \texttt{mbujosab@ucm.es}}
% 
% \maketitle
% 
% \begin{abstract}
%   Paquete para utilizar los comandos de \LaTeX{} del Curso de Álgebra
%   Lineal con Notación Asociativa (NAcAL) con el paquete moodle.sty y
%   así poder generar bancos de preguntas en formato \texttt{xml} con el
%   paquete \href{https://ctan.org/pkg/moodle}{\textsf{moodle}}
%   (\url{https://ctan.org/pkg/moodle}) que tengan una notación
%   aproximadamente igual a la del libro
%   \href{https://github.com/mbujosab/CursoDeAlgebraLineal}{Un Curso de
%     Álgebra Lineal}
%   (\url{https://github.com/mbujosab/CursoDeAlgebraLineal}).
% \end{abstract}
% 
% \tableofcontents
% 
% \section*{Introducción}
% 
% Para el Curso de Álgebra Lineal con Notación Asociativa he creado
% multitud de macros que definen la notación empleada en el material
% docente (libro, transparencias, ejercicios, notebooks o vídeos). Fijar
% la notación en los bancos de preguntas de Moodle no es sencillo (hay
% que convertir el código \LaTeX{} a \texttt{xml}). Este paquete es un
% intento de aproximar la notación de banco de preguntas al resto del
% material. 
% 
% La idea la obtuve al encontrar el paquete
% \href{https://github.com/alephsub0/LaTeX_aleph-moodle}{\textsf{aleph-comandos}}
% (\url{https://github.com/alephsub0/LaTeX_aleph-moodle/blob/main/aleph-moodle.pdf})
% de Jonathan Ortiz y Andrés Merino y que hace uso la macro |\html@\def|
% del paquete \href{https://ctan.org/pkg/moodle}{\textsf{moodle}} de
% Anders Hendrickson y Matthieu Guerquin-Kern.
% 
% Este método tiene una importante limitación. No es posible implementar
% todas la macros que definí al escribir el libro, pues al convertir el
% código \LaTeX{} a \texttt{xml} con |\html@\def| no podemos usar ni las
% versiones con asterisco de los comandos, ni tampoco comandos con
% argumentos opcionales. Así pues, la redefinición de las macros en este
% paquete no usa ni comandos con asterisco ni argumentos opcionales.
% 
% \section{Uso}
% 
% \subsection{Conjuntos de números}
% Respecto a estos comandos, véase el párrafo explicativo de la
% Sección~\ref{sec:TransfElem}
% 
% \DescribeMacro{\Nn}
% \DescribeMacro{\Zz}
% \DescribeMacro{\Rr}
% \DescribeMacro{\CC}
% Los comandos \cs{Nn}, \cs{Zz}, \cs{Rr} y \cs{CC} no tienen argumentos
% y denotan el conjunto de números naturales, de números enteros, de números reales y
% números complejos respectivamente
% \begin{center}
%   |\Nn \Zz \Rr \CC|
%   \hspace{1.2cm}
%   \fbox{$\Nn\ \Zz\ \Rr\ \CC$}
% \end{center}
% 
% \subsection{Paréntesis y corchetes}
% Me resulta agradable normalizar el tamaño de los paréntesis y otros
% tipos de llaves. En general prefiero que en las expresiones
% matemáticas de tipo \emph{ecuación} o \emph{displaymath} los
% paréntesis sean un poco mayores que aquello que encierran. Pero prefiero paréntesis pequeños en las
% expresiones entre líneas dentro de los párrafos. 
% 
% La conversión de comandos \LaTeX{} a \texttt{xml} no
% permite ni comandos con estrella ni con argumentos opcionales, así que
% estamos muy limitados. Tan solo he podido definir dos comandos en este caso. Con
% \cs{parentesis} (con las primera letra en minúsculas) escribiremos
% paréntesis pequeños y con \cs{Parentesis} (con la primera letra en
% mayúsculas) el tamaño del paréntesis se ajusta al objeto encerrado
% (desgraciadamente parece que no puedo hacer más al convertir a
% \texttt{xml}). Seguiré idéntico convenio con los \emph{corchetes}.
% \medskip
% 
% \DescribeMacro{\parentesis}
% El comando \cs{parentesis} tiene 1
% argumento,\;\cs{parentesis}\marg{contenido},\; y pone un paréntesis
% con |(| y |)| alrededor del \marg{contenido}
% \begin{center}
%   |\parentesis{A}| 
%   \hspace{1.2cm}
%   \fbox{$\parentesis*{A}$}
% \end{center}
% 
% \DescribeMacro{\Parentesis}
% El comando \cs{Parentesis} tiene 1
% argumento,\;\cs{Parentesis}\marg{contenido},\; y pone un paréntesis
% con |\left(| y |\right)| alrededor del \marg{contenido}, por lo que el
% paréntesis se ajusta al tamaño del \marg{contenido}.
% \begin{center}
%   |\Parentesis{A}| 
%   \hspace{1.2cm}
%   \fbox{$\Parentesis*{A}$}
% \end{center}
% \begin{center}
%   |\Parentesis{ \int\limits_a^b h(x) dx }| 
%   \hspace{1.2cm}
%   \fbox{$\Parentesis*{\int\limits_a^b h(x) dx}$}
% \end{center}
% 
% \DescribeMacro{\corchetes}
% El comando \cs{corchetes} tiene 1 argumento,\;
% \cs{corchetes}\marg{contenido},\; y pone un corchete con |[| y |]|
% alrededor del \marg{contenido}
% \begin{center}
%   |\corchetes{A}| 
%   \hspace{1.2cm}
%   \fbox{$\corchetes*{A}$}
% \end{center}
% 
% \DescribeMacro{\Corchetes}
% El comando \cs{Corchetes} tiene 1
% argumento,\;\cs{Corchetes}\marg{contenido},\; y pone un corchete con
% |\left[| y |\right]| alrededor del \marg{contenido}, por lo que el
% corchete se ajusta al tamaño del \marg{contenido}.
% \begin{center}
%   |\Corchetes{ \int\limits_a^b h(x) dx }| 
%   \hspace{1.2cm}
%   \fbox{$\Corchetes*{\int\limits_a^b h(x) dx}$}
% \end{center}
% 
% \subsubsection{Regla mnemotécnica para comandos que escriben expresiones con paréntesis}
% 
% \noindent
% \emph{Seguiré la siguiente regla con la nomenclatura de algunos
%   comandos}.
% \begin{itemize}
% \item Si terminan en ``|p|'' minúscula se pondrá un paréntesis
%   \emph{pequeño} alrededor del objeto sobre el que se esta realizando
%   una operación.
% \item Si terminan en ``|P|'' mayúscula se pondrá un paréntesis que
%   tendrá un \emph{tamaño ajustado al objeto}.
% \item Si terminan en ``|pE|'' se pondrá un paréntesis \emph{pequeño} alrededor de
%   toda la operación.
% \item Si terminan en ``|PE|'' se pondrá un paréntesis \emph{ajustado
%     al tamaño del objeto} alrededor de toda la operación
% \end{itemize}
% 
% \emph{Es decir, en Moodle se pintan las versiones con estrella del paquete original}.
% \subsection{Subíndices}
% \subsubsection{Subíndices y exponente}
% 
% \DescribeMacro{\LRidxE} El comando \cs{LRidxE} tiene 4
% argumentos,\;\cs{LRidxE}\marg{objeto}\marg{indIzda}\marg{indDcha}\marg{exponente},\; y
% pone un subíndice a cada lado del objeto (con exponente)
% \begin{center}
%   |\LRidxE{A}{1}{7}{*}| 
%   \hspace{1.2cm}
%   \fbox{$\LRidxE{A}{1}{7}{*}$}
% \end{center}
% 
% \DescribeMacro{\LidxE} El comando \cs{LidxE} tiene 3
% argumentos,\;\cs{LidxE}\marg{objeto}\marg{indIzda}\marg{exponente},\; y
% pone un subíndice a la izquierda del objeto (con exponente)
% \begin{center}
%   |\LidxE{A}{1}{*}| 
%   \hspace{1.2cm}
%   \fbox{$\LidxE{A}{1}{*}$}
% \end{center}
% 
% \DescribeMacro{\RidxE} El comando \cs{RidxE} tiene 3
% argumentos,\;\cs{RidxE}\marg{objeto}\marg{indDcha}\marg{exponente},\; y
% pone un a la derecha del objeto (con exponente)
% \begin{center}
%   |\RidxE{A}{7}{*}| 
%   \hspace{1.2cm}
%   \fbox{$\RidxE{A}{7}{*}$}
% \end{center}
% 
% \subsubsection{Solo subíndices}
% %%%%%%%%%%%%%%%%%%%%%%%%%%%%%%%%%%%%%%%%%%%%%%%%%%%%%%%%%%%%%%%%%%%%%%%%%%
% \DescribeMacro{\LRidx}
% \DescribeMacro{\LRidxp}
% \DescribeMacro{\LRidxP}
% \DescribeMacro{\LRidxpE}
% \DescribeMacro{\LRidxPE}
% El comando \cs{LRidx} tiene 3
% argumentos,\;\cs{LRidx}\marg{objeto}\marg{indIzda}\marg{indDcha},\; y
% pone un subíndice a cada lado del objeto
% \begin{center}
%   |\LRidx{A}{1}{7}| 
%   \hspace{1.2cm}
%   \fbox{$\LRidx{A}{1}{7}$}
% \end{center}
% \begin{center}
%   |\LRidxp{\Mat{A}}{1}{7} \LRidxP{\Mat{A}}{1}{7}|
%   \hspace{1.2cm}
%   \fbox{$\LRidxp {\Mat{A}}{1}{7}$}
%   \fbox{$\LRidxP {\Mat{A}}{1}{7}$}
% \end{center}
% \begin{center}
%   |\LRidxpE{\Mat{A}}{1}{7} \LRidxPE{\Mat{A}}{1}{7}|
%   \hspace{1.2cm}
%   \fbox{$\LRidxpE {\Mat{A}}{1}{7}$}
%   \fbox{$\LRidxPE {\Mat{A}}{1}{7}$}
% \end{center}
% 
% \DescribeMacro{\Lidx}
% \DescribeMacro{\Lidxp}
% \DescribeMacro{\LidxP}
% \DescribeMacro{\LidxpE}
% \DescribeMacro{\LidxPE}
% El comando \cs{Lidx} tiene 2
% argumentos,\;\cs{LidxE}\marg{objeto}\marg{indIzda},\; y pone un
% subíndice a la izquierda del objeto
% \begin{center}
%   |\Lidx{A}{1}| 
%   \hspace{1.2cm}
%   \fbox{$\Lidx{A}{1}$}
% \end{center}
% \begin{center}
%   |\Lidxp{\Mat{A}}{1} \LidxP{\Mat{A}}{1}|
%   \hspace{1.2cm}
%   \fbox{$\Lidxp {\Mat{A}}{1}$}
%   \fbox{$\LidxP {\Mat{A}}{1}$}
% \end{center}
% \begin{center}
%   |\LidxpE{\Mat{A}}{1} \LidxPE{\Mat{A}}{1}|
%   \hspace{1.2cm}
%   \fbox{$\LidxpE {\Mat{A}}{1}$}
%   \fbox{$\LidxPE {\Mat{A}}{1}$}
% \end{center}
% 
% \DescribeMacro{\Ridx}
% \DescribeMacro{\Ridxp}
% \DescribeMacro{\RidxP}
% \DescribeMacro{\RidxpE}
% \DescribeMacro{\RidxPE}
% El comando \cs{Lidx} tiene 2
% argumentos,\;\cs{LidxE}\marg{objeto}\marg{indIzda},\; y pone un
% subíndice a la derecha del objeto
% \begin{center}
%   |\Ridx{A}{7}| 
%   \hspace{1.2cm}
%   \fbox{$\Ridx{A}{7}$}
% \end{center}
% \begin{center}
%   |\Ridxp{\Mat{A}}{7} \RidxP{\Mat{A}}{7}|
%   \hspace{1.2cm}
%   \fbox{$\Ridxp {\Mat{A}}{7}$}
%   \fbox{$\RidxP {\Mat{A}}{7}$}
% \end{center}
% \begin{center}
%   |\RidxpE{\Mat{A}}{7} \RidxPE{\Mat{A}}{7}|
%   \hspace{1.2cm}
%   \fbox{$\RidxpE {\Mat{A}}{7}$}
%   \fbox{$\RidxPE {\Mat{A}}{7}$}
% \end{center}
% 
% \subsection{Operadores}
% \subsubsection{Conjugación y concatenación}
% 
% Definimos un operador con una barra ancha.\\
% \DescribeMacro{\widebar}
% El comando \cs{widebar} tiene 1 argumento,\;\cs{widebar}\marg{objeto},\;
% y pone una barra ancha sobre el \marg{objeto}.
% \begin{center}
%   |\widebar{x}| 
%   \hspace{1.2cm}
%   \fbox{$\widebar{x}$}
% \end{center}
% 
% Con dicha barra ancha denotaremos el operador conjugación:\\
% \DescribeMacro{\conj}
% El comando \cs{conj} tiene 1 argumento,\;\cs{conj}\marg{objeto},\;
% y pone una barra ancha sobre el \marg{objeto}.
% \begin{center}
%   |\conj{5+2i}| 
%   \hspace{1.2cm}
%   \fbox{$\conj{5+2i}$}
% \end{center}
% 
% Con el comando \cs{concat} denotaremos la concatenación de dos sistemas\\
% \DescribeMacro{\concat}
% El comando \cs{concat} no tiene argumentos,\;\cs{concat}.
% \begin{center}
%   |\concat| 
%   \hspace{1.2cm}
%   \fbox{$\concat$}
% \end{center}
% 
% \subsubsection{Norma y valor absoluto}
% \DescribeMacro{\norma}
% El comando \cs{norma} tiene 1 argumento,\;\cs{norma}\marg{objeto},\; y
% denota la norma del \marg{objeto}. Las dobles barras verticales se
% ajustan al tamaño del \marg{objeto}.
% \begin{center}
%   |\norma{ \int\limits_a^b h(x) dx }| 
%   \hspace{1.2cm}
%   \fbox{$\norma*{\int\limits_a^b h(x) dx}$}
% \end{center}
% 
% \DescribeMacro{\modulus}
% El comando \cs{modulus} tiene 1
% argumento,\;\cs{modulus}\marg{objeto},\; y denota el valor absoluto
% del \marg{objeto}. Las barras verticales se ajustan al tamaño del
% \marg{objeto}.
% \begin{center}
%   |\modulus{ \int\limits_a^b h(x) dx }| 
%   \hspace{1.2cm}
%   \fbox{$\modulus*{\int\limits_a^b h(x) dx}$}
% \end{center}
% 
% \subsubsection{Transposición}
% 
% \DescribeMacro{\T}
% El comando \cs{T} no tiene argumentos y denota el símbolo de la transposición.
% \begin{center}
%   |\T| 
%   \hspace{1.2cm}
%   \fbox{$\T$}
% \end{center}
% 
% \DescribeMacro{\Trans}
% \DescribeMacro{\Transp}
% \DescribeMacro{\TransP}
% \DescribeMacro{\TranspE}
% \DescribeMacro{\TransPE}
% El comando \cs{Trans} tiene 1 argumento,\;\cs{Trans}\marg{objeto},\; y
% denota la transposición del \marg{objeto}
% \begin{center}
%   |\Trans{\Mat{A}}| 
%   \hspace{1.2cm}
%   \fbox{$\Trans{\Mat{A}}$}
% \end{center}
% \begin{center}
%   |\Transp{\widehat{\Mat{A}}} \TransP{\widehat{\Mat{A}}}| 
%   \hspace{1.2cm}
%   \fbox{$\Transp*{\widehat{\Mat{A}}}$}
%   \fbox{$\TransP*{\widehat{\Mat{A}}}$}
% \end{center}
% \begin{center}
%   |\TranspE{\Mat{A}} \TransPE{\Mat{A}}| 
%   \hspace{1.2cm}
%   \fbox{$\TranspE*{\Mat{A}}$}
%   \fbox{$\TransPE*{\Mat{A}}$}
% \end{center}
% 
% \subsubsection{Inversa}
% 
% Me gusta que el signo negativo que indica la inversa sea ligeramente
% más corto que el habitual. Así logramos que las expresiones sean un
% poco más compactas.
% 
% \DescribeMacro{\minus}
% El comando \cs{minus} no tiene argumentos
% \begin{center}
%   |\minus| 
%   \hspace{1.2cm}
%   \fbox{$\minus$}
% \end{center}
% 
% \DescribeMacro{\Inv}
% \DescribeMacro{\Invp}
% \DescribeMacro{\InvP}
% \DescribeMacro{\InvpE}
% \DescribeMacro{\InvPE}
% Tiene 1 argumento,\;\cs{Inv}\marg{objeto},\; y denota el inverso del
% \marg{objeto}.
% \begin{center}
%   |\Inv{x}| 
%   \hspace{1.2cm}
%   \fbox{$\Inv{x}$}
% \end{center}
% \begin{center}
%   |\Invp{x} \InvP{\int\limits_a^b h(x) dx}| 
%   \hspace{1.2cm}
%   \fbox{$\Invp*{x}$}
%   \fbox{$\InvP*{\int\limits_a^b h(x) dx}$}
% \end{center}
% \begin{center}
%   |\InvpE{x} \InvPE{x}| 
%   \hspace{1.2cm}
%   \fbox{$\InvpE*{x}$}
%   \fbox{$\InvPE*{x}$}
% \end{center}
% 
% \subsubsection{Operador selector}
% 
% Denotaremos el operador selector con una barra vertical.
% 
% \DescribeMacro{\getItem}
% El comando \cs{getItem} no tiene argumentos
% \begin{center}
%   |\getItem| 
%   \hspace{1.2cm}
%   \fbox{$\getItem$}
% \end{center}
% 
% \DescribeMacro{\getitemL}
% El comando \cs{getitemL} tiene 1 argumento,\;\cs{getitemL}\marg{objeto}.
% \begin{center}
%   |\getitemL{i}| 
%   \hspace{1.2cm}
%   \fbox{$\getitemL{i}$}
% \end{center}
% 
% \DescribeMacro{\getitemR}
% El comando \cs{getitemR} tiene 1 argumento,\;\cs{getitemR}\marg{objeto}.
% \begin{center}
%   |\getitemR{j}| 
%   \hspace{1.2cm}
%   \fbox{$\getitemR{j}$}
% \end{center}
% 
% \paragraph{por la izquierda de un objeto}
% 
% \DescribeMacro{\elemL}
% \DescribeMacro{\elemLp}
% \DescribeMacro{\elemLP}
% \DescribeMacro{\elemLpE}
% \DescribeMacro{\elemLPE}
% El comando \cs{elemL} tiene 2
% argumentos,\;\cs{elemL}\marg{objeto}\marg{indice(s)},\; y denota la
% selección de elementos por la izquierda.
% \begin{center}
%   |\elemL{\Mat{A}}{i}| 
%   \hspace{1.2cm}
%   \fbox{$\elemL{\Mat{A}}{i}$}
% \end{center}
% \begin{center}
%   |\elemLp{\Mat{A}}{i} \elemLP{\Mat{A}}{i}| 
%   \hspace{1.2cm}
%   \fbox{$\elemLp*{\Mat{A}}{i}$}
%   \fbox{$\elemLP*{\Mat{A}}{i}$}
% \end{center}
% \begin{center}
%   |\elemLpE{\Mat{A}}{i} \elemLPE{\Mat{A}}{i}| 
%   \hspace{1.2cm}
%   \fbox{$\elemLpE*{\Mat{A}}{i}$}
%   \fbox{$\elemLPE*{\Mat{A}}{i}$}
% \end{center}
% 
% \paragraph{por la derecha de un objeto}
% 
% \DescribeMacro{\elemR}
% \DescribeMacro{\elemRp}
% \DescribeMacro{\elemRP}
% \DescribeMacro{\elemRpE}
% \DescribeMacro{\elemRPE}
% El comando \cs{elemR} tiene 2
% argumentos,\;\cs{elemR}\marg{objeto}\marg{indice(s)},\; y denota la
% selección de elementos por la derecha.
% \begin{center}
%   |\elemR{\Mat{A}}{j}| 
%   \hspace{1.2cm}
%   \fbox{$\elemR{\Mat{A}}{j}$}
% \end{center}
% \begin{center}
%   |\elemRp{\Mat{A}}{j} \elemRP{\Mat{A}}{j}| 
%   \hspace{1.2cm}
%   \fbox{$\elemRp*{\Mat{A}}{j}$}
%   \fbox{$\elemRP*{\Mat{A}}{j}$}
% \end{center}
% \begin{center}
%   |\elemRpE{\Mat{A}}{j} \elemRPE{\Mat{A}}{j}| 
%   \hspace{1.2cm}
%   \fbox{$\elemRpE*{\Mat{A}}{j}$}
%   \fbox{$\elemRPE*{\Mat{A}}{j}$}
% \end{center}
% 
% \paragraph{por ambos lados de un objeto}
% 
% \DescribeMacro{\elemLR}
% \DescribeMacro{\elemLRp}
% \DescribeMacro{\elemLRP}
% \DescribeMacro{\elemLRpE}
% \DescribeMacro{\elemLRPE}
% El comando \cs{elemLR} tiene 3
% argumentos,\;\cs{elemLR}\marg{objeto}\marg{indice(s)Izda}\marg{indice(s)Dcha},\;
% y denota la selección de elementos por ambos lados.
% \begin{center}
%   |\elemLR{\Mat{A}}{i}{j}| 
%   \hspace{1.2cm}
%   \fbox{$\elemLR{\Mat{A}}{i}{j}$}
% \end{center}
% \begin{center}
%   |\elemLRp{\Mat{A}}{i}{j} \elemLRP{\Mat{A}}{i}{j}| 
%   \hspace{1.2cm}
%   \fbox{$\elemLRp*{\Mat{A}}{i}{j}$}
%   \fbox{$\elemLRP*{\Mat{A}}{i}{j}$}
% \end{center}
% \begin{center}
%   |\elemLRpE{\Mat{A}}{i}{j} \elemLRPE{\Mat{A}}{i}{j}| 
%   \hspace{1.2cm}
%   \fbox{$\elemLRpE*{\Mat{A}}{i}{j}$}
%   \fbox{$\elemLRPE*{\Mat{A}}{i}{j}$}
% \end{center}
% 
% %%%%%%%%%%%%%%%%%%%%%%%%%%%%%%%%%%%%%%
% \paragraph{por la izquierda de un vector}
% 
% \DescribeMacro{\eleVL}
% \DescribeMacro{\eleVLp}
% \DescribeMacro{\eleVLP}
% \DescribeMacro{\eleVLpE}
% \DescribeMacro{\eleVLPE}
% El comando \cs{eleVL} tiene 2
% argumentos,\;\cs{eleVL}\marg{nombre}\marg{indice(s)},\; y denota la
% selección de elementos por la izquierda de un vector.
% \begin{center}
%   |\eleVL{a}{i}| 
%   \hspace{1.2cm}
%   \fbox{$\eleVL{a}{i}$}
% \end{center}
% \begin{center}
%   |\eleVLp{a}{i} \eleVLP{a}{i}| 
%   \hspace{1.2cm}
%   \fbox{$\eleVLp*{a}{i}$}
%   \fbox{$\eleVLP*{a}{i}$}
% \end{center}
% \begin{center}
%   |\eleVLpE{a}{i} \eleVLPE{a}{i}| 
%   \hspace{1.2cm}
%   \fbox{$\eleVLpE*{a}{i}$}
%   \fbox{$\eleVLPE*{a}{i}$}
% \end{center}
% 
% %%%%%%%%%%%%%%%%%%%%%%%%%%%%%%%%%%%%%%
% \paragraph{por la derecha de un vector}
% 
% \DescribeMacro{\eleVR}
% \DescribeMacro{\eleVRp}
% \DescribeMacro{\eleVRP}
% \DescribeMacro{\eleVRpE}
% \DescribeMacro{\eleVRPE}
% El comando \cs{eleVR} tiene 2
% argumentos,\;\cs{eleVL}\marg{nombre}\marg{indice(s)},\; y denota la
% selección de elementos por la derecha de un vector.
% \begin{center}
%   |\eleVR{a}{j}| 
%   \hspace{1.2cm}
%   \fbox{$\eleVR{a}{j}$}
% \end{center}
% \begin{center}
%   |\eleVRp{a}{j} \eleVRP{a}{j}| 
%   \hspace{1.2cm}
%   \fbox{$\eleVRp*{a}{j}$}
%   \fbox{$\eleVRP*{a}{j}$}
% \end{center}
% \begin{center}
%   |\eleVRpE{a}{j} \eleVRPE{a}{j}| 
%   \hspace{1.2cm}
%   \fbox{$\eleVRpE*{a}{j}$}
%   \fbox{$\eleVRPE*{a}{j}$}
% \end{center}
% 
% %%%%%%%%%%%%%%%%%%%%%%%%%%%%%%%%%%%%%%
% \paragraph{de filas de una matriz}
% 
% \DescribeMacro{\VectF}
% \DescribeMacro{\VectFp}
% \DescribeMacro{\VectFP}
% \DescribeMacro{\VectFpE}
% \DescribeMacro{\VectFPE}
% El comando \cs{VectF} tiene 2
% argumentos,\;\cs{VectF}\marg{nombre}\marg{indice(s)},\; y denota la
% selección de filas de una matriz
% \begin{center}
%   |\VectF{A}{i}| 
%   \hspace{1.2cm}
%   \fbox{$\VectF{A}{i}$}
% \end{center}
% \begin{center}
%   |\VectFp{A}{i} \VectFP{A}{i}| 
%   \hspace{1.2cm}
%   \fbox{$\VectFp*{A}{i}$}
%   \fbox{$\VectFP*{A}{i}$}
% \end{center}
% \begin{center}
%   |\VectFpE{A}{i} \VectFPE{A}{i}| 
%   \hspace{1.2cm}
%   \fbox{$\VectFpE*{A}{i}$}
%   \fbox{$\VectFPE*{A}{i}$}
% \end{center}
% 
% %%%%%%%%%%%%%%%%%%%%%%%%%%%%%%%%%%%%%%
% \paragraph{de columnas de una matriz}
% 
% \DescribeMacro{\VectC}
% \DescribeMacro{\VectCp}
% \DescribeMacro{\VectCP}
% \DescribeMacro{\VectCpE}
% \DescribeMacro{\VectCPE}
% El comando \cs{VectC} tiene 2
% argumentos,\;\cs{VectC}\marg{nombre}\marg{indice(s)},\; y denota la
% selección de filas de una matriz
% \begin{center}
%   |\VectC{A}{j}| 
%   \hspace{1.2cm}
%   \fbox{$\VectC{A}{j}$}
% \end{center}
% \begin{center}
%   |\VectCp{A}{j} \VectCP{A}{j}| 
%   \hspace{1.2cm}
%   \fbox{$\VectCp*{A}{j}$}
%   \fbox{$\VectCP*{A}{j}$}
% \end{center}
% \begin{center}
%   |\VectCpE{A}{j} \VectCPE{A}{j}| 
%   \hspace{1.2cm}
%   \fbox{$\VectCpE*{A}{j}$}
%   \fbox{$\VectCPE*{A}{j}$}
% \end{center}
% 
% 
% %%%%%%%%%%%%%%%%%%%%%%%%%%%%%%%%%%%%%%
% \paragraph{de elementos de una matriz}
% 
% \DescribeMacro{\eleM}
% \DescribeMacro{\eleMp}
% \DescribeMacro{\eleMP}
% \DescribeMacro{\eleMpE}
% \DescribeMacro{\eleMPE}
% El comando \cs{eleM} tiene 3
% argumentos,\;\cs{eleM}\marg{nombre}\marg{indice(s)Fil}\marg{indice(s)Col},\;
% y denota la selección de filas y columnas de una matriz
% \begin{center}
%   |\eleM{A}{i}{j}| 
%   \hspace{1.2cm}
%   \fbox{$\eleM{A}{i}{j}$}
% \end{center}
% \begin{center}
%   |\eleMp{A}{i}{j} \eleMP{A}{i}{j}| 
%   \hspace{1.2cm}
%   \fbox{$\eleMp*{A}{i}{j}$}
%   \fbox{$\eleMP*{A}{i}{j}$}
% \end{center}
% \begin{center}
%   |\eleMpE{A}{i}{j} \eleMpE{A}{i}{j}| 
%   \hspace{1.2cm}
%   \fbox{$\eleMpE*{A}{i}{j}$}
%   \fbox{$\eleMPE*{A}{i}{j}$}
% \end{center}
% 
% %%%%%%%%%%%%%%%%%%%%%%%%%%%%%%%%%%%%%%
% \paragraph{de elementos de una matriz transpuesta}
% 
% \DescribeMacro{\eleMT}
% El comando \cs{eleMT} tiene 3
% argumentos,\;\cs{eleMT}\marg{nombre}\marg{indice(s)Fil}\marg{indice(s)Col},\;
% y denota la selección de filas y columnas de una matriz
% \begin{center}
%   |\eleMT{A}{i}{j}| 
%   \hspace{1.2cm}
%   \fbox{$\eleMT{A}{i}{j}$}
% \end{center}
% 
% \DescribeMacro{\eleMTp}
% El comando \cs{eleMTp} tiene 3
% argumentos,\;\cs{eleMTp}\marg{nombre}\marg{indice(s)Fil}\marg{indice(s)Col},\;
% y denota la selección de filas y columnas de una matriz
% \begin{center}
%   |\eleMTp{A}{i}{j}| 
%   \hspace{1.2cm}
%   \fbox{$\eleMTp*{A}{i}{j}$}
% \end{center}
% 
% \DescribeMacro{\eleMTP}
% El comando \cs{eleMTP} tiene 3
% argumentos,\;\cs{eleMTP}\marg{nombre}\marg{indice(s)Fil}\marg{indice(s)Col},\;
% y denota la selección de filas y columnas de una matriz
% \begin{center}
%   |\eleMTP{A}{i}{j}| 
%   \hspace{1.2cm}
%   \fbox{$\eleMTP*{A}{i}{j}$}
% \end{center}
% 
% \DescribeMacro{\eleMTpE}
% El comando \cs{eleMTpE} tiene 3
% argumentos,\;\cs{eleMTpE}\marg{nombre}\marg{indice(s)Fil}\marg{indice(s)Col},\;
% y denota la selección de filas y columnas de una matriz
% \begin{center}
%   |\eleMTpE{A}{i}{j}| 
%   \hspace{1.2cm}
%   \fbox{$\eleMTpE*{A}{i}{j}$}
% \end{center}
% 
% \DescribeMacro{\eleMTPE}
% El comando \cs{eleMTPE} tiene 3
% argumentos,\;\cs{eleMTPE}\marg{nombre}\marg{indice(s)Fil}\marg{indice(s)Col},\;
% y denota la selección de filas y columnas de una matriz
% \begin{center}
%   |\eleMTPE{A}{i}{j}| 
%   \hspace{1.2cm}
%   \fbox{$\eleMTPE*{A}{i}{j}$}
% \end{center}
% 
% \subsubsection{Operaciones elementales}
% \label{sec:TransfElem}
% 
% Primero fijamos la notación de las operaciones elementales tipo I y II, los intercambios y las reordenaciones (o permutaciones).
% 
% \DescribeMacro{\su}
% El comando \cs{su} tiene 3
% argumentos,\;\cs{pe}\marg{escalar}\marg{índice}\marg{índice},\; e indica una
% transformación Tipo I.
% \begin{center}
%   |\su{a}{j}{k}| 
%   \hspace{1.2cm}
%   \fbox{$\su{a}{j}{k}$}
% \end{center}
% 
% \DescribeMacro{\pr}
% El comando \cs{pr} tiene 2
% argumento,\;\cs{pr}\marg{escalar}\marg{índice},\; e indica una
% transformación Tipo II.
% \begin{center}
%   |\pr{a}{k}| 
%   \hspace{1.2cm}
%   \fbox{$\pr{a}{k}$}
% \end{center}
% 
% \DescribeMacro{\pe}
% El comando \cs{pr} tiene 2
% argumento,\;\cs{pr}\marg{índice}\marg{índice},\; e indica un intercambio.
% \begin{center}
%   |\pe{i}{k}| 
%   \hspace{1.2cm}
%   \fbox{$\pe{i}{k}$}
% \end{center}
% 
% \DescribeMacro{\perm}
% El comando \cs{perm} no tiene 
% argumentos e indica un reordenamiento o permutación.
% \begin{center}
%   |\perm| 
%   \hspace{1.2cm}
%   \fbox{$\perm$}
% \end{center}
% 
% Usaremos letra griega tau como símbolo para denotar una operación
% elemental (o una secuencia de ellas).
% 
% \DescribeMacro{\TrEl}
% El comando \cs{TrEl} no tiene argumentos
% \begin{center}
%   |\TrEl| 
%   \hspace{1.2cm}
%   \fbox{$\TrEl$}
% \end{center}
% 
% %%%%%%%%%%%%%%%%%%%%%%%%%%%%%
% \DescribeMacro{\OpE}
% El comando \cs{OpE} tiene 1
% argumento,\;\cs{OpE}\marg{detalles},\; e indica una operación elemental.
% \begin{center}
%   |\OpE{xyz}| 
%   \hspace{1.2cm}
%   \fbox{$\OpE{xyz}$}
% \end{center}
% 
% \DescribeMacro{\OEsu}
% El comando \cs{OEsu} tiene 3
% argumentos,\;\cs{OEsu}\marg{num}\marg{índice}\marg{índice},\; e indica una operación elemental de Tipo I
% \begin{center}
%   |\OEsu{a}{j}{k}| 
%   \hspace{1.2cm}
%   \fbox{$\OEsu{a}{j}{k}$}
% \end{center}
% 
% \DescribeMacro{\OEpr}
% El comando \cs{OEpr} tiene 2
% argumentos,\;\cs{OEpr}\marg{num}\marg{índice},\; e indica una operación elemental de Tipo II
% \begin{center}
%   |\OEpr{a}{j}| 
%   \hspace{1.2cm}
%   \fbox{$\OEpr{a}{j}$}
% \end{center}
% 
% \DescribeMacro{\OEin}
% El comando \cs{OEin} tiene 2
% argumentos,\;\cs{OEin}\marg{índice}\marg{índice},\; e indica un intercambio de posición entre componentes
% \begin{center}
%   |\OEin{k}{j}| 
%   \hspace{1.2cm}
%   \fbox{$\OEin{k}{j}$}
% \end{center}
% 
% \DescribeMacro{\OEper}
% El comando \cs{OEper} no tiene 
% argumentos e indica un reordenamiento o permutación entre componentes
% \begin{center}
%   |\OEper| 
%   \hspace{1.2cm}
%   \fbox{$\OEper$}
% \end{center}
% 
% \DescribeMacro{\EOEsu}
% El comando \cs{EOEsu} tiene 3
% argumentos,\;\cs{EOEsu}\marg{num}\marg{índice}\marg{índice},\; e indica la operación espejo de una elemental de Tipo I
% \begin{center}
%   |\EOEsu{a}{j}{k}| 
%   \hspace{1.2cm}
%   \fbox{$\EOEsu{a}{j}{k}$}
% \end{center}
% 
% \DescribeMacro{\EOEpr}
% El comando \cs{EOEpr} tiene 2
% argumentos,\;\cs{EOEpr}\marg{num}\marg{índice},\; e indica la operación espejo de una elemental de Tipo II
% \begin{center}
%   |\EOEpr{a}{j}| 
%   \hspace{1.2cm}
%   \fbox{$\EOEpr{a}{j}$}
% \end{center}
% 
% \paragraph{Operaciones elementales generales}
% Desgraciadamente para el propósito de este paquete, las macros que
% definí para escribir el libro usan mayoritariamente argumentos
% opcionales, que aquí no se pueden usar. Cambiar las macros originales
% supondría modificar los archivos del libro, las transparencias de
% clase, los problemas propuestos, los exámenes pasados\dots{} demasiado
% trabajo. La alternativa que me queda tampoco me gusta, pero al menos
% no supone tanto trabajo. Dicha alternativa consiste en duplicar
% comandos, es decir, que por cada comando original (con argumentos
% opcionales) creemos otro comando que pinte los mismos símbolos pero
% sin argumentos opcionales (esta solución ya la he tomado con los
% comandos de notación de los conjuntos de números, de manera que para
% escribir \R[n] ahora tenemos |\R[n]| (el argumento opcional es el
% superíndice) o bien |\Rr^n| (que no tiene argumentos opcionales y que
% es lo que debemos usar al escribir preguntas para Moodle).
% 
% El criterio de nomenclatura que he adoptado ha sido repetir la letra
% del comando pero en minúscula (salvo en el caso de los complejos); es
% decir, los comandos definidos para el libro son: |\N|, |\Z|, |\R| y
% |\Cc| (debido a que |\C| ya es un comando del paquete
% \textsf{hyperref}). Así, que los nuevos comandos que he creado para
% duplicar los anteriores pero sin argumentos opcionales son |\Nn|,
% |\Zz|, |\Rr| y |\CC|.
% 
% Ahora tengo que pensar en un criterio análogo para que sea fácil pasar
% del comando original a duplicado sin argumentos opcionales. No lo
% tengo claro así que voy a probar con mantener los mismo nombres pero
% con una |d| delante para indicar que es el comando duplicado (no sé
% que tal resultará esta solución).
% 
% 
% \DescribeMacro{\dOEgE}
% El comando \cs{dOEgE} tiene 2
% argumentos,\;\cs{dOEgE}\marg{índice}\marg{exponente},\; e indica una
% operación elemental genérica con un exponente (y replica el
% comando |\OEg| que tiene argumentos opcionales)
% \begin{center}
%   |\dOEgE{}{} \dOEgE{k}{} \dOEgE{k}{*} \Oeg[k][*]| 
%   \hspace{1.2cm}
%   \fbox{$\dOEgE{}{} \dOEgE{k}{} \dOEgE{k}{*} \OEg[k][*]$}
% \end{center}
% 
% \DescribeMacro{\dOEg}
% El comando \cs{dOEg} tiene 1
% argumento,\;\cs{dOEg}\marg{índice},\; e indica una
% operación elemental genérica (y replica el
% comando |\OEg| que tiene argumentos opcionales)
% \begin{center}
%   |\dOEg{} \dOEg{k} \OEg[k]| 
%   \hspace{1.2cm}
%   \fbox{$\dOEg{} \dOEg{k} \OEg[k]$}
% \end{center}
% 
% También fijamos la notación para operación inversa, la operación espejo y el espejo de la inversa de una operación elemental
% 
% \DescribeMacro{\dEOEgE}
% El comando \cs{dEOEgE} tiene 2
% argumentos,\;\cs{dEOEgE}\marg{índice}\marg{exponente},\; e indica la
% operación espejo de una elemental genérica con un exponente (y replica
% el comando |\EOEg| que tiene argumentos opcionales)
% \begin{center}
%   |\dEOEgE{}{} \dEOEgE{k}{*} \EOEg[k][*]| 
%   \hspace{1.2cm}
%   \fbox{$\dEOEgE{}{} \dEOEgE{k}{*} \EOEg[k][*]$}
% \end{center}
% 
% \DescribeMacro{\dEOEg}
% El comando \cs{dEOEg} tiene 1
% argumento,\;\cs{dEOEgE}\marg{índice},\; e indica la
% operación espejo de una elemental genérica (y replica
% el comando |\EOEg| que tiene argumentos opcionales)
% \begin{center}
%   |\dEOEg{} \dEOEg{k} \EOEg[k]| 
%   \hspace{1.2cm}
%   \fbox{$\dEOEg{} \dEOEg{k} \EOEg[k]$}
% \end{center}
% 
% \DescribeMacro{\dInvOEg}
% El comando \cs{dInvOEg} tiene 1
% argumento,\;\cs{dInvOEgE}\marg{índice},\; e indica la 
% la inversa de una elemental genérica (y replica
% el comando |\InvOEg| que tiene argumentos opcionales)
% \begin{center}
%   |\dInvOEg{} \dInvOEg{k} \InvOEg[k]| 
%   \hspace{1.2cm}
%   \fbox{$\dInvOEg{} \dInvOEg{k} \InvOEg[k]$}
% \end{center}
% 
% \DescribeMacro{\dEInvOEg}
% El comando \cs{dEInvOEg} tiene 1
% argumento,\;\cs{dEInvOEgE}\marg{índice},\; e indica la 
% operación espejo de la inversa de una elemental genérica (y replica
% el comando |\EInvOEg| que tiene argumentos opcionales)
% \begin{center}
%   |\dEInvOEg{} \dEInvOEg{k} \EInvOEg[k]| 
%   \hspace{1.2cm}
%   \fbox{$\dEInvOEg{} \dEInvOEg{k} \EInvOEg[k]$}
% \end{center}
% 
% \DescribeMacro{\dSOEgE}
% El comando \cs{dSOEgE} tiene 3
% argumento3,\;\cs{dSOEgE}\marg{indiceInic}\marg{indiceFin}\marg{exponente},\; e indica una sucesión de operaciones elementales genéricas con exponente
% \begin{center}
%   |\dSOEgE{j}{k}{*} \SOEg[j][k][*]| 
%   \hspace{1.2cm}
%   \fbox{$\dSOEgE{j}{k}{*} \SOEg[j][k][*]$}
% \end{center}
% 
% \DescribeMacro{\dSOEg}
% El comando \cs{dSOEg} tiene 2
% argumento3,\;\cs{dSOEg}\marg{indiceInic}\marg{indiceFin},\; e indica una sucesión de operaciones elementales genéricas 
% \begin{center}
%   |\dSOEg{j}{k} \SOEg[j][k]| 
%   \hspace{1.2cm}
%   \fbox{$\dSOEg{j}{k} \SOEg[j][k]$}
% \end{center}
% 
% \subsubsection{Transformaciones elementales}
% 
% \paragraph{Transf. elemental aplicada la izquierda o derecha de un objeto}
% 
% \DescribeMacro{\TESF}
% \DescribeMacro{\TESFp}
% \DescribeMacro{\TESFP}
% \DescribeMacro{\TESFpE}
% \DescribeMacro{\TESFPE}
% El comando \cs{TESF} tiene 4
% argumentos,\;\cs{TESF}\marg{escalar}\marg{índice}\marg{índice}\marg{objeto},\;
% e indica una transformación elemental de Tipo  I por la izquierda del objeto.
% \begin{center}
%   |\TESF{\lambda}{i}{j}{\Mat{A}}|
%   \hspace{0.2cm}
%   \fbox{$\TESF{\lambda}{i}{j}{\Mat{A}}$}
% \end{center}
% \begin{center}
%   |\TESFp{\lambda}{i}{j}{\SV{A}} \TESFP{\lambda}{i}{j}{\SV{A}}|
%   \hspace{0.2cm}
%   \fbox{$\TESFp*{\lambda}{i}{j}{\SV{A}}$}
%   \fbox{$\TESFP*{\lambda}{i}{j}{\SV{A}}$}
% \end{center}
% \begin{center}
%   |\TESFpE{\lambda}{i}{j}{\SV{A}} \TESFPE{\lambda}{i}{j}{\SV{A}}|
%   \hspace{0.2cm}
%   \fbox{$\TESFpE*{\lambda}{i}{j}{\SV{A}}$}
%   \fbox{$\TESFPE*{\lambda}{i}{j}{\SV{A}}$}
% \end{center}
% 
% \DescribeMacro{\TESC}
% \DescribeMacro{\TESCp}
% \DescribeMacro{\TESCP}
% \DescribeMacro{\TESCpE}
% \DescribeMacro{\TESCPE}
% El comando \cs{TESC} tiene 4
% argumentos,\;\cs{TESC}\marg{escalar}\marg{índice}\marg{índice}\marg{objeto},\;
% e indica una transformación elemental de Tipo I por la derecha del objeto.
% \begin{center}
%   |\TESC{\lambda}{i}{j}{\Mat{A}}|
%   \hspace{1.2cm}
%   \fbox{$\TESC{\lambda}{i}{j}{\Mat{A}}$}
% \end{center}
% \begin{center}
%   |\TESCp{\lambda}{i}{j}{\SV{A}} \TESCP{\lambda}{i}{j}{\SV{A}}|
%   \hspace{0.2cm}
%   \fbox{$\TESCp*{\lambda}{i}{j}{\SV{A}}$}
%   \fbox{$\TESCP*{\lambda}{i}{j}{\SV{A}}$}
% \end{center}
% \begin{center}
%   |\TESCpE{\lambda}{i}{j}{\SV{A}} \TESCPE{\lambda}{i}{j}{\SV{A}}|
%   \hspace{0.2cm}
%   \fbox{$\TESCpE*{\lambda}{i}{j}{\SV{A}}$}
%   \fbox{$\TESCPE*{\lambda}{i}{j}{\SV{A}}$}
% \end{center}
% 
% %%%%%%%%%%%%%%%%%%%%%%%%%%%%%%%%%%%%%%%%%%%%%%%%%%%%%%%%
% 
% \DescribeMacro{\TEPF}
% \DescribeMacro{\TEPFp}
% \DescribeMacro{\TEPFP}
% \DescribeMacro{\TEPFpE}
% \DescribeMacro{\TEPFPE}
% El comando \cs{TEPF} tiene 3
% argumentos,\;\cs{TEPF}\marg{escalar}\marg{índice}\marg{objeto},\;
% e indica una transformación elemental de Tipo II por la izquierda del objeto.
% \begin{center}
%   |\TEPF{\lambda}{i}{\Mat{A}}|
%   \hspace{1.2cm}
%   \fbox{$\TEPF{\lambda}{i}{\Mat{A}}$}
% \end{center}
% \begin{center}
%   |\TEPFp{\lambda}{i}{\Mat{A}} \TEPFP{\lambda}{i}{\Mat{A}}|
%   \hspace{0.2cm}
%   \fbox{$\TEPFp*{\lambda}{i}{\Mat{A}}$}
%   \fbox{$\TEPFP*{\lambda}{i}{\Mat{A}}$}
% \end{center}
% \begin{center}
%   |\TEPFpE{\lambda}{i}{\Mat{A}} \TEPFPE{\lambda}{i}{\Mat{A}}|
%   \hspace{0.2cm}
%   \fbox{$\TEPFpE*{\lambda}{i}{\Mat{A}}$}
%   \fbox{$\TEPFPE*{\lambda}{i}{\Mat{A}}$}
% \end{center}
% 
% 
% 
% \DescribeMacro{\TEPC}
% \DescribeMacro{\TEPCp}
% \DescribeMacro{\TEPCP}
% \DescribeMacro{\TEPCpE}
% \DescribeMacro{\TEPCPE}
% El comando \cs{TEPC} tiene 3
% argumentos,\;\cs{TEPC}\marg{escalar}\marg{índice}\marg{objeto},\;
% e indica una transformación elemental de Tipo II por la derecha del objeto.
% \begin{center}
%   |\TEPC{\lambda}{j}{\Mat{A}}|
%   \hspace{1.2cm}
%   \fbox{$\TEPC{\lambda}{j}{\Mat{A}}$}
% \end{center}
% \begin{center}
%   |\TEPCp{\lambda}{j}{\Mat{A}} \TEPCP{\lambda}{j}{\Mat{A}}|
%   \hspace{1.2cm}
%   \fbox{$\TEPCp*{\lambda}{j}{\Mat{A}}$}
%   \fbox{$\TEPCP*{\lambda}{j}{\Mat{A}}$}
% \end{center}
% \begin{center}
%   |\TEPCpE{\lambda}{j}{\Mat{A}} \TEPCPE{\lambda}{j}{\Mat{A}}|
%   \hspace{1.2cm}
%   \fbox{$\TEPCpE*{\lambda}{j}{\Mat{A}}$}
%   \fbox{$\TEPCPE*{\lambda}{j}{\Mat{A}}$}
% \end{center}
% 
% %%%%%%%%%%%%%%%%%%%%%%%%%%%%%%%%%%%%%%%%%%%%%%%%%%%%%%%%
% 
% \DescribeMacro{\TEIF}
% \DescribeMacro{\TEIFp}
% \DescribeMacro{\TEIFP}
% \DescribeMacro{\TEIFpE}
% \DescribeMacro{\TEIFPE}
% El comando \cs{TEIF} tiene 3
% argumentos,\;\cs{TEIF}\marg{índice}\marg{índice}\marg{objeto},\;
% e indica un intercambio por la izquierda del objeto.
% \begin{center}
%   |\TEIF{i}{j}{\Mat{A}}|
%   \hspace{1.2cm}
%   \fbox{$\TEIF{i}{j}{\Mat{A}}$}
% \end{center}
% \begin{center}
%   |\TEIFp{i}{j}{\Mat{A}} \TEIFP{i}{j}{\Mat{A}}|
%   \hspace{1.2cm}
%   \fbox{$\TEIFp*{i}{j}{\Mat{A}}$}
%   \fbox{$\TEIFP*{i}{j}{\Mat{A}}$}
% \end{center}
% \begin{center}
%   |\TEIFpE{i}{j}{\Mat{A}} \TEIFPE{i}{j}{\Mat{A}}|
%   \hspace{1.2cm}
%   \fbox{$\TEIFpE*{i}{j}{\Mat{A}}$}
%   \fbox{$\TEIFPE*{i}{j}{\Mat{A}}$}
% \end{center}
% 
% \DescribeMacro{\TEIC}
% \DescribeMacro{\TEICp}
% \DescribeMacro{\TEICP}
% \DescribeMacro{\TEICpE}
% \DescribeMacro{\TEICPE}
% El comando \cs{TEIC} tiene 3
% argumentos,\;\cs{TEIC}\marg{índice}\marg{índice}\marg{objeto},\;
% e indica un intercambio por la derecha del objeto.
% \begin{center}
%   |\TEIC{i}{j}{\Mat{A}}|
%   \hspace{1.2cm}
%   \fbox{$\TEIC{i}{j}{\Mat{A}}$}
% \end{center}
% \begin{center}
%   |\TEICp{i}{j}{\Mat{A}} \TEICP{i}{j}{\Mat{A}}|
%   \hspace{1.2cm}
%   \fbox{$\TEICp*{i}{j}{\Mat{A}}$}
%   \fbox{$\TEICP*{i}{j}{\Mat{A}}$}
% \end{center}
% \begin{center}
%   |\TEICpE{i}{j}{\Mat{A}} \TEICPE{i}{j}{\Mat{A}}|
%   \hspace{1.2cm}
%   \fbox{$\TEICpE*{i}{j}{\Mat{A}}$}
%   \fbox{$\TEICPE*{i}{j}{\Mat{A}}$}
% \end{center}
% 
% %%%%%%%%%%%%%%%%%%%%%%%%%%%%%%%%%%%%%%%%%%%%%%%%%%%%%%%%
% 
% \DescribeMacro{\Mint}
% El comando \cs{Mint} tiene 2
% argumentos,\;\cs{Mint}\marg{índice}\marg{índice},\;
% e indica una matriz intercambio.
% \begin{center}
%   |\Mint{i}{j}|
%   \hspace{1.2cm}
%   \fbox{$\Mint{i}{j}$}
% \end{center}
% 
% \DescribeMacro{\MintT}
% El comando \cs{MintT} tiene 2
% argumentos,\;\cs{MintT}\marg{índice}\marg{índice},\;
% e indica una matriz intercambio (filas).
% \begin{center}
%   |\MintT{i}{j}|
%   \hspace{1.2cm}
%   \fbox{$\MintT{i}{j}$}
% \end{center}
% 
% \DescribeMacro{\PF}
% El comando \cs{PF} tiene 1
% argumento,\;\cs{PF}\marg{objeto},\;
% e indica una permutación de los elementos de un objeto por la izquierda.
% \begin{center}
%   |\PF{\Mat{A}}|
%   \hspace{1.2cm}
%   \fbox{$\PF{\Mat{A}}$}
% \end{center}
% 
% \DescribeMacro{\PC}
% El comando \cs{PC} tiene 1
% argumento,\;\cs{PC}\marg{objeto},\;
% e indica una permutación de los elementos de un objeto por la derecha.
% \begin{center}
%   |\PC{\Mat{A}}|
%   \hspace{1.2cm}
%   \fbox{$\PC{\Mat{A}}$}
% \end{center}
% 
% \DescribeMacro{\MP}
% El comando \cs{MP} no tiene argumentos e indica
% una matriz permutación.
% \begin{center}
%   |\MP|
%   \hspace{1.2cm}
%   \fbox{$\MP$}
% \end{center}
% 
% \DescribeMacro{\MPT}
% El comando \cs{MPT} no tiene argumentos e indica
% una matriz permutación.
% \begin{center}
%   |\MPT|
%   \hspace{1.2cm}
%   \fbox{$\MPT$}
% \end{center}
% 
% %%%%%%%%%%%%%%%%%%%%%%%%%%%%%%%%%%%%%%%%%%%%%%%%%%%%%%%%
% \paragraph{Sucesiones indiciadas de Transf. elementales} por la
% izquierda, o por la derecha, o por ambos lados.
% 
% \DescribeMacro{\SITEF}
% \DescribeMacro{\SITEFp}
% \DescribeMacro{\SITEFP}
% \DescribeMacro{\SITEFpE}
% \DescribeMacro{\SITEFPE}
% El comando \cs{SITEF} tiene 3
% argumentos,\;\cs{SITEF}\marg{indInic}\marg{indFinal}\marg{objeto},\; e indica una
% sucesión de transformaciones elementales genéricas por la izquierda del \marg{objeto}.
% \begin{center}
%   |\SITEF{j}{k}{\Mat{A}}| 
%   \hspace{1.2cm}
%   \fbox{$\SITEF{j}{k}{\Mat{A}}$}
% \end{center}
% \begin{center}
%   |\SITEFp{j}{k}{\Mat{A}} \SITEFP{j}{k}{\Mat{A}}| 
%   \hspace{0.2cm}
%   \fbox{$\SITEFp*{j}{k}{\Mat{A}}$}
%   \fbox{$\SITEFP*{j}{k}{\Mat{A}}$}
% \end{center}
% \begin{center}
%   |\SITEFpE{j}{k}{\Mat{A}} \SITEFpE{j}{k}{\Mat{A}}| 
%   \hspace{0.2cm}
%   \fbox{$\SITEFpE*{j}{k}{\Mat{A}}$}
%   \fbox{$\SITEFPE*{j}{k}{\Mat{A}}$}
% \end{center}
% 
% \DescribeMacro{\SITEC}
% \DescribeMacro{\SITECp}
% \DescribeMacro{\SITECP}
% \DescribeMacro{\SITECpE}
% \DescribeMacro{\SITECPE}
% El comando \cs{SITEC} tiene 3
% argumentos,\;\cs{SITEC}\marg{indInic}\marg{indFinal}\marg{objeto},\; e indica una
% sucesión de transformaciones elementales genéricas por la derecha del \marg{objeto}.
% \begin{center}
%   |\SITEC{j}{k}{\Mat{A}}| 
%   \hspace{1.2cm}
%   \fbox{$\SITEC{j}{k}{\Mat{A}}$}
% \end{center}
% \begin{center}
%   |\SITECp{j}{k}{\Mat{A}} \SITECP{j}{k}{\Mat{A}}| 
%   \hspace{0.2cm}
%   \fbox{$\SITECp*{j}{k}{\Mat{A}}$}
%   \fbox{$\SITECP*{j}{k}{\Mat{A}}$}
% \end{center}
% \begin{center}
%   |\SITECpE{j}{k}{\Mat{A}} \SITECPE{j}{k}{\Mat{A}}| 
%   \hspace{0.2cm}
%   \fbox{$\SITECpE*{j}{k}{\Mat{A}}$}
%   \fbox{$\SITECPE*{j}{k}{\Mat{A}}$}
% \end{center}
% 
% \DescribeMacro{\SITEFC}
% \DescribeMacro{\SITEFCp}
% \DescribeMacro{\SITEFCP}
% \DescribeMacro{\SITEFCpE}
% \DescribeMacro{\SITEFCPE}
% El comando \cs{SITEFC} tiene 3
% argumentos,\;\cs{SITEFC}\marg{indInic}\marg{indFinal}\marg{objeto},\;
% e indica una sucesión de transformaciones elementales genéricas por la
% derecha y la izquierda del \marg{objeto} (fíjese en el orden de los
% índices a cada lado).
% \begin{center}
%   |\SITEFC{j}{k}{\Mat{A}}| 
%   \hspace{1.2cm}
%   \fbox{$\SITEFC{j}{k}{\Mat{A}}$}
% \end{center}
% \begin{center}
%   |\SITEFCp{j}{k}{\Mat{A}} \SITEFCP{j}{k}{\Mat{A}}| 
%   \hspace{0.2cm}
%   \fbox{$\SITEFCp*{j}{k}{\Mat{A}}$}
%   \fbox{$\SITEFCP*{j}{k}{\Mat{A}}$}
% \end{center}
% \begin{center}
%   |\SITEFCpE{j}{k}{\Mat{A}} \SITEFCPE{j}{k}{\Mat{A}}| 
%   \hspace{0.2cm}
%   \fbox{$\SITEFCpE*{j}{k}{\Mat{A}}$}
%   \fbox{$\SITEFCPE*{j}{k}{\Mat{A}}$}
% \end{center}
% 
% \DescribeMacro{\SITEFCR}
% \DescribeMacro{\SITEFCRp}
% \DescribeMacro{\SITEFCRP}
% \DescribeMacro{\SITEFCRpE}
% \DescribeMacro{\SITEFCRPE}
% El comando \cs{SITEFCR} tiene 3
% argumentos,\;\cs{SITEFCR}\marg{indInic}\marg{indFinal}\marg{objeto},\;
% e indica una sucesión de transformaciones elementales genéricas por la
% derecha y la izquierda del \marg{objeto} (fíjese en el orden de los
% índices a cada lado).
% \begin{center}
%   |\SITEFCR{j}{k}{\Mat{A}}| 
%   \hspace{1.2cm}
%   \fbox{$\SITEFCR{j}{k}{\Mat{A}}$}
% \end{center}
% \begin{center}
%   |\SITEFCRp{j}{k}{\Mat{A}} \SITEFCRp{j}{k}{\Mat{A}}| 
%   \hspace{0.2cm}
%   \fbox{$\SITEFCRp*{j}{k}{\Mat{A}}$}
%   \fbox{$\SITEFCRP*{j}{k}{\Mat{A}}$}
% \end{center}
% \begin{center}
%   |\SITEFCRpE{j}{k}{\Mat{A}} \SITEFCRPE{j}{k}{\Mat{A}}| 
%   \hspace{0.2cm}
%   \fbox{$\SITEFCRpE*{j}{k}{\Mat{A}}$}
%   \fbox{$\SITEFCRPE*{j}{k}{\Mat{A}}$}
% \end{center}
% 
% %%%%%%%%%%%%%%%%%%%%%%%%%%%%%%%%%%%%%%
% 
% \paragraph{Transf. elemental genérica aplicada a la izquierda de un objeto (funciones duplicadas sin argumentos opcionales.} Cuando la aplicamos a la izquierda de una matriz corresponde a una transformación de sus filas
% 
% \DescribeMacro{\dTEEF}
% \DescribeMacro{\dTEEFp}
% \DescribeMacro{\dTEEFP}
% \DescribeMacro{\dTEEFpE}
% \DescribeMacro{\dTEEFPE}
% El comando \cs{dTEEF} tiene 3
% argumentos,\;\cs{dTEEF}\marg{índice}\marg{exponente}\marg{objeto},\; e
% indica una transformación elemental genérica (con exponente) por la
% izquierda del objeto.
% \begin{center}
%   |\dTEEF{}{}{\SV{A}} \dTEEF{2}{}{\SV{A}} \dTEEF{2}{*}{\SV{A}}|
%   \hspace{1.2cm}
%   \fbox{$\dTEEF{}{}{\SV{A}}$} \fbox{$\dTEEF{2}{}{\SV{A}}$} \fbox{$\dTEEF{2}{*}{\SV{A}}$}
% \end{center}
% 
% \begin{center}
%   |\dTEEFp{}{}{A} \dTEEFp{2}{}{A} \dTEEFp{2}{*}{A}|
%   \hspace{1.2cm}
%   \fbox{$\dTEEFp{}{}{A}$} \fbox{$\dTEEFp{2}{}{A}$} \fbox{$\dTEEFp{2}{*}{A}$}
% \end{center}
% 
% \begin{center}
%   |\dTEEFP{}{}{A} \dTEEFP{2}{}{A} \dTEEFP{2}{*}{A}|
%   \hspace{1.2cm}
%   \fbox{$\dTEEFP{}{}{A}$} \fbox{$\dTEEFP{2}{}{A}$} \fbox{$\dTEEFP{2}{*}{A}$}
% \end{center}
% 
% \begin{center}
%   |\dTEEFpE{}{}{A} \dTEEFpE{2}{}{A} \dTEEFpE{2}{*}{A}|
%   \hspace{1.2cm}
%   \fbox{$\dTEEFpE{}{}{A}$} \fbox{$\dTEEFpE{2}{}{A}$} \fbox{$\dTEEFpE{2}{*}{A}$}
% \end{center}
% 
% \begin{center}
%   |\dTEEFPE{}{}{A} \dTEEFPE{2}{}{A} \dTEEFPE{2}{*}{A}|
%   \hspace{1.2cm}
%   \fbox{$\dTEEFPE{}{}{A}$} \fbox{$\dTEEFPE{2}{}{A}$} \fbox{$\dTEEFPE{2}{*}{A}$}
% \end{center}
% 
% 
% \DescribeMacro{\dTEF}
% \DescribeMacro{\dTEFp}
% \DescribeMacro{\dTEFP}
% \DescribeMacro{\dTEFpE}
% \DescribeMacro{\dTEFPE}
% El comando \cs{dTEF} tiene 2
% argumentos,\;\cs{dTEF}\marg{índice}\marg{objeto},\;
% e indica una transformación elemental genérica por la izquierda del objeto.
% \begin{center}
%   |\dTEF{}{\Mat{A}} \dTEF{2}{\Mat{A}}|
%   \hspace{1.2cm}
%   \fbox{$\dTEF{}{\Mat{A}}$} \fbox{$\dTEF{2}{\Mat{A}}$}
% \end{center}
% 
% \begin{center}
%   |\dTEFpE{}{\Mat{A}} \dTEFpE{2}{\Mat{A}}|
%   \hspace{1.2cm}
%   \fbox{$\dTEFpE{}{\Mat{A}}$} \fbox{$\dTEFpE{2}{\Mat{A}}$}
% \end{center}
% 
% \begin{center}
%   |\dTEFPE{}{\Mat{A}} \dTEFPE{2}{\Mat{A}}|
%   \hspace{1.2cm}
%   \fbox{$\dTEFPE{}{\Mat{A}}$} \fbox{$\dTEFPE{2}{\Mat{A}}$}
% \end{center}
% 
% \DescribeMacro{\dETEF}
% \DescribeMacro{\dETEFp}
% \DescribeMacro{\dETEFP}
% \DescribeMacro{\dETEFpE}
% \DescribeMacro{\dETEFPE}
% El comando \cs{dETEF} tiene 2
% argumentos,\;\cs{dETEF}\marg{índice}\marg{objeto},\;
% e indica una transformación elemental espejo genérica por la izquierda del objeto.
% \begin{center}
%   |\dETEF{}{\Mat{A}} \dETEF{2}{\Mat{A}}|
%   \hspace{1.2cm}
%   \fbox{$\dETEF{}{\Mat{A}}$} \fbox{$\dETEF{2}{\Mat{A}}$}
% \end{center}
% 
% \begin{center}
%   |\dETEFp{}{\Mat{A}} \dETEFp{2}{\Mat{A}}|
%   \hspace{1.2cm}
%   \fbox{$\dETEFp{}{\Mat{A}}$} \fbox{$\dETEFp{2}{\Mat{A}}$}
% \end{center}
% 
% \begin{center}
%   |\dETEFP{}{\Mat{A}} \dETEFP{2}{\Mat{A}}|
%   \hspace{1.2cm}
%   \fbox{$\dETEFP{}{\Mat{A}}$} \fbox{$\dETEFP{2}{\Mat{A}}$}
% \end{center}
% 
% \begin{center}
%   |\dETEFpE{}{\Mat{A}} \dETEFpE{2}{\Mat{A}}|
%   \hspace{1.2cm}
%   \fbox{$\dETEFpE{}{\Mat{A}}$} \fbox{$\dETEFpE{2}{\Mat{A}}$}
% \end{center}
% 
% \begin{center}
%   |\dETEFPE{}{\Mat{A}} \dETEFPE{2}{\Mat{A}}|
%   \hspace{1.2cm}
%   \fbox{$\dETEFPE{}{\Mat{A}}$} \fbox{$\dETEFPE{2}{\Mat{A}}$}
% \end{center}
% 
% \DescribeMacro{\dInvTEF}
% \DescribeMacro{\dInvTEFp}
% \DescribeMacro{\dInvTEFP}
% \DescribeMacro{\dInvTEFpE}
% \DescribeMacro{\dInvTEFPE}
% El comando \cs{dInvTEF} tiene 2
% argumentos,\;\cs{dInvTEF}\marg{índice}\marg{objeto},\;
% e indica una transformación elemental espejo inversa genérica por la izquierda del objeto.
% \begin{center}
%   |\dInvTEF{}{\Mat{A}} \dInvTEF{2}{\Mat{A}}|
%   \hspace{1.2cm}
%   \fbox{$\dInvTEF{}{\Mat{A}}$} \fbox{$\dInvTEF{2}{\Mat{A}}$}
% \end{center}
% 
% \begin{center}
%   |\dInvTEFp{}{\Mat{A}} \dInvTEFp{2}{\Mat{A}}|
%   \hspace{1.2cm}
%   \fbox{$\dInvTEFp{}{\Mat{A}}$} \fbox{$\dInvTEFp{2}{\Mat{A}}$}
% \end{center}
% 
% \begin{center}
%   |\dInvTEFP{}{\Mat{A}} \dInvTEFP{2}{\Mat{A}}|
%   \hspace{1.2cm}
%   \fbox{$\dInvTEFP{}{\Mat{A}}$} \fbox{$\dInvTEFP{2}{\Mat{A}}$}
% \end{center}
% 
% \begin{center}
%   |\dInvTEFpE{}{\Mat{A}} \dInvTEFpE{2}{\Mat{A}}|
%   \hspace{1.2cm}
%   \fbox{$\dInvTEFpE{}{\Mat{A}}$} \fbox{$\dInvTEFpE{2}{\Mat{A}}$}
% \end{center}
% 
% \begin{center}
%   |\dInvTEFPE{}{\Mat{A}} \dInvTEFPE{2}{\Mat{A}}|
%   \hspace{1.2cm}
%   \fbox{$\dInvTEFPE{}{\Mat{A}}$} \fbox{$\dInvTEFPE{2}{\Mat{A}}$}
% \end{center}
% 
% \DescribeMacro{\dEInvTEF}
% \DescribeMacro{\dEInvTEFp}
% \DescribeMacro{\dEInvTEFP}
% \DescribeMacro{\dEInvTEFpE}
% \DescribeMacro{\dEInvTEFPE}
% El comando \cs{dEInvTEF} tiene 2
% argumentos,\;\cs{dEInvTEF}\marg{índice}\marg{objeto},\;
% e indica una transformación elemental espejo inversa genérica por la izquierda del objeto.
% \begin{center}
%   |\dEInvTEF{}{\Mat{A}} \dEInvTEF{2}{\Mat{A}}|
%   \hspace{1.2cm}
%   \fbox{$\dEInvTEF{}{\Mat{A}}$} \fbox{$\dEInvTEF{2}{\Mat{A}}$}
% \end{center}
% 
% \begin{center}
%   |\dEInvTEFp{}{\Mat{A}} \dEInvTEFp{2}{\Mat{A}}|
%   \hspace{1.2cm}
%   \fbox{$\dEInvTEFp{}{\Mat{A}}$} \fbox{$\dEInvTEFp{2}{\Mat{A}}$}
% \end{center}
% 
% \begin{center}
%   |\dEInvTEFP{}{\Mat{A}} \dEInvTEFP{2}{\Mat{A}}|
%   \hspace{1.2cm}
%   \fbox{$\dEInvTEFP{}{\Mat{A}}$} \fbox{$\dEInvTEFP{2}{\Mat{A}}$}
% \end{center}
% 
% \begin{center}
%   |\dEInvTEFpE{}{\Mat{A}} \dEInvTEFpE{2}{\Mat{A}}|
%   \hspace{1.2cm}
%   \fbox{$\dEInvTEFpE{}{\Mat{A}}$} \fbox{$\dEInvTEFpE{2}{\Mat{A}}$}
% \end{center}
% 
% \begin{center}
%   |\dEInvTEFPE{}{\Mat{A}} \dEInvTEFPE{2}{\Mat{A}}|
%   \hspace{1.2cm}
%   \fbox{$\dEInvTEFPE{}{\Mat{A}}$} \fbox{$\dEInvTEFPE{2}{\Mat{A}}$}
% \end{center}
% 
% 
% \paragraph{Transf. elemental aplicada la derecha de un objeto (funciones duplicadas sin argumentos opcionales.} Cuando la aplicamos a la derecha de una matriz corresponde a una transformación de sus columnas
% 
% \DescribeMacro{\dTEEC}
% \DescribeMacro{\dTEECp}
% \DescribeMacro{\dTEECP}
% \DescribeMacro{\dTEECpE}
% \DescribeMacro{\dTEECPE}
% El comando \cs{dTEEC} tiene 3
% argumentos,\;\cs{dTEEC}\marg{índice}\marg{exponente}\marg{objeto},\; e
% indica una transformación elemental genérica (con exponente) por la
% derecha del objeto.
% \begin{center}
%   |\dTEEC{}{}{\SV{A}} \dTEEC{2}{}{\SV{A}} \dTEEC{2}{*}{\SV{A}}|
%   \hspace{1.2cm}
%   \fbox{$\dTEEC{}{}{\SV{A}}$} \fbox{$\dTEEC{2}{}{\SV{A}}$} \fbox{$\dTEEC{2}{*}{\SV{A}}$}
% \end{center}
% 
% \begin{center}
%   |\dTEECp{}{}{A} \dTEECp{2}{}{A} \dTEECp{2}{*}{A}|
%   \hspace{1.2cm}
%   \fbox{$\dTEECp{}{}{A}$} \fbox{$\dTEECp{2}{}{A}$} \fbox{$\dTEECp{2}{*}{A}$}
% \end{center}
% 
% \begin{center}
%   |\dTEECP{}{}{A} \dTEECP{2}{}{A} \dTEECP{2}{*}{A}|
%   \hspace{1.2cm}
%   \fbox{$\dTEECP{}{}{A}$} \fbox{$\dTEECP{2}{}{A}$} \fbox{$\dTEECP{2}{*}{A}$}
% \end{center}
% 
% \begin{center}
%   |\dTEECpE{}{}{A} \dTEECpE{2}{}{A} \dTEECpE{2}{*}{A}|
%   \hspace{1.2cm}
%   \fbox{$\dTEECpE{}{}{A}$} \fbox{$\dTEECpE{2}{}{A}$} \fbox{$\dTEECpE{2}{*}{A}$}
% \end{center}
% 
% \begin{center}
%   |\dTEECPE{}{}{A} \dTEECPE{2}{}{A} \dTEECPE{2}{*}{A}|
%   \hspace{1.2cm}
%   \fbox{$\dTEECPE{}{}{A}$} \fbox{$\dTEECPE{2}{}{A}$} \fbox{$\dTEECPE{2}{*}{A}$}
% \end{center}
% 
% \DescribeMacro{\dTEC}
% \DescribeMacro{\dTECp}
% \DescribeMacro{\dTECP}
% \DescribeMacro{\dTECpE}
% \DescribeMacro{\dTECPE}
% El comando \cs{dTEC} tiene 2
% argumentos,\;\cs{dTEC}\marg{índice}\marg{objeto},\;
% e indica una transformación elemental genérica por la derecha del objeto.
% \begin{center}
%   |\dTEC{}{\Mat{A}} \dTEC{2}{\Mat{A}}|
%   \hspace{1.2cm}
%   \fbox{$\dTEC{}{\Mat{A}}$} \fbox{$\dTEC{2}{\Mat{A}}$}
% \end{center}
% 
% \begin{center}
%   |\dTECpE{}{\Mat{A}} \dTECpE{2}{\Mat{A}}|
%   \hspace{1.2cm}
%   \fbox{$\dTECpE{}{\Mat{A}}$} \fbox{$\dTECpE{2}{\Mat{A}}$}
% \end{center}
% 
% \begin{center}
%   |\dTECPE{}{\Mat{A}} \dTECPE{2}{\Mat{A}}|
%   \hspace{1.2cm}
%   \fbox{$\dTECPE{}{\Mat{A}}$} \fbox{$\dTECPE{2}{\Mat{A}}$}
% \end{center}
% 
% \DescribeMacro{\dETEC}
% \DescribeMacro{\dETECp}
% \DescribeMacro{\dETECP}
% \DescribeMacro{\dETECpE}
% \DescribeMacro{\dETECPE}
% El comando \cs{dETEC} tiene 2
% argumentos,\;\cs{dETEC}\marg{índice}\marg{objeto},\;
% e indica una transformación elemental espejo genérica por la derecha del objeto.
% \begin{center}
%   |\dETEC{}{\Mat{A}} \dETEC{2}{\Mat{A}}|
%   \hspace{1.2cm}
%   \fbox{$\dETEC{}{\Mat{A}}$} \fbox{$\dETEC{2}{\Mat{A}}$}
% \end{center}
% 
% \begin{center}
%   |\dETECp{}{\Mat{A}} \dETECp{2}{\Mat{A}}|
%   \hspace{1.2cm}
%   \fbox{$\dETECp{}{\Mat{A}}$} \fbox{$\dETECp{2}{\Mat{A}}$}
% \end{center}
% 
% \begin{center}
%   |\dETECP{}{\Mat{A}} \dETECP{2}{\Mat{A}}|
%   \hspace{1.2cm}
%   \fbox{$\dETECP{}{\Mat{A}}$} \fbox{$\dETECP{2}{\Mat{A}}$}
% \end{center}
% 
% \begin{center}
%   |\dETECpE{}{\Mat{A}} \dETECpE{2}{\Mat{A}}|
%   \hspace{1.2cm}
%   \fbox{$\dETECpE{}{\Mat{A}}$} \fbox{$\dETECpE{2}{\Mat{A}}$}
% \end{center}
% 
% \begin{center}
%   |\dETECPE{}{\Mat{A}} \dETECPE{2}{\Mat{A}}|
%   \hspace{1.2cm}
%   \fbox{$\dETECPE{}{\Mat{A}}$} \fbox{$\dETECPE{2}{\Mat{A}}$}
% \end{center}
% 
% \DescribeMacro{\dInvTEC}
% \DescribeMacro{\dInvTECp}
% \DescribeMacro{\dInvTECP}
% \DescribeMacro{\dInvTECpE}
% \DescribeMacro{\dInvTECPE}
% El comando \cs{dInvTEC} tiene 2
% argumentos,\;\cs{dInvTEC}\marg{índice}\marg{objeto},\;
% e indica una transformación elemental espejo inversa genérica por la derecha del objeto.
% \begin{center}
%   |\dInvTEC{}{\Mat{A}} \dInvTEC{2}{\Mat{A}}|
%   \hspace{1.2cm}
%   \fbox{$\dInvTEC{}{\Mat{A}}$} \fbox{$\dInvTEC{2}{\Mat{A}}$}
% \end{center}
% 
% \begin{center}
%   |\dInvTECp{}{\Mat{A}} \dInvTECp{2}{\Mat{A}}|
%   \hspace{1.2cm}
%   \fbox{$\dInvTECp{}{\Mat{A}}$} \fbox{$\dInvTECp{2}{\Mat{A}}$}
% \end{center}
% 
% \begin{center}
%   |\dInvTECP{}{\Mat{A}} \dInvTECP{2}{\Mat{A}}|
%   \hspace{1.2cm}
%   \fbox{$\dInvTECP{}{\Mat{A}}$} \fbox{$\dInvTECP{2}{\Mat{A}}$}
% \end{center}
% 
% \begin{center}
%   |\dInvTECpE{}{\Mat{A}} \dInvTECpE{2}{\Mat{A}}|
%   \hspace{1.2cm}
%   \fbox{$\dInvTECpE{}{\Mat{A}}$} \fbox{$\dInvTECpE{2}{\Mat{A}}$}
% \end{center}
% 
% \begin{center}
%   |\dInvTECPE{}{\Mat{A}} \dInvTECPE{2}{\Mat{A}}|
%   \hspace{1.2cm}
%   \fbox{$\dInvTECPE{}{\Mat{A}}$} \fbox{$\dInvTECPE{2}{\Mat{A}}$}
% \end{center}
% 
% \DescribeMacro{\dEInvTEC}
% \DescribeMacro{\dEInvTECp}
% \DescribeMacro{\dEInvTECP}
% \DescribeMacro{\dEInvTECpE}
% \DescribeMacro{\dEInvTECPE}
% El comando \cs{dEInvTEC} tiene 2
% argumentos,\;\cs{dEInvTEC}\marg{índice}\marg{objeto},\;
% e indica una transformación elemental espejo inversa genérica por la derecha del objeto.
% \begin{center}
%   |\dEInvTEC{}{\Mat{A}} \dEInvTEC{2}{\Mat{A}}|
%   \hspace{1.2cm}
%   \fbox{$\dEInvTEC{}{\Mat{A}}$} \fbox{$\dEInvTEC{2}{\Mat{A}}$}
% \end{center}
% 
% \begin{center}
%   |\dEInvTECp{}{\Mat{A}} \dEInvTECp{2}{\Mat{A}}|
%   \hspace{1.2cm}
%   \fbox{$\dEInvTECp{}{\Mat{A}}$} \fbox{$\dEInvTECp{2}{\Mat{A}}$}
% \end{center}
% 
% \begin{center}
%   |\dEInvTECP{}{\Mat{A}} \dEInvTECP{2}{\Mat{A}}|
%   \hspace{1.2cm}
%   \fbox{$\dEInvTECP{}{\Mat{A}}$} \fbox{$\dEInvTECP{2}{\Mat{A}}$}
% \end{center}
% 
% \begin{center}
%   |\dEInvTECpE{}{\Mat{A}} \dEInvTECpE{2}{\Mat{A}}|
%   \hspace{1.2cm}
%   \fbox{$\dEInvTECpE{}{\Mat{A}}$} \fbox{$\dEInvTECpE{2}{\Mat{A}}$}
% \end{center}
% 
% \begin{center}
%   |\dEInvTECPE{}{\Mat{A}} \dEInvTECPE{2}{\Mat{A}}|
%   \hspace{1.2cm}
%   \fbox{$\dEInvTECPE{}{\Mat{A}}$} \fbox{$\dEInvTECPE{2}{\Mat{A}}$}
% \end{center}
% 
% %%%%%%%%%%%%%%%%%%%%%%%%%%%%%%%%%%%%%%
% \paragraph{Transformaciones elementales particulares} Aquí describimos la notación de transformaciones específicas.
% 
% \DescribeMacro{\dTrF}
% \DescribeMacro{\dTrFp}
% \DescribeMacro{\dTrFP}
% \DescribeMacro{\dTrFpE}
% \DescribeMacro{\dTrFPE}
% El comando \cs{dTrF} tiene 2
% argumentos,\;\cs{dTrF}\marg{operación(es)}\marg{objeto},\; e indica una
% transformación (o transformaciones) elemental(es) por la izquierda del objeto.
% \begin{center}
%   |\dTrF{ \dOEgE{1}{'}\cdots\dOEgE{p}{'} }{\Mat{I}}|
%   \hspace{1.2cm}
%   \fbox{$\dTrF{\dOEgE{1}{'}\cdots\dOEgE{p}{'}}{\Mat{I}}$}
% \end{center}
% \begin{center}
%   |\dTrF{ \OpE{\su{5}{i}{j}}\OpE{\pr{-7}{j}} }{\Mat{A}}|
%   \hspace{1.2cm}
%   \fbox{$\dTrF{\OpE{\su{5}{i}{j}}\OpE{\pr{-7}{j}}}{\Mat{A}}$}
% \end{center}
% 
% \begin{center}
%   |\dTrFp{ \dOEgE{1}{'}\cdots\dOEgE{p}{'} }{\Mat{I}}|
%   \hspace{1.2cm}
%   \fbox{$\dTrFp{\dOEgE{1}{'}\cdots\dOEgE{p}{'}}{\Mat{I}}$}
% \end{center}
% \begin{center}
%   |\dTrFp{ \OpE{\su{5}{i}{j}}\OpE{\pr{-7}{j}} }{\Mat{A}}|
%   \hspace{1.2cm}
%   \fbox{$\dTrFp{\OpE{\su{5}{i}{j}}\OpE{\pr{-7}{j}}}{\Mat{A}}$}
% \end{center}
% 
% \begin{center}
%   |\dTrFP{ \dOEgE{1}{'}\cdots\dOEgE{p}{'} }{\Mat{I}}|
%   \hspace{1.2cm}
%   \fbox{$\dTrFP{\dOEgE{1}{'}\cdots\dOEgE{p}{'}}{\Mat{I}}$}
% \end{center}
% \begin{center}
%   |\dTrFP{ \OpE{\su{5}{i}{j}}\OpE{\pr{-7}{j}} }{\Mat{A}}|
%   \hspace{1.2cm}
%   \fbox{$\dTrFP{\OpE{\su{5}{i}{j}}\OpE{\pr{-7}{j}}}{\Mat{A}}$}
% \end{center}
% 
% \begin{center}
%   |\dTrFpE{ \dOEgE{1}{'}\cdots\dOEgE{p}{'} }{\Mat{I}}|
%   \hspace{1.2cm}
%   \fbox{$\dTrFpE{\dOEgE{1}{'}\cdots\dOEgE{p}{'}}{\Mat{I}}$}
% \end{center}
% \begin{center}
%   |\dTrFpE{ \OpE{\su{5}{i}{j}}\OpE{\pr{-7}{j}} }{\Mat{A}}|
%   \hspace{1.2cm}
%   \fbox{$\dTrFpE{\OpE{\su{5}{i}{j}}\OpE{\pr{-7}{j}}}{\Mat{A}}$}
% \end{center}
% 
% \begin{center}
%   |\dTrFPE{ \dOEgE{1}{'}\cdots\dOEgE{p}{'} }{\Mat{I}}|
%   \hspace{1.2cm}
%   \fbox{$\dTrFPE{\dOEgE{1}{'}\cdots\dOEgE{p}{'}}{\Mat{I}}$}
% \end{center}
% \begin{center}
%   |\dTrFPE{ \OpE{\su{5}{i}{j}}\OpE{\pr{-7}{j}} }{\Mat{A}}|
%   \hspace{1.2cm}
%   \fbox{$\dTrFPE{\OpE{\su{5}{i}{j}}\OpE{\pr{-7}{j}}}{\Mat{A}}$}
% \end{center}
% 
% \DescribeMacro{\dTrC}
% \DescribeMacro{\dTrCp}
% \DescribeMacro{\dTrCP}
% \DescribeMacro{\dTrCpE}
% \DescribeMacro{\dTrCPE}
% El comando \cs{dTrC} tiene 2
% argumentos,\;\cs{dTrC}\marg{operación(es)}\marg{objeto},\; e indica una
% transformación (o transformaciones) elemental(es) por la derecha del objeto.
% \begin{center}
%   |\dTrC{ \dOEgE{1}{'}\cdots\dOEgE{p}{'} }{\Mat{I}}|
%   \hspace{1.2cm}
%   \fbox{$\dTrC{\dOEgE{1}{'}\cdots\dOEgE{p}{'}}{\Mat{I}}$}
% \end{center}
% \begin{center}
%   |\dTrC{ \OpE{\su{5}{i}{j}}\OpE{\pr{-7}{j}} }{\Mat{A}}|
%   \hspace{1.2cm}
%   \fbox{$\dTrC{\OpE{\su{5}{i}{j}}\OpE{\pr{-7}{j}}}{\Mat{A}}$}
% \end{center}
% 
% \begin{center}
%   |\dTrCp{ \dOEgE{1}{'}\cdots\dOEgE{p}{'} }{\Mat{I}}|
%   \hspace{1.2cm}
%   \fbox{$\dTrCp{\dOEgE{1}{'}\cdots\dOEgE{p}{'}}{\Mat{I}}$}
% \end{center}
% \begin{center}
%   |\dTrCp{ \OpE{\su{5}{i}{j}}\OpE{\pr{-7}{j}} }{\Mat{A}}|
%   \hspace{1.2cm}
%   \fbox{$\dTrCp{\OpE{\su{5}{i}{j}}\OpE{\pr{-7}{j}}}{\Mat{A}}$}
% \end{center}
% 
% \begin{center}
%   |\dTrCP{ \dOEgE{1}{'}\cdots\dOEgE{p}{'} }{\Mat{I}}|
%   \hspace{1.2cm}
%   \fbox{$\dTrCP{\dOEgE{1}{'}\cdots\dOEgE{p}{'}}{\Mat{I}}$}
% \end{center}
% \begin{center}
%   |\dTrCP{ \OpE{\su{5}{i}{j}}\OpE{\pr{-7}{j}} }{\Mat{A}}|
%   \hspace{1.2cm}
%   \fbox{$\dTrCP{\OpE{\su{5}{i}{j}}\OpE{\pr{-7}{j}}}{\Mat{A}}$}
% \end{center}
% 
% \begin{center}
%   |\dTrCpE{ \dOEgE{1}{'}\cdots\dOEgE{p}{'} }{\Mat{I}}|
%   \hspace{1.2cm}
%   \fbox{$\dTrCpE{\dOEgE{1}{'}\cdots\dOEgE{p}{'}}{\Mat{I}}$}
% \end{center}
% \begin{center}
%   |\dTrCpE{ \OpE{\su{5}{i}{j}}\OpE{\pr{-7}{j}} }{\Mat{A}}|
%   \hspace{1.2cm}
%   \fbox{$\dTrCpE{\OpE{\su{5}{i}{j}}\OpE{\pr{-7}{j}}}{\Mat{A}}$}
% \end{center}
% 
% \begin{center}
%   |\dTrCPE{ \dOEgE{1}{'}\cdots\dOEgE{p}{'} }{\Mat{I}}|
%   \hspace{1.2cm}
%   \fbox{$\dTrCPE{\dOEgE{1}{'}\cdots\dOEgE{p}{'}}{\Mat{I}}$}
% \end{center}
% \begin{center}
%   |\dTrCPE{ \OpE{\su{5}{i}{j}}\OpE{\pr{-7}{j}} }{\Mat{A}}|
%   \hspace{1.2cm}
%   \fbox{$\dTrCPE{\OpE{\su{5}{i}{j}}\OpE{\pr{-7}{j}}}{\Mat{A}}$}
% \end{center}
% 
% \DescribeMacro{\dTrFC}
% \DescribeMacro{\dTrFCp}
% \DescribeMacro{\dTrFCP}
% \DescribeMacro{\dTrFCpE}
% \DescribeMacro{\dTrFCPE}
% El comando \cs{dTrFC} tiene 3
% argumentos,\;\cs{dTrFC}\marg{operacionesIzda}\marg{operacionesDcha}\marg{objeto},\;
% e indica una transformación (o transformaciones) elemental(es) por
% cada lado del objeto.
% \begin{center}
%   |\dTrFC{\OpE{\su{-5}{i}{j}}}{\OpE{\pr{-7}{j}}}{\Mat{A}}|
%   \hspace{1.2cm}
%   \fbox{$\dTrFC{\OpE{\su{-5}{i}{j}}}{\OpE{\pr{-7}{j}}}{\Mat{A}}$}
% \end{center}
% 
% \begin{center}
%   |\dTrFCp{\OpE{\su{-5}{i}{j}}}{\OpE{\pr{-7}{j}}}{\Mat{A}}|
%   \hspace{1.2cm}
%   \fbox{$\dTrFCp{\OpE{\su{-5}{i}{j}}}{\OpE{\pr{-7}{j}}}{\Mat{A}}$}
% \end{center}
% 
% \begin{center}
%   |\dTrFCP{\OpE{\su{-5}{i}{j}}}{\OpE{\pr{-7}{j}}}{\Mat{A}}|
%   \hspace{1.2cm}
%   \fbox{$\dTrFCP{\OpE{\su{-5}{i}{j}}}{\OpE{\pr{-7}{j}}}{\Mat{A}}$}
% \end{center}
% 
% \begin{center}
%   |\dTrFCpE{\OpE{\su{-5}{i}{j}}}{\OpE{\pr{-7}{j}}}{\Mat{A}}|
%   \hspace{1.2cm}
%   \fbox{$\dTrFCpE{\OpE{\su{-5}{i}{j}}}{\OpE{\pr{-7}{j}}}{\Mat{A}}$}
% \end{center}
% 
% \begin{center}
%   |\dTrFCPE{\OpE{\su{-5}{i}{j}}}{\OpE{\pr{-7}{j}}}{\Mat{A}}|
%   \hspace{1.2cm}
%   \fbox{$\dTrFCPE{\OpE{\su{-5}{i}{j}}}{\OpE{\pr{-7}{j}}}{\Mat{A}}$}
% \end{center}
% 
% \subsubsection{Operador que quita un elemento}
% 
% \DescribeMacro{\fueraitemL}
% El comando \cs{fueraitemL} tiene 1
% argumento,\;\cs{fueraitemL}\marg{indice},\; y denota la eliminación por
% la izquierda del elemento correspondiente al \marg{indice}
% \begin{center}
%   |\fueraitemL{i}| 
%   \hspace{1.2cm}
%   \fbox{$\fueraitemL{i}$}
% \end{center}
% 
% \DescribeMacro{\fueraitemR}
% El comando \cs{fueraitemR} tiene 1
% argumento,\;\cs{fueraitemR}\marg{indice},\; y denota la eliminación por
% la derecha del elemento correspondiente al \marg{indice}
% \begin{center}
%   |\fueraitemR{j}| 
%   \hspace{1.2cm}
%   \fbox{$\fueraitemR{j}$}
% \end{center}
% 
% \DescribeMacro{\quitaLR}
% El comando \cs{quitaLR} tiene 3
% argumentos,\;\cs{quitaLR}\marg{objeto}\marg{indIzda}\marg{indDcha},\;
% y denota el resultante de quitar un elemento por la izquierda y otro
% por la derecha
% \begin{center}
%   |\quitaLR{\Mat{A}}{i}{j}| 
%   \hspace{1.2cm}
%   \fbox{$\quitaLR{\Mat{A}}{i}{j}$}
% \end{center}
% 
% \DescribeMacro{\quitaL}
% El comando \cs{quitaL} tiene 2
% argumentos,\;\cs{quitaL}\marg{objeto}\marg{indIzda},\;
% y denota el resultante de quitar un elemento por la izquierda
% \begin{center}
%   |\quitaL{\Mat{A}}{i}| 
%   \hspace{1.2cm}
%   \fbox{$\quitaL{\Mat{A}}{i}$}
% \end{center}
% 
% \DescribeMacro{\quitaR}
% El comando \cs{quitaR} tiene 2
% argumentos,\;\cs{quitaR}\marg{objeto}\marg{indDcha},\;
% y denota el resultante de quitar un elemento por la derecha
% \begin{center}
%   |\quitaR{\Mat{A}}{j}| 
%   \hspace{1.2cm}
%   \fbox{$\quitaR{\Mat{A}}{j}$}
% \end{center}
% 
% \subsubsection{Selección de elementos sin emplear el operador selector}
% 
% \DescribeMacro{\elemUUU}
% El comando \cs{elemUUU} tiene 2
% argumentos,\;\cs{elemUUU}\marg{sistema}\marg{indice},\; y denota la
% selección del elemento correspondiente al \marg{indice}
% \begin{center}
%   |\elemUUU{\SV{Z}}{i}|
%   \hspace{1.2cm}
%   \fbox{$\elemUUU{\SV{Z}}{i}$}
% \end{center}
% 
% \DescribeMacro{\VectFFF}
% \DescribeMacro{\VectFFFT}
% El comando \cs{VectFFF} tiene 2
% argumentos,\;\cs{VectFFF}\marg{nombre}\marg{indice},\; y denota la
% selección de la fila correspondiente al \marg{indice}
% \begin{center}
%   |\VectFFF{A}{i} \VectFFFT{A}{i}|
%   \hspace{1.2cm}
%   \fbox{$\VectFFF {A}{i}$}
%   \fbox{$\VectFFFT{A}{i}$}
% \end{center}
% 
% \DescribeMacro{\VectCCC}
% \DescribeMacro{\VectCCCT}
% El comando \cs{VectCCC} tiene 2
% argumentos,\;\cs{VectCCC}\marg{nombre}\marg{indice},\; y denota la
% selección de la columna correspondiente al \marg{indice}
% \begin{center}
%   |\VectCCC{A}{i} \VectCCCT{A}{i}|
%   \hspace{1.2cm}
%   \fbox{$\VectCCC {A}{i}$}
%   \fbox{$\VectCCCT{A}{i}$}
% \end{center}
% 
% \DescribeMacro{\eleMMM}
% \DescribeMacro{\eleMMMT}
% \DescribeMacro{\eleMM}
%  tiene 3
% argumentos,\;\marg{nombre}\marg{indiceFil}\marg{indiceCol},\; y denota la
% selección del elemento correspondiente a los índices indicados
% \begin{center}
%   |\eleMMM{A}{i}{j} \eleMMMT{A}{i}{j} \eleMM{A}{i}{j}|
%   \hspace{1.2cm}
%   \fbox{$\eleMMM {A}{i}{j}$}
%   \fbox{$\eleMMMT{A}{i}{j}$}
%   \fbox{$\eleMM  {A}{i}{j}$}
% \end{center}
% 
% %%%%%%%%%%%%%%%%%%%%%%%%%%%%%%%%%%%%%%
% \subsection{Sistemas genéricos}
% 
% \DescribeMacro{\SV}
% El comando \cs{SV} tiene 1 argumento,\;\cs{SV}\marg{nombre}
% \begin{center}
%   |\SV{A}| 
%   \hspace{1.2cm}
%   \fbox{$\SV{A}$}
% \end{center}
% 
% \DescribeMacro{\concatSV}
% El comando \cs{concatSV} tiene 2
% argumentos,\;\cs{concatSV}\marg{sistemaA}\marg{sistemaB},\; y denota
% la concatenación del \marg{sistemaA} con el \marg{sistemaB}.
% \begin{center}
%   |\concatSV{\Mat{A}}{\Mat{B}}| 
%   \hspace{1.2cm}
%   \fbox{$\concatSV{\Mat{A}}{\Mat{B}}$}
% \end{center}
% 
% %%%%%%%%%%%%%%%%%%%%%%%%%%%%%%%%%%%%%%
% 
% \subsection{Vectores y matrices}
% 
% \subsubsection{Vectores genéricos}
% 
% \DescribeMacro{\vect}
% \DescribeMacro{\vectp}
% \DescribeMacro{\vectP}
% tiene 1 argumento,\;\cs{vect}\marg{nombre},\; y denota un vector
% genérico.
% \begin{center}
%   |\vect{a} \vectp{a} \vectP{a}| 
%   \hspace{1.2cm}
%   \fbox{$\vect  {a}$}
%   \fbox{$\vectp*{a}$}
%   \fbox{$\vectP*{a}$}
% \end{center}
% 
% %%%%%%%%%%%%%%%%%%%%%%%%%%%%%%%%%%%%%%
% \subsubsection{Vectores de $\R[n]$}
% 
% \DescribeMacro{\Vect}
% \DescribeMacro{\Vectp}
% \DescribeMacro{\VectP}
% tiene 1 argumento,\;\cs{Vect}\marg{nombre},\; y denota un vector de
% $\R[n]$
% \begin{center}
%   |\Vect{a} \Vectp{a} \VectP{a}| 
%   \hspace{1.2cm}
%   \fbox{$\Vect  {a}$}
%   \fbox{$\Vectp*{a}$}
%   \fbox{$\VectP*{a}$}
% \end{center}
% 
% \subsubsection{Matrices}
% 
% \DescribeMacro{\Mat}
% \DescribeMacro{\Matp}
% \DescribeMacro{\MatP}
% tiene 1 argumento,\;\cs{Mat}\marg{nombre},\; y denota una matriz
% \begin{center}
%   |\Mat{A} \Matp{A} \MatP{A}| 
%   \hspace{1.2cm}
%   \fbox{$\Mat  {A}$}
%   \fbox{$\Matp*{A}$}
%   \fbox{$\MatP*{A}$}
% \end{center}
% 
% %%%%%%%%%%%%%%%%%%%%%%%%%%%%%%%%%%%%%%
% \paragraph{Matrices transpuestas}{\mbox{ }}
% 
% \DescribeMacro{\MatT}
% \DescribeMacro{\MatTp}
% \DescribeMacro{\MatTP}
% \DescribeMacro{\MatTpE}
% \DescribeMacro{\MatTPE}
% El comando \cs{MatT} tiene 1 argumento,\;\cs{MatT}\marg{nombre}
% \begin{center}
%   |\MatT{A}| 
%   \hspace{1.2cm}
%   \fbox{$\MatT{A}$}
% \end{center}
% \begin{center}
%   |\MatTp{A} \MatTP{A}| 
%   \hspace{1.2cm}
%   \fbox{$\MatTp*{A}$}
%   \fbox{$\MatTP*{A}$}
% \end{center}
% \begin{center}
%   |\MatTpE{A} \MatTPE{A}| 
%   \hspace{1.2cm}
%   \fbox{$\MatTpE*{A}$}
%   \fbox{$\MatTPE*{A}$}
% \end{center}
% 
% %%%%%%%%%%%%%%%%%%%%%%%%%%%%%%%%%%%%%%
% \subparagraph{Matriz transpuesta de la transpuesta}{\mbox{ }}
% 
% \DescribeMacro{\MatTT}
% \DescribeMacro{\MatTTPE}
% El comando \cs{MatTT} tiene 1 argumento,\;\cs{MatTT}\marg{nombre}
% \begin{center}
%   |\MatTT{A} \MatTTPE{A}| 
%   \hspace{1.2cm}
%   \fbox{$\MatTT*  {A}$}
%   \fbox{$\MatTTPE*{A}$}
% \end{center}
% 
% %%%%%%%%%%%%%%%%%%%%%%%%%%%%%%%%%%%%%%
% \paragraph{Matrices columna}{\mbox{ }}
% 
% \DescribeMacro{\MVectF}
% El comando \cs{MVectF} tiene 2
% argumentos,\;\cs{MVectF}\marg{nombre}\marg{índice},\; y denota una
% matriz columna creada a partir de una \emph{fila} de una matriz
% 
% El comando \cs{MVectF} tiene 2 argumentos,\;\cs{MVectF}\marg{nombre}\marg{índice}
% \begin{center}
%   |\MVectF{A}{i}| 
%   \hspace{1.2cm}
%   \fbox{$\MVectF{A}{i}$}
% \end{center}
% 
% \DescribeMacro{\MVectC}
% El comando \cs{MVectC} tiene 2
% argumentos,\;\cs{MVectC}\marg{nombre}\marg{índice},\; y denota una
% matriz columna creada a partir de una \emph{columna} de una matriz
% \begin{center}
%   |\MVectC{A}{j}| 
%   \hspace{1.2cm}
%   \fbox{$\MVectC{A}{j}$}
% \end{center}
% 
% %%%%%%%%%%%%%%%%%%%%%%%%%%%%%%%%%%%%%%
% \paragraph{Matrices fila}{\mbox{ }}
% 
% \DescribeMacro{\MVectFT}
% El comando \cs{MVectFT} tiene 2 argumentos,\;\cs{MVectFT}\marg{nombre}\marg{índice},\;
% y denota una matriz fila creada a partir de una \emph{fila} de una matriz
% \begin{center}
%   |\MVectFT{A}{i}| 
%   \hspace{1.2cm}
%   \fbox{$\MVectFT{A}{i}$}
% \end{center}
% 
% \DescribeMacro{\MVectCT}
% El comando \cs{MVectCT} tiene 2 argumentos,\;\cs{MVectCT}\marg{nombre}\marg{índice},\;
% y denota una matriz fila creada a partir de una \emph{columna} de una matriz
% \begin{center}
%   |\MVectCT{A}{j}| 
%   \hspace{1.2cm}
%   \fbox{$\MVectCT{A}{j}$}
% \end{center}
% 
% %%%%%%%%%%%%%%%%%%%%%%%%%%%%%%%%%%%%%%
% \paragraph{Matriz inversa} Notación para las matrices inversas
% 
% \DescribeMacro{\InvMat}
% \DescribeMacro{\InvMatp}
% \DescribeMacro{\InvMatP}
% \DescribeMacro{\InvMatpE}
% \DescribeMacro{\InvMatPE}
% El comando \cs{InvMat} tiene 1 argumento,\;\cs{InvMat}\marg{nombre},\;
% y denota la inversa de una matriz
% \begin{center}
%   |\InvMat{A}| 
%   \hspace{1.2cm}
%   \fbox{$\InvMat{A}$}
% \end{center}
% \begin{center}
%   |\InvMatp{A} \InvMatP{A}| 
%   \hspace{1.2cm}
%   \fbox{$\InvMatp*{A}$}
%   \fbox{$\InvMatP*{A}$}
% \end{center}
% \begin{center}
%   |\InvMatpE{A} \InvMatPE{A}| 
%   \hspace{1.2cm}
%   \fbox{$\InvMatpE*{A}$}
%   \fbox{$\InvMatPE*{A}$}
% \end{center}
% 
% \DescribeMacro{\InvMatT}
% \DescribeMacro{\InvMatTpE}
% \DescribeMacro{\InvMatTPE}
% El comando \cs{InvMatT} tiene 1 argumento,\;\cs{InvMatT}\marg{nombre},\;
% y denota la inversa de una matriz transpuesta
% \begin{center}
%   |\InvMatT{A} \InvMatTpE{A} \InvMatTPE{A}| 
%   \hspace{0.2cm}
%   \fbox{$\InvMatT{A}$}
%   \fbox{$\InvMatTpE*{A}$}
%   \fbox{$\InvMatTPE*{A}$}
% \end{center}
% 
% \DescribeMacro{\TInvMat}
% \DescribeMacro{\TInvMatpE}
% \DescribeMacro{\TInvMatPE}
% El comando \cs{TInvMat} tiene 1 argumento,\;\cs{TInvMat}\marg{nombre},\;
% y denota la transpuesta de la inversa de una matriz
% \begin{center}
%   |\TInvMat{A} \TInvMatpE{A} \TInvMatPE{A}| 
%   \hspace{0.2cm}
%   \fbox{$\TInvMat{A}$}
%   \fbox{$\TInvMatpE*{A}$}
%   \fbox{$\TInvMatPE*{A}$}
% \end{center}
% 
% %%%%%%%%%%%%%%%%%%%%%%%%%%%%%%%%%%%%%%
% \subsubsection{Miscelánea matrices}
% 
% \DescribeMacro{\Traza}
% El comando \cs{Traza} no  tiene 
% argumentos
% \begin{center}
%   |\Traza| 
%   \hspace{1.2cm}
%   \fbox{$\Traza$}
% \end{center}
% 
% \DescribeMacro{\rg}
% El comando \cs{rg} no  tiene 
% argumentos
% \begin{center}
%   |\rg| 
%   \hspace{1.2cm}
%   \fbox{$\rg$}
% \end{center}
% 
% \DescribeMacro{\traza}
% El comando \cs{traza} tiene 1 
% argumento,\;\cs{traza}\marg{objeto}
% \begin{center}
%   |\traza{\Mat{A}}| 
%   \hspace{1.2cm}
%   \fbox{$\traza{\Mat{A}}$}
% \end{center}
% 
% \DescribeMacro{\rango}
% El comando \cs{rango} tiene 1 
% argumento,\;\cs{rango}\marg{objeto}
% \begin{center}
%   |\rango{\Mat{A}}| 
%   \hspace{1.2cm}
%   \fbox{$\rango{\Mat{A}}$}
% \end{center}
% 
% %%%%%%%%%%%%%%%%%%%%%%%%%%%%%%%%%%%%%% 
% \paragraph{Determinante de una matriz}
% 
% \DescribeMacro{\cof}
% El comando \cs{cof} no  tiene 
% argumentos
% \begin{center}
%   |\cof| 
%   \hspace{1.2cm}
%   \fbox{$\cof$}
% \end{center}
% 
% \DescribeMacro{\adj}
% El comando \cs{adj} no  tiene 
% argumentos
% \begin{center}
%   |\adj| 
%   \hspace{1.2cm}
%   \fbox{$\adj$}
% \end{center}
% 
% \DescribeMacro{\determinante}
% El comando \cs{determinante} tiene 1
% argumento,\;\cs{determinante}\marg{objeto},\; y denota el determinante
% del \marg{objeto} usando las barras verticales
% \begin{center}
%   |\determinante{\Mat{A}}| 
%   \hspace{1.2cm}
%   \fbox{$\determinante{\Mat{A}}$}
% \end{center}
% 
% \DescribeMacro{\subMat}
% El comando \cs{subMat} tiene 3
% argumentos,\;\cs{subMat}\marg{nombre}\marg{indIzda}\marg{indDcha},\; y denota la submatriz resultante de quitar una o más filas y columnas de la matriz \marg{nombre}
% \begin{center}
%   |\subMat{A}{i}{j}| 
%   \hspace{1.2cm}
%   \fbox{$\subMat{A}{i}{j}$}
% \end{center}
% 
% \DescribeMacro{\Menor}
% \DescribeMacro{\MenoR}
% El comando \cs{Menor} tiene 3
% argumentos,\;\cs{Menor}\marg{nombre}\marg{indFila}\marg{indCol},\; y denota el menor de la matriz correspondiente a la fila y columna indicadas
% \begin{center}
%   |\Menor{A}{i}{j} \MenoR{A}{i}{j}| 
%   \hspace{1.2cm}
%   \fbox{$\Menor{A}{i}{j}$}
%   \fbox{$\MenoR{A}{i}{j}$}
% \end{center}
% 
% \DescribeMacro{\Cof}
% El comando \cs{Cof} tiene 3
% argumentos,\;\cs{Cof}\marg{nombre}\marg{indFila}\marg{indCol},\; y denota el cofactor de la fila y columna indicadas
% \begin{center}
%   |\Cof{A}{i}{j}| 
%   \hspace{1.2cm}
%   \fbox{$\Cof{A}{i}{j}$}
% \end{center}
% 
% %%%%%%%%%%%%%%%%%%%%%%%%%%%%%%%%%%%%%% 
% \paragraph{Orden de las matrices}
% 
% \DescribeMacro{\Dim}
% \DescribeMacro{\Dimp}
% \DescribeMacro{\DimP}
% \DescribeMacro{\DimpE}
% \DescribeMacro{\DimPE}
% El comando \cs{Dim} tiene 3
% argumentos,\;\cs{Dim}\marg{objeto}\marg{filas}\marg{columnas}
% \begin{center}
%   |\Dim{xxx}{n}{m}| 
%   \hspace{1.2cm}
%   \fbox{$\Dim{xxx}{n}{m}$}
% \end{center}
% \begin{center}
%   |\Dimp{x}{n}{m} \DimP{x}{n}{m}| 
%   \hspace{1.2cm}
%   \fbox{$\Dimp*{x}{n}{m}$}
%   \fbox{$\DimP*{x}{n}{m}$}
% \end{center}
% \begin{center}
%   |\DimpE{x}{n}{m} \DimPE{x}{n}{m}| 
%   \hspace{1.2cm}
%   \fbox{$\DimpE*{x}{n}{m}$}
%   \fbox{$\DimPE*{x}{n}{m}$}
% \end{center}
% 
% %%%%%%%%%%%%%%%%%%%%%%%%%%%%%%%%%%%%%%
% 
% \DescribeMacro{\Matdim}
% \DescribeMacro{\Matdimp}
% \DescribeMacro{\MatdimP}
% \DescribeMacro{\MatdimpE}
% \DescribeMacro{\MatdimPE}
% El comando \cs{Matdim} tiene 3
% argumentos,\;\cs{Matdim}\marg{nombre}\marg{filas}\marg{columnas}
% \begin{center}
%   |\Matdim{A}{n}{m}| 
%   \hspace{1.2cm}
%   \fbox{$\Matdim{A}{n}{m}$}
% \end{center}
% \begin{center}
%   |\Matdimp{A}{n}{m} \MatdimP{A}{n}{m}| 
%   \hspace{1.2cm}
%   \fbox{$\Matdimp*{A}{n}{m}$}
%   \fbox{$\MatdimP*{A}{n}{m}$}
% \end{center}
% \begin{center}
%   |\MatdimpE{A}{n}{m} \MatdimPE{A}{n}{m}| 
%   \hspace{1.2cm}
%   \fbox{$\MatdimpE*{A}{n}{m}$}
%   \fbox{$\MatdimPE*{A}{n}{m}$}
% \end{center}
% 
% %%%%%%%%%%%%%%%%%%%%%%%%%%%%%%%%%%%%%%
% \paragraph{Matriz de autovalores}{\mbox{ } }
% 
% \DescribeMacro{\MDaV}
% \cs{MDaV} no  tiene
% argumentos e indica la letra usada par las matrices de autovalores
% \begin{center}
%   |\MDaV|
%   \hspace{1.2cm}
%   \fbox{$\MDaV$}
% \end{center}
% 
% %%%%%%%%%%%%%%%%%%%%%%%%%%%%%%%%%%%%%%
% \subsection{Productos entre vectores}
% 
% \subsubsection{Producto escalar}
% 
% \DescribeMacro{\eSc}
% tiene 2 argumentos,\;\cs{eSc}\marg{objeto}\marg{objeto},\; y denota el
% producto escalar entre dos objetos
% \begin{center}
%   |\eSc{f(x)}{g(x)}| 
%   \hspace{1.2cm}
%   \fbox{$\eSc{f(x)}{g(x)}$}
% \end{center}
% 
% \DescribeMacro{\esc}
% tiene 2 argumentos,\;\cs{esc}\marg{nombre}\marg{nombre},\; y denota el
% producto escalar entre dos vectores genéricos
% \begin{center}
%   |\esc{a}{b}| 
%   \hspace{1.2cm}
%   \fbox{$\esc{a}{b}$}
% \end{center}
% 
% \subsubsection{Producto punto}
% 
% \DescribeMacro{\dotProd}
% \DescribeMacro{\dotProdp}
% \DescribeMacro{\dotProdP}
% tiene 2 argumentos,\;\cs{dotProd}\marg{objeto}\marg{objeto},\; y
% denota el producto punto entre dos objetos
% \begin{center}
%   |\dotProd{(\Vect{a}+\Vect{b})}{\Vect{c}}| 
%   \hspace{1.2cm}
%   \fbox{$\dotProd{(\Vect{a}+\Vect{b})}{\Vect{c}}$}
% \end{center}
% 
% \noindent
% \emph{¡Ojo! en las versiones con paréntesis he me saltado en convenio y
%   en lugar de terminar en \texttt{pE} o \texttt{PE}, sencillamente
%   terminan en \texttt{p} o \texttt{P}}.
% 
% \begin{center}
%   |\dotProdp{(\Vect{a}+\Vect{b})}{\Vect{c}} \dotProdpP{(\Vect{a}+\Vect{b})}{\Vect{c}}| 
%   \hspace{1.2cm}
%   \fbox{$\dotProdp*{(\Vect{a}+\Vect{b})}{\Vect{c}}$}
%   \fbox{$\dotProdP*{(\Vect{a}+\Vect{b})}{\Vect{c}}$}
% \end{center}
% 
% \DescribeMacro{\dotprod}
% \DescribeMacro{\dotprodp}
% \DescribeMacro{\dotprodP}
% tiene 2 argumentos,\;\cs{dotprod}\marg{nombre}\marg{nombre},\; y
% denota el producto punto entre dos vectores de $\R[n]$
% \begin{center}
%   |\dotprod{a}{b} \dotprodp{a}{b} \dotprodP{a}{b}| 
%   \hspace{1.2cm}
%   \fbox{$\dotprod  {a}{b}$}
%   \fbox{$\dotprodp*{a}{b}$}
%   \fbox{$\dotprodP*{a}{b}$}
% \end{center}
% 
% \subsubsection{Producto punto a punto o \emph{Hadamard}}
% 
% \DescribeMacro{\prodH}
% \DescribeMacro{\prodHp}
% \DescribeMacro{\prodHP}
% tiene 2 argumentos,\;\cs{prodH}\marg{objeto}\marg{objeto},\; y
% denota el producto punto a punto entre dos objetos
% \begin{center}
%   |\prodH{(\Vect{a}+\Vect{b})}{\Vect{c}}| 
%   \hspace{1.2cm}
%   \fbox{$\prodH{(\Vect{a}+\Vect{b})}{\Vect{c}}$}
% \end{center}
% \begin{center}
%   |\prodHp{\widehat{\Vect{b}}}{\Vect{c}} \prodHP{\widehat{\Vect{b}}}{\Vect{c}}| 
%   \hspace{1.2cm}
%   \fbox{$\prodHp*{\widehat{\Vect{b}}}{\Vect{c}}$}
%   \fbox{$\prodHP*{\widehat{\Vect{b}}}{\Vect{c}}$}
% \end{center}
% 
% \DescribeMacro{\prodh}
% \DescribeMacro{\prodhp}
% \DescribeMacro{\prodhP}
% tiene 2 argumentos,\;\cs{prodh}\marg{nombre}\marg{nombre},\; y
% denota el producto punto a punto entre dos vectores de $\R[n]$
% \begin{center}
%   |\prodh{a}{b} \prodhp{a}{b} \prodhP{a}{b}| 
%   \hspace{1.2cm}
%   \fbox{$\prodh  {a}{b}$}
%   \fbox{$\prodhp*{a}{b}$}
%   \fbox{$\prodhP*{a}{b}$}
% \end{center}
% 
% \subsection{Matriz por vector y vector por matriz}
% 
% \DescribeMacro{\MV}
% \DescribeMacro{\MvpE}
% \DescribeMacro{\MVPE}
% tiene 2 argumentos,\;\cs{MV}\marg{nombre}\marg{nombre},\; y
% denota el producto de una matriz por un vector de $\R[n]$
% \begin{center}
%   |\MV{A}{b} \MVpE{A}{b} \MVPE{A}{b}| 
%   \hspace{1.2cm}
%   \fbox{$\MV   {A}{b}$}
%   \fbox{$\MVpE*{A}{b}$}
%   \fbox{$\MVpE*{A}{b}$}
% \end{center}
% 
% \DescribeMacro{\VM}
% \DescribeMacro{\VMpE}
% \DescribeMacro{\VMPE}
% tiene 2 argumentos,\;\cs{MV}\marg{nombre}\marg{nombre},\; y
% denota el producto de un vector de $\R[n]$ por una matriz
% \begin{center}
%   |\VM{a}{B} \VMpE{a}{B}|
%   \hspace{1.2cm}
%   \fbox{$\VM   {a}{B}$}
%   \fbox{$\VMpE*{a}{B}$}
%   \fbox{$\VMPE*{a}{B}$}
% \end{center}
% 
% \DescribeMacro{\MTV}
% \DescribeMacro{\MTVp}
% \DescribeMacro{\MTVP}
% tiene 2 argumentos,\;\cs{MTV}\marg{nombre}\marg{nombre},\; y
% denota el producto de una matriz transpuesta por un vector de $\R[n]$
% \begin{center}
%   |\MTV{A}{b} \MTVp{A}{b} \MTVP{A}{b}|
%   \hspace{1.2cm}
%   \fbox{$\MTV  {A}{b}$}
%   \fbox{$\MTVp*{A}{b}$}
%   \fbox{$\MTVP*{A}{b}$}
% \end{center}
% 
% 
% \DescribeMacro{\VMT}
% \DescribeMacro{\VMTp}
% \DescribeMacro{\VMTP}
% tiene 2 argumentos,\;\cs{MTV}\marg{nombre}\marg{nombre},\; y
% denota el producto de un vector de $\R[n]$ por una matriz transpuesta
% \begin{center}
%   |\VMT{a}{B} \VMTp{a}{B} \VMTP{a}{B}| 
%   \hspace{1.2cm}
%   \fbox{$\VMT  {a}{B}$}
%   \fbox{$\VMTp*{a}{B}$}
%   \fbox{$\VMTP*{a}{B}$}
% \end{center}
% 
% \subsection{Matriz por matriz}
% 
% \DescribeMacro{\MN}
% tiene 2 argumentos,\;\cs{MN}\marg{nombre}\marg{nombre},\; y
% denota el producto matriz por matriz
% \begin{center}
%   |\MN{A}{B}| 
%   \hspace{1.2cm}
%   \fbox{$\MN{A}{B}$}
% \end{center}
% 
% \DescribeMacro{\MTN}
% \DescribeMacro{\MTNp}
% \DescribeMacro{\MTNP}
% tiene 2 argumentos,\;\cs{MTN}\marg{nombre}\marg{nombre},\; y
% denota el producto matriz transpuesta por matriz
% \begin{center}
%   |\MTN{A}{B} \MTNp{A}{B} \MTNP{A}{B}| 
%   \hspace{1.2cm}
%   \fbox{$\MTN  {A}{B}$}
%   \fbox{$\MTNp*{A}{B}$}
%   \fbox{$\MTNP*{A}{B}$}
% \end{center}
% 
% \DescribeMacro{\MNT}
% \DescribeMacro{\MNTp}
% \DescribeMacro{\MNTP}
% tiene 2 argumentos,\;\cs{MNT}\marg{nombre}\marg{nombre},\; y
% denota el producto matriz por matriz transpuesta
% \begin{center}
%   |\MNT{A}{B} \MNTp{A}{B} \MNTP{A}{B}|
%   \hspace{1.2cm}
%   \fbox{$\MNT  {A}{B}$}
%   \fbox{$\MNTp*{A}{B}$}
%   \fbox{$\MNTP*{A}{B}$}
% \end{center}
% 
% \DescribeMacro{\MTM}
% \DescribeMacro{\MTMp}
% \DescribeMacro{\MTMP}
% tiene 2 argumentos,\;\cs{MTM}\marg{nombre}\marg{nombre},\; y
% denota el producto matriz transpuesta por matriz
% \begin{center}
%   |\MTM{A} \MTMp{A} \MTMP{A}|
%   \hspace{1.2cm}
%   \fbox{$\MTM  {A}$}
%   \fbox{$\MTMp*{A}$}
%   \fbox{$\MTMP*{A}$}
% \end{center}
% 
% \DescribeMacro{\MMT}
% \DescribeMacro{\MMTp}
% \DescribeMacro{\MMTP}
% tiene 2 argumentos,\;\cs{MMT}\marg{nombre}\marg{nombre},\; y
% denota el producto matriz por su transpuesta
% \begin{center}
%   |\MMT{A} \MMTp{A} \MMTP{A}|
%   \hspace{1.2cm}
%   \fbox{$\MMT  {A}$}
%   \fbox{$\MMTp*{A}$}
%   \fbox{$\MMTP*{A}$}
% \end{center}
% 
% \DescribeMacro{\MNMT}
% \DescribeMacro{\MNMTp}
% \DescribeMacro{\MNMTP}
% tiene 2 argumentos,\;\cs{MNMT}\marg{nombre}\marg{nombre},\; y
% denota el producto matriz por matriz por matriz transpuesta
% \begin{center}
%   |\MNMT{A}{D} \MNMTp{A}{D} \MNMTP{A}{D}|
%   \hspace{1.2cm}
%   \fbox{$\MNMT  {A}{B}$}
%   \fbox{$\MNMTp*{A}{B}$}
%   \fbox{$\MNMTP*{A}{B}$}
% \end{center}
% 
% \DescribeMacro{\MTNM}
% \DescribeMacro{\MTNMp}
% \DescribeMacro{\MTNMP}
% tiene 2 argumentos,\;\cs{MTNM}\marg{nombre}\marg{nombre},\; y
% denota el producto matriz transpuesta por matriz por matriz transpuesta
% \begin{center}
%   |\MTNM{A}{D} \MTNMp{A}{D} \MTNMP{A}{D}|
%   \hspace{1.2cm}
%   \fbox{$\MTNM  {A}{B}$}
%   \fbox{$\MTNMp*{A}{B}$}
%   \fbox{$\MTNMP*{A}{B}$}
% \end{center}
% 
% \subsection{Otros productos entre matrices y vectores}
% 
% \DescribeMacro{\MTMV}
% \DescribeMacro{\MTMVp}
% \DescribeMacro{\MTMVP}
% tiene 2 argumentos,\;\cs{MTMV}\marg{nombre}\marg{nombre},\; y
% denota el producto matriz transpuesta por matriz por vector
% \begin{center}
%   |\MTMV{A}{b} \MTMVp{A}{b} \MTMVP{A}{b}|
%   \hspace{1.2cm}
%   \fbox{$\MTMV  {A}{b}$}
%   \fbox{$\MTMVp*{A}{b}$}
%   \fbox{$\MTMVP*{A}{b}$}
% \end{center}
% 
% \DescribeMacro{\VMW}
% tiene 3 argumentos,\;\cs{VMW}\marg{nombre}\marg{nombre}\marg{nombre},\; y
% denota el producto vector por matriz por vector
% \begin{center}
%   |\VMW{a}{B}{c}| 
%   \hspace{1.2cm}
%   \fbox{$\VMW{a}{B}{c}$}
% \end{center}
% 
% \DescribeMacro{\VMV}
% tiene 2 argumentos,\;\cs{VMV}\marg{nombre}\marg{nombre},\; y
% denota el producto vector por matriz por vector
% \begin{center}
%   |\VMV{a}{B}| 
%   \hspace{1.2cm}
%   \fbox{$\VMV{a}{B}$}
% \end{center}
% 
% \DescribeMacro{\VMTW}
% \DescribeMacro{\VMTWp}
% \DescribeMacro{\VMTWP}
% tiene 3 argumentos,\;\cs{VMTW}\marg{nombre}\marg{nombre}\marg{nombre},\; y
% denota el producto vector por matriz transpuesta por vector
% \begin{center}
%   |\VMTW{a}{B}{c} \VMTWp{a}{B}{c} \VMTWP{a}{B}{c}|
%   \hspace{1.2cm}
%   \fbox{$\VMTW  {a}{B}{c}$}
%   \fbox{$\VMTWp*{a}{B}{c}$}
%   \fbox{$\VMTWP*{a}{B}{c}$}
% \end{center}
% 
% \DescribeMacro{\VMTV}
% \DescribeMacro{\VMTVp}
% \DescribeMacro{\VMTVP}
% tiene 2 argumentos,\;\cs{VMTV}\marg{nombre}\marg{nombre},\; y
% denota el producto vector por matriz por vector
% \begin{center}
%   |\VMTV{a}{B} \VMTVp{a}{B} \VMTVP{a}{B}| 
%   \hspace{1.2cm}
%   \fbox{$\VMTV  {a}{B}$}
%   \fbox{$\VMTVp*{a}{B}$}
%   \fbox{$\VMTVP*{a}{B}$}
% \end{center}
% 
% \DescribeMacro{\InvMTM}
% tiene 1 argumento,\;\cs{InvMTM}\marg{nombre},\; y
% denota la inversa del producto de una matriz transpuesta por ella misma
% \begin{center}
%   |\InvMTM{A}| 
%   \hspace{1.2cm}
%   \fbox{$\InvMTM{A}$}
% \end{center}
% 
% \subsection{Sistemas de ecuaciones}
% 
% \DescribeMacro{\SEL}
% tiene 3 argumentos,\;\cs{SEL}\marg{nombre}\marg{nombre}\marg{nombre},\; y
% denota un sistema de ecuaciones lineales (con notación matricial)
% \begin{center}
%   |\SEL{A}{x}{b}| 
%   \hspace{1.2cm}
%   \fbox{$\SEL{A}{x}{b}$}
% \end{center}
% 
% \DescribeMacro{\SELT}
% tiene 3
% argumentos,\;\cs{SELT}\marg{nombre}\marg{nombre}\marg{nombre},\; y
% denota un sistema de ecuaciones lineales (con notación matricial y
% matriz de coeficientes transpuesta)
% \begin{center}
%   |\SELT{A}{x}{b}| 
%   \hspace{1.2cm}
%   \fbox{$\SELT{A}{x}{b}$}
% \end{center}
% 
% \DescribeMacro{\SELTP}
% tiene 3
% argumentos,\;\cs{SELTP}\marg{nombre}\marg{nombre}\marg{nombre},\; y
% denota un sistema de ecuaciones lineales (con notación matricial y
% matriz de coeficientes transpuesta entre paréntesis)
% \begin{center}
%   |\SELTP{A}{x}{b}| 
%   \hspace{1.2cm}
%   \fbox{$\SELTP{A}{x}{b}$}
% \end{center}
% 
% \DescribeMacro{\SELF}
% tiene 3
% argumentos,\;\cs{SELF}\marg{nombre}\marg{nombre}\marg{nombre},\; y
% denota un sistema de ecuaciones lineales en forma de combinaciones de
% lineales de las filas de la matriz de coeficientes (con notación
% matricial)
% \begin{center}
%   |\SELF{y}{A}{b}| 
%   \hspace{1.2cm}
%   \fbox{$\SELF{y}{A}{b}$}
% \end{center}
% 
% \subsection{Espacios vectoriales}
% 
% \DescribeMacro{\EV}
% tiene 1
% argumento,\;\cs{EV}\marg{nombre},\; y
% denota un espacio vectorial
% \begin{center}
%   |\EV{A} \EV{V} \EV{E}| 
%   \hspace{1.2cm}
%   \fbox{$\EV{A} \EV{V} \EV{E}$}
% \end{center}
% 
% \DescribeMacro{\EspacioNul}
% no tiene argumentos y denota al espacio nulo (o núcleo) 
% \begin{center}
%   |\EspacioNul| 
%   \hspace{1.2cm}
%   \fbox{$\EspacioNul$}
% \end{center}
% 
% \DescribeMacro{\EspacioCol}
% no tiene argumentos y denota al espacio columna 
% \begin{center}
%   |\EspacioCol| 
%   \hspace{1.2cm}
%   \fbox{$\EspacioCol$}
% \end{center}
% 
% \DescribeMacro{\Nulls}
% tiene 1 argumento,\;\cs{Nulls}\marg{objeto},\; y denota el espacio
% nulo (o núcleo) del objeto
% \begin{center}
%   |\Nulls{f}| 
%   \hspace{1.2cm}
%   \fbox{$\Nulls{f}$}
% \end{center}
% 
% \DescribeMacro{\nulls}
% tiene 1 argumento,\;\cs{nulls}\marg{nombre},\; y denota el espacio
% nulo (o núcleo) de una matriz
% \begin{center}
%   |\nulls{A}| 
%   \hspace{1.2cm}
%   \fbox{$\nulls{A}$}
% \end{center}
% 
% \DescribeMacro{\Cols}
% tiene 1 argumento,\;\cs{Cols}\marg{objeto},\; y denota el espacio
% columna del objeto
% \begin{center}
%   |\Cols{f}| 
%   \hspace{1.2cm}
%   \fbox{$\Cols{f}$}
% \end{center}
% 
% \DescribeMacro{\cols}
% tiene 1 argumento,\;\cs{cols}\marg{nombre},\; y denota el espacio
% columna de una matriz
% \begin{center}
%   |\cols{A}| 
%   \hspace{1.2cm}
%   \fbox{$\cols{A}$}
% \end{center}
% 
% \DescribeMacro{\Span}
% tiene 1 argumento,\;\cs{Span}\marg{sistema},\; y denota el espacio
% vectorial generado con los elementos del \marg{sistema} o conjunto
% \begin{center}
%   |\Span{\SV{Z}}| 
%   \hspace{1.2cm}
%   \fbox{$\Span{\SV{Z}}$}
% \end{center}
% 
% \DescribeMacro{\PSpan}
% tiene 1 argumento,\;\cs{PSpan}\marg{sistema},\; y denota el espacio semi-euclídeo de probabilidad generado con los elementos del \marg{sistema} o conjunto
% \begin{center}
%   |\PSpan{\SV{Z}}| 
%   \hspace{1.2cm}
%   \fbox{$\PSpan{\SV{Z}}$}
% \end{center}
% 
% \DescribeMacro{\coord}
% \DescribeMacro{\coordP}
% \DescribeMacro{\coordPE}
% tiene 1 argumento,\;\cs{coord}\marg{vector}\marg{base},\; y denota
% las coordenadas de un vector respecto de una base
% \begin{center}
%   |\coord{\vect{x}}{\SV{Z}}| 
%   \hspace{1.2cm}
%   \fbox{$\coord{\vect{x}}{\SV{Z}}$}
% \end{center}
% \begin{center}
%   |\coordP{\vect{x}+\vect{y}}{\SV{Z}} \coordPE{\Vect{x}}{\Mat{B}}| 
%   \hspace{0.2cm}
%   \fbox{$\coordP* {\vect{x}+\vect{y}}{\SV{Z}}$}
%   \fbox{$\coordPE*{\Vect{x}}{\Mat{B}}$}
% \end{center}
% 
% \subsection{Notación funcional}
% 
% \DescribeMacro{\dom}
% El comando \cs{dom} no  tiene 
% argumentos y denota el \emph{dominio} de una función
% \begin{center}
%   |\dom(f)| 
%   \hspace{1.2cm}
%   \fbox{$\dom(f)$}
% \end{center}
% 
% \DescribeMacro{\mifun}
% tiene 3
% argumentos,\;\cs{mifun}\marg{nombre}\marg{dominio}\marg{conjLlegada},\;
% y denota una función que asigna a los elementos de su dominio
% elementos del \emph{conjunto de llegada}
% \begin{center}
%   |\mifun{f}{X}{Y}| 
%   \hspace{1.2cm}
%   \fbox{$\mifun{f}{X}{Y}$}
% \end{center}
% 
% \DescribeMacro{\deffun}
% 
% tiene 3 argumentos,
% \;\cs{deffun}\marg{nombre}\marg{dominio}\marg{conjLlegada}\marg{variable}\marg{imagen},\;
% y denota una función que asigna a los elementos de su dominio
% elementos del \emph{conjunto de llegada}
% \begin{center}
%   |\deffun{[f \circ g ]}{\R{}}{\Rr{n}}{x}{\Vect{x}}| 
%   \hspace{1.2cm}
%   \fbox{$\deffun{[f \circ g ]}{\Rr}{\Rr^{n}}{x}{\Vect{x}}$}
% \end{center}
% 
% 
% %%%%%%%%%%%%%%%%%%%%%%%%%%%%%%%%%%%%%%%%%%%%%
% 
% \subsection{Estadística}
% 
% \DescribeMacro{\Estmc}
% El comando \cs{Estmc}\marg{objeto} tiene 1
% argumento y denota el ajuste MCO del \marg{objeto}\begin{center}
%   |\Estmc{A}|
%   \hspace{1.2cm}
%   \fbox{$\Estmc{A}$}
% \end{center}
% 
% \DescribeMacro{\Media}
% El comando \cs{Media}\marg{objeto} tiene 1
% argumento y pinta una barra horizontal que denota la media (proyección ortogonal sobre los vectores contantes) del \marg{objeto}
% \begin{center}
%   |\Media{\Vect{x}}|
%   \hspace{1.2cm}
%   \fbox{$\Media{\Vect{x}}$}
% \end{center}
% 
% 
% \DescribeMacro{\Smedia}
% El comando \cs{Smedia} no tiene
% argumentos y pinta el símbolo del valor medio
% \begin{center}
%   |\Smedia|
%   \hspace{1.2cm}
%   \fbox{$\Smedia$}
% \end{center}
% 
% \DescribeMacro{\media}
% El comando \cs{media} tiene 1 argumento, \cs{Media}\marg{objeto}, y
% denota el valor medio del objeto.
% \begin{center}
%   |\media{\Vect{x}} \media{\Vect{x}}^2 \media{} |
%   \hspace{1.2cm}
%   \fbox{$\media{\Vect{x}}$}
%   \fbox{$\media{\Vect{x}}^2$}
%   \fbox{$\media{}$}
% \end{center}
% 
% 
% \StopEventually{\PrintChanges\PrintIndex}
%
% \section{Implementación}
%
%
% \iffalse
%%%%%%%%%%%%%%%%%%%%%%%%%%%%%%%%%%%%
%% --- Conjuntos de números
%%%%%%%%%%%%%%%%%%%%%%%%%%%%%%%%%%%%
% \fi
%
% \subsection{Conjuntos de números}
%
% \begin{macro}{\Nn}
% \begin{macro}{\Zz}
% \begin{macro}{\Rr}
% \begin{macro}{\CC}
% Números naturales, enteros, reales y complejos
%    \begin{macrocode}
\html@def\Nn{\mathbb{N}}
\html@def\Zz{\mathbb{Z}}
\html@def\Rr{\mathbb{R}}
\html@def\CC{\mathbb{C}}
%    \end{macrocode}
% \end{macro}
% \end{macro}
% \end{macro}
% \end{macro}
%
%
% \subsection{Paréntesis y corchetes}
%
% \iffalse
%%%%%%%%%%%%%%%%%%%%%%%%%%%%%%%%%%%%
%% --- Paréntesis y corchetes
%%%%%%%%%%%%%%%%%%%%%%%%%%%%%%%%%%%%
% \fi
%
% \begin{macro}{\parentesis}
% \begin{macro}{\Parentesis}
% Paréntesis pequeños
%    \begin{macrocode}
\html@def\parentesis#1{(#1)}
\html@def\Parentesis#1{\left(#1\right)}
%    \end{macrocode}
% \end{macro}
% \end{macro}
%
%
% \begin{macro}{\corchetes}
% \begin{macro}{\Corchetes}
% Corchetes pequeños
%    \begin{macrocode}
\html@def\corchetes#1{[#1]}
\html@def\Corchetes#1{\left[#1\right]}
%    \end{macrocode}
% \end{macro}
% \end{macro}
%
%
% \begin{macro}{\Corchetes}
% Corchetes de tamaño variable
%    \begin{macrocode}
\html@def\Corchetes#1{\left[#1\right]}
%    \end{macrocode}
% \end{macro}
%
%
% \subsection{Subíndices}
%
% \iffalse
%%%%%%%%%%%%%%%%%%%%%%%%%%%%%%%%%%%%
%% --- Algunos operadores (subíndices a derecha e izquierda)
%%%%%%%%%%%%%%%%%%%%%%%%%%%%%%%%%%%%
% \fi
%
% \begin{macro}{\LRidxE}
% \begin{macro}{\LidxE}
% \begin{macro}{\RidxE}
% Comando para escribir un índice a la derecha y otro a la izquierda de un objeto (con exponente)
%    \begin{macrocode}
\html@def\LRidxE#1#2#3#4{ {_{#2}^{}}{{#1}}{_{#3}^{#4}} }
\html@def\LidxE   #1#2#3{ {_{#2}^{}}{{#1}}{_{  }^{#3}} }
\html@def\RidxE   #1#2#3{ {        }{{#1}}{_{#2}^{#3}} }
%    \end{macrocode}
% \end{macro}
% \end{macro}
% \end{macro}
%
%
% \begin{macro}{\LRidx}
% \begin{macro}{\LRidxp}
% \begin{macro}{\LRidxP}
% \begin{macro}{\LRidxpE}
% \begin{macro}{\LRidxPE}
% Comando para escribir un índice a la derecha y otro a la izquierda de un objeto
%    \begin{macrocode}
\html@def\LRidx  #1#2#3{ \LRidxE{#1}{#2}{#3}{} }
\html@def\LRidxp #1#2#3{ \LRidxE{\parentesis{#1}}{#2}{#3}{} }
\html@def\LRidxP #1#2#3{ \LRidxE{\Parentesis{#1}}{#2}{#3}{} }
\html@def\LRidxpE#1#2#3{ \parentesis{\LRidxE{#1}{#2}{#3}{}} }
\html@def\LRidxPE#1#2#3{ \Parentesis{\LRidxE{#1}{#2}{#3}{}} }
%    \end{macrocode}
% \end{macro}
% \end{macro}
% \end{macro}
% \end{macro}
% \end{macro}
%
%
% \begin{macro}{\Lidx}
% \begin{macro}{\Lidxp}
% \begin{macro}{\LidxP}
% \begin{macro}{\LidxpE}
% \begin{macro}{\LidxPE}
% Comando para escribir un índice a la izquierda de un objeto
%    \begin{macrocode}
\html@def\Lidx  #1#2{ \LidxE{#1}{#2}{} }
\html@def\Lidxp #1#2{ \Lidx{\parentesis{#1}}{#2}{} }
\html@def\LidxP #1#2{ \Lidx{\Parentesis{#1}}{#2}{} }
\html@def\LidxpE#1#2{ \parentesis{\Lidx{#1}{#2}{}} }
\html@def\LidxPE#1#2{ \Parentesis{\Lidx{#1}{#2}{}} }
%    \end{macrocode}
% \end{macro}
% \end{macro}
% \end{macro}
% \end{macro}
% \end{macro}
%
%
% \begin{macro}{\Ridx}
% \begin{macro}{\Ridxp}
% \begin{macro}{\RidxP}
% \begin{macro}{\RidxpE}
% \begin{macro}{\RidxPE}
% Comando para escribir un índice a la derecha de un objeto
%    \begin{macrocode}
\html@def\Ridx  #1#2{ \RidxE{#1}{#2}{} }
\html@def\Ridxp #1#2{ \Ridx{\parentesis{#1}}{#2}{} }
\html@def\RidxP #1#2{ \Ridx{\Parentesis{#1}}{#2}{} }
\html@def\RidxpE#1#2{ \parentesis{\Ridx{#1}{#2}{}} }
\html@def\RidxPE#1#2{ \Parentesis{\Ridx{#1}{#2}{}} }
%    \end{macrocode}
% \end{macro}
% \end{macro}
% \end{macro}
% \end{macro}
% \end{macro}
%
%
% \subsection{Operadores}
%
% \iffalse
%%%%%%%%%%%%%%%%%%%%%%%%%%%%%%%%%%%%
%% --- ALGUNOS OPERADORES
%%%%%%%%%%%%%%%%%%%%%%%%%%%%%%%%%%%%
% \fi
%
%
% \subsubsection{Conjugación y concatenación}
%
% \iffalse
%%%%%%%%%%%%%%%%%%%%%%%%%%%%%%%%%%%%
%% --- Conjugación y concatenación
%%%%%%%%%%%%%%%%%%%%%%%%%%%%%%%%%%%%
% \fi
% \begin{macro}{\widebar}
% Barra ancha para indicar media o conjugación
%    \begin{macrocode}
\html@def\widebar#1{\mathop{\overline{#1}}}
%    \end{macrocode}
% \end{macro}
%
%
% \begin{macro}{\conj}
% Signo de conjugación
%    \begin{macrocode}
\html@def\conj#1{\widebar{#1}}
%    \end{macrocode}
% \end{macro}
%
%
% \begin{macro}{\concat}
% Concatenación
%    \begin{macrocode}
\html@def\concat{\large\&\#x29FA;}
%    \end{macrocode}
% \end{macro}
%
%
% \subsubsection{Norma y valor absoluto}
%
% \iffalse
%%%%%%%%%%%%%%%%%%%%%%%%%%%%%%%%%%%%
%% --- Normas y valor absoluto
%%%%%%%%%%%%%%%%%%%%%%%%%%%%%%%%%%%%
% \fi
%
% \begin{macro}{\norma}
% Norma de un objeto
%    \begin{macrocode}
\html@def\norma#1{\left\lVert{#1}\right\rVert}
%    \end{macrocode}
% \end{macro}
%
%
% \begin{macro}{\modulus}
% Valor absoluto
%    \begin{macrocode}
\html@def\modulus#1{\left|{#1}\right|}
%    \end{macrocode}
% \end{macro}
%
%
% \subsubsection{Transposición}
%
% \iffalse
%%%%%%%%%%%%%%%%%%%%%%%%%%%%%%%%%%%%
%% --- Transposición
%%%%%%%%%%%%%%%%%%%%%%%%%%%%%%%%%%%%
% \fi
%
%
%
% \begin{macro}{\T}
% Signo de transposición
%    \begin{macrocode}
\html@def\T{\intercal}
%    \end{macrocode}
% \end{macro}
%
%
% \begin{macro}{\Trans}
% \begin{macro}{\Transp}
% \begin{macro}{\TransP}
% \begin{macro}{\TranspE}
% \begin{macro}{\TransPE}
% Transposición
%    \begin{macrocode}
\html@def\Trans  #1{#1^{\mathbin{\T}}}
\html@def\Transp #1{\Trans{\parentesis{#1}}}
\html@def\TransP #1{\Trans{\Parentesis{#1}}}
\html@def\TranspE#1{\parentesis{\Trans{#1}}}
\html@def\TransPE#1{\Parentesis{\Trans{#1}}}
%    \end{macrocode}
% \end{macro}
% \end{macro}
% \end{macro}
% \end{macro}
% \end{macro}
%
%
% \subsubsection{Inversa}
%
% \iffalse
%%%%%%%%%%%%%%%%%%%%%%%%%%%%%%%%%%%%
%% --- Algunos operadores (Inversa)
%%%%%%%%%%%%%%%%%%%%%%%%%%%%%%%%%%%%
% \fi
%
% \begin{macro}{\minus}
% Signo negativo para indicar la inversa
%    \begin{macrocode}
\html@def\minus{\hbox{-}}
%    \end{macrocode}
% \end{macro}
%
% \begin{macro}{\Inv}
% \begin{macro}{\Invp}
% \begin{macro}{\InvP}
% \begin{macro}{\InvpE}
% \begin{macro}{\InvPE}
% Notación de la inversa
%    \begin{macrocode}
\html@def\Inv  #1{{#1}^{\minus1}}
\html@def\Invp #1{\Inv{\parentesis{#1}}}
\html@def\InvP #1{\Inv{\Parentesis{#1}}}
\html@def\InvpE#1{\parentesis{\Inv{#1}}}
\html@def\InvPE#1{\Parentesis{\Inv{#1}}}
%    \end{macrocode}
% \end{macro}
% \end{macro}
% \end{macro}
% \end{macro}
% \end{macro}
%
%
% \subsubsection{Operador selector}
%
% \iffalse
%%%%%%%%%%%%%%%%%%%%%%%%%%%%%%%%%%%%
%% --- OPERADOR SELECTOR
%%%%%%%%%%%%%%%%%%%%%%%%%%%%%%%%%%%%
% \fi
%
% \begin{macro}{\getItem}
% Signo de operador selector
%    \begin{macrocode}
\html@def\getItem{\mathbf{|}}
%    \end{macrocode}
% \end{macro}
%
%
% \begin{macro}{\getitemL}
% \begin{macro}{\getitemR}
% Operador selector por la izquierda y operador selector por la derecha
%    \begin{macrocode}
\html@def\getitemL#1{{#1}\mathbin{\getItem}}
\html@def\getitemR#1{\mathbin{\getItem}{#1}}
%    \end{macrocode}
% \end{macro}
% \end{macro}
%
%%%%%%%%%%%%%%%%%%%%%%%%%%%%%%%%%%%%%%%%%%%%%
% \textbf{por la izquierda de un objeto}
%
% \begin{macro}{\elemL}
% \begin{macro}{\elemLp}
% \begin{macro}{\elemLP}
% \begin{macro}{\elemLpE}
% \begin{macro}{\elemLPE}
% Selector por la izquierda
%    \begin{macrocode}
\html@def\elemL  #1#2{\Lidx{#1}{\getitemL{#2}}}
\html@def\elemLp #1#2{\elemL{\parentesis{#1}}{#2}}
\html@def\elemLP #1#2{\elemL{\Parentesis{#1}}{#2}}
\html@def\elemLpE#1#2{\parentesis{\elemL{#1}{#2}}}
\html@def\elemLPE#1#2{\Parentesis{\elemL{#1}{#2}}}
%    \end{macrocode}
% \end{macro}
% \end{macro}
% \end{macro}
% \end{macro}
% \end{macro}
%
%%%%%%%%%%%%%%%%%%%%%%%%%%%%%%%%%%%%%%%%%%%%%
% \textbf{por la derecha de un objeto}
%
% \begin{macro}{\elemR}
% \begin{macro}{\elemRp}
% \begin{macro}{\elemRP}
% \begin{macro}{\elemRpE}
% \begin{macro}{\elemRPE}
% Selector por la derecha
%    \begin{macrocode}
\html@def\elemR  #1#2{\Ridx{#1}{\getitemR{#2}}}
\html@def\elemRp #1#2{\elemR{\parentesis{#1}}{#2}}
\html@def\elemRP #1#2{\elemR{\Parentesis{#1}}{#2}}
\html@def\elemRpE#1#2{\parentesis{\elemR{#1}{#2}}}
\html@def\elemRPE#1#2{\Parentesis{\elemR{#1}{#2}}}
%    \end{macrocode}
% \end{macro}
% \end{macro}
% \end{macro}
% \end{macro}
% \end{macro}
%
%%%%%%%%%%%%%%%%%%%%%%%%%%%%%%%%%%%%%%%%%%%%%
% \textbf{por ambos lados de un objeto}
%
% \begin{macro}{\elemLR}
% \begin{macro}{\elemLRp}
% \begin{macro}{\elemLRP}
% \begin{macro}{\elemLRpE}
% \begin{macro}{\elemLRPE}
% Selectores por ambos lados
%    \begin{macrocode}
\html@def\elemLR  #1#2#3{\LRidx{#1}{\getitemL{#2}}{\getitemR{#3}}}
\html@def\elemLRp #1#2#3{\elemLR{\parentesis{#1}}{#2}{#3}}
\html@def\elemLRP #1#2#3{\elemLR{\Parentesis{#1}}{#2}{#3}}
\html@def\elemLRpE#1#2#3{\parentesis{\elemLR{#1}{#2}{#3}}}
\html@def\elemLRPE#1#2#3{\Parentesis{\elemLR{#1}{#2}{#3}}}
%    \end{macrocode}
% \end{macro}
% \end{macro}
% \end{macro}
% \end{macro}
% \end{macro}
%
% \iffalse
%%%%%%%%%%%%%%%%%%%%%%%%%%%%%%%%%%%%
%% --- Operador selector con vectores
%%%%%%%%%%%%%%%%%%%%%%%%%%%%%%%%%%%%
% \fi
%
%%%%%%%%%%%%%%%%%%%%%%%%%%%%%%%%%%%%%%
% \textbf{por la izquierda de un vector}
%
% \begin{macro}{\eleVL}
% \begin{macro}{\eleVLp}
% \begin{macro}{\eleVLP}
% \begin{macro}{\eleVLpE}
% \begin{macro}{\eleVLPE}
% Selector de elementos de un vector por la izquierda
%    \begin{macrocode}
\html@def\eleVL  #1#2{\elemL  {\Vect{#1}}{#2}}
\html@def\eleVLp #1#2{\elemLp {\Vect{#1}}{#2}}
\html@def\eleVLP #1#2{\elemLP {\Vect{#1}}{#2}}
\html@def\eleVLpE#1#2{\elemLpE{\Vect{#1}}{#2}}
\html@def\eleVLPE#1#2{\elemLPE{\Vect{#1}}{#2}}
%    \end{macrocode}
% \end{macro}
% \end{macro}
% \end{macro}
% \end{macro}
% \end{macro}
%
%%%%%%%%%%%%%%%%%%%%%%%%%%%%%%%%%%%%%%%%%%%%%
% \textbf{por la derecha de un vector}
%
% \begin{macro}{\eleVR}
% \begin{macro}{\eleVRp}
% \begin{macro}{\eleVRP}
% \begin{macro}{\eleVRpE}
% \begin{macro}{\eleVRPE}
% Selector de elementos de un vector por la derecha
%    \begin{macrocode}
\html@def\eleVR  #1#2{\elemR  {\Vect{#1}}{#2}}
\html@def\eleVRp #1#2{\elemRp {\Vect{#1}}{#2}}
\html@def\eleVRP #1#2{\elemRP {\Vect{#1}}{#2}}
\html@def\eleVRpE#1#2{\elemRpE{\Vect{#1}}{#2}}
\html@def\eleVRPE#1#2{\elemRPE{\Vect{#1}}{#2}}
%    \end{macrocode}
% \end{macro}
% \end{macro}
% \end{macro}
% \end{macro}
% \end{macro}
%
% \iffalse
%%%%%%%%%%%%%%%%%%%%%%%%%%%%%%%%%%%%
%% --- Operador selector con matrices
%%%%%%%%%%%%%%%%%%%%%%%%%%%%%%%%%%%%
% \fi
%
% \textbf{de filas de una matriz}

% \begin{macro}{\VectF}
% \begin{macro}{\VectFp}
% \begin{macro}{\VectFP}
% \begin{macro}{\VectFpE}
% \begin{macro}{\VectFPE}
% Selector de filas de una matriz
%    \begin{macrocode}
\html@def\VectF  #1#2{\elemL  {\Mat{#1}}{#2}}
\html@def\VectFp #1#2{\elemLp {\Mat{#1}}{#2}}
\html@def\VectFP #1#2{\elemLP {\Mat{#1}}{#2}}
\html@def\VectFpE#1#2{\elemLpE{\Mat{#1}}{#2}}
\html@def\VectFPE#1#2{\elemLPE{\Mat{#1}}{#2}}
%    \end{macrocode}
% \end{macro}
% \end{macro}
% \end{macro}
% \end{macro}
% \end{macro}
%
%%%%%%%%%%%%%%%%%%%%%%%%%%%%%%%%%%%%%%%%%%%%%
% \textbf{de columnas de una matriz}
%
% \begin{macro}{\VectC}
% \begin{macro}{\VectCp}
% \begin{macro}{\VectCP}
% \begin{macro}{\VectCpE}
% \begin{macro}{\VectCPE}
% Selector de columnas de una matriz
%    \begin{macrocode}
\html@def\VectC  #1#2{\elemR  {\Mat{#1}}{#2}}
\html@def\VectCp #1#2{\elemRp {\Mat{#1}}{#2}}
\html@def\VectCP #1#2{\elemRP {\Mat{#1}}{#2}}
\html@def\VectCpE#1#2{\elemRpE{\Mat{#1}}{#2}}
\html@def\VectCPE#1#2{\elemRPE{\Mat{#1}}{#2}}
%    \end{macrocode}
% \end{macro}
% \end{macro}
% \end{macro}
% \end{macro}
% \end{macro}
%
%%%%%%%%%%%%%%%%%%%%%%%%%%%%%%%%%%%%%%%%%%%%%
% \textbf{de elementos de una matriz}
%
% \begin{macro}{\eleM}
% \begin{macro}{\eleMp}
% \begin{macro}{\eleMP}
% \begin{macro}{\eleMpE}
% \begin{macro}{\eleMPE}
% Selector de elementos de una matriz
%    \begin{macrocode}
\html@def\eleM  #1#2#3{\elemLR  {\Mat{#1}}{#2}{#3}}
\html@def\eleMp #1#2#3{\elemLRp {\Mat{#1}}{#2}{#3}}
\html@def\eleMP #1#2#3{\elemLRP {\Mat{#1}}{#2}{#3}}
\html@def\eleMpE#1#2#3{\elemLRpE{\Mat{#1}}{#2}{#3}}
\html@def\eleMPE#1#2#3{\elemLRPE{\Mat{#1}}{#2}{#3}}
%    \end{macrocode}
% \end{macro}
% \end{macro}
% \end{macro}
% \end{macro}
% \end{macro}
%
%%%%%%%%%%%%%%%%%%%%%%%%%%%%%%%%%%%%%%%%%%%%%
% \textbf{de elementos de una matriz transpuesta}
%
% \begin{macro}{\eleMT}
% \begin{macro}{\eleMTp}
% \begin{macro}{\eleMTP}
% \begin{macro}{\eleMTpE}
% \begin{macro}{\eleMTPE}
% Selector de elementos de una matriz transpuesta
%    \begin{macrocode}
\html@def\eleMT  #1#2#3{\elemLRP {\MatT  {#1}}{#2}{#3}}
\html@def\eleMTp #1#2#3{\elemLRp {\MatTpE{#1}}{#2}{#3}}
\html@def\eleMTP #1#2#3{\elemLRP {\MatTPE{#1}}{#2}{#3}}
\html@def\eleMTpE#1#2#3{\elemLRpE{\MatTpE{#1}}{#2}{#3}}
\html@def\eleMTPE#1#2#3{\elemLRPE{\MatTPE{#1}}{#2}{#3}}
%    \end{macrocode}
% \end{macro}
% \end{macro}
% \end{macro}
% \end{macro}
% \end{macro}
%
%
% \subsubsection{Operaciones elementales}
%
% \iffalse
%%%%%%%%%%%%%%%%%%%%%%%%%%%%%%%%%%%%
%% --- Operaciones elementales y permutaciones
%%%%%%%%%%%%%%%%%%%%%%%%%%%%%%%%%%%%
% \fi
%
% \begin{macro}{\TrEl}
% Signo de transformación elemental
%    \begin{macrocode}
\html@def\TrEl{\boldsymbol{\tau}}
%    \end{macrocode}
% \end{macro}
%
%
% \begin{macro}{\su}
% Transformación elemental Tipo I
%    \begin{macrocode}
\html@def\su#1#2#3{\left(#1\right){\boldsymbol{#2}}+{\boldsymbol{#3}}}
%    \end{macrocode}
% \end{macro}
%
%
% \begin{macro}{\pr}
% Transformación elemental Tipo II
%    \begin{macrocode}
\html@def\pr#1#2{\left(#1\right){\boldsymbol{#2}}}
%    \end{macrocode}
% \end{macro}
%
%
% \begin{macro}{\pe}
% Intercambio (permuta de dos elementos)
%    \begin{macrocode}
\html@def\pe#1#2{\boldsymbol{#1} \rightleftharpoons \boldsymbol{#2}}
%    \end{macrocode}
% \end{macro}
%
%
% \begin{macro}{\perm}
% Reordenamiento de los elementos (permutación)
%    \begin{macrocode}
\html@def\perm{\mathfrak{S}}
%    \end{macrocode}
% \end{macro}
%
%%%%%%%%%%%%%%%%%%%%%%%%%%%%%%%%%%%%%%%%%%%%%%%%%%%%%%%%%%%
%
%
% \begin{macro}{\OpE}
% Operación elemental
%    \begin{macrocode}
\html@def\OpE#1{\underset{\left[{#1}\right]}{\TrEl}}
%    \end{macrocode}
% \end{macro}
%
%
% \begin{macro}{\OEsu}
% Oper. elem. que suma un múltiplo de una componente a otra
%    \begin{macrocode}
\html@def\OEsu#1#2#3{\OpE{ \su{#1}{#2}{#3} }}
%    \end{macrocode}
% \end{macro}
%
%
% \begin{macro}{\OEpr}
% Oper. elem. que multiplica una componente por un número
%    \begin{macrocode}
\html@def\OEpr#1#2{\OpE{ \pr{#1}{#2} }}
%    \end{macrocode}
% \end{macro}
%
%
% \begin{macro}{\OEin}
% Intercambio de posición entre componentes
%    \begin{macrocode}
\html@def\OEin#1#2{\OpE{ \pe{#1}{#2} }}
%    \end{macrocode}
% \end{macro}
%
%
% \begin{macro}{\OEper}
% Reordenamiento o permutación entre componentes
%    \begin{macrocode}
\html@def\OEper{\OpE{ \perm }}
%    \end{macrocode}
% \end{macro}
%
%
% \begin{macro}{\EOEsu}
% Espejo de oper. elem. que suma un múltiplo de una componente a otra
%    \begin{macrocode}
\html@def\EOEsu#1#2#3{esp\Big(\OEsu{#1}{#2}{#3}\Big)}
%    \end{macrocode}
% \end{macro}
%
%
% \begin{macro}{\EOEpr}
% Espejo de oper. elem. que multiplica una componente por un número
%    \begin{macrocode}
\html@def\EOEpr#1#2{esp\Big(\Oepr{#1}{#2}\Big)}
%    \end{macrocode}
% \end{macro}
%
%
% \paragraph{Transformaciones elementales generales}
%
% \iffalse
%%%%%%%%%%%%%%%%%%%%%%%%%%%%%%%%%%%%
%% --- Operaciones elementales genéricas
%%%%%%%%%%%%%%%%%%%%%%%%%%%%%%%%%%%%
% \fi
%
% \begin{macro}{\dOEgE}
% \begin{macro}{\dOEg}
% Operación elemental genérica con exponente y sin exponente
%    \begin{macrocode}
\html@def\dOEgE#1#2{\RidxE{\TrEl}{#1}{#2}}
\html@def\dOEg#1{\dOEgE{#1}{}}
%    \end{macrocode}
% \end{macro}
% \end{macro}
%
%
% \begin{macro}{\dEOEgE}
% \begin{macro}{\dEOEg}
% Operación espejo de una elemental genérica con exponente y sin exponente
%    \begin{macrocode}
\html@def\dEOEgE#1#2{esp(\dOEgE{#1}{#2})}
\html@def\dEOEg#1{esp(\dOEg{#1})}
%    \end{macrocode}
% \end{macro}
% \end{macro}
%
%
% \begin{macro}{\dInvOEg}
% Operación inversa de una elemental genérica
%    \begin{macrocode}
\html@def\dInvOEg#1{\dOEgE{#1}{\minus1}}
%    \end{macrocode}
% \end{macro}
%
%
% \begin{macro}{\dEInvOEg}
% Operación espejo de la inversa de una elemental genérica
%    \begin{macrocode}
\html@def\dEInvOEg#1{esp(\dInvOEg{#1})}
%    \end{macrocode}
% \end{macro}
%
%
% \iffalse
%%%%%%%%%%%%%%%%%%%%%%%%%%%%%%%%%%%%
%% --- Sucesión de operaciones elementales genéricas
%%%%%%%%%%%%%%%%%%%%%%%%%%%%%%%%%%%%
% \fi
%
%
% \begin{macro}{\dSOEgE}
% \begin{macro}{\dSOEg}
% Sucesión de operaciones elementales genéricas con exponente y sin exponente
%    \begin{macrocode}
\html@def\dSOEgE#1#2#3{\dOEgE{#1}{#3}\cdots\dOEgE{#2}{#3}}
\html@def\dSOEg#1#2{\dOEg{#1}\cdots\dOEg{#2}}
%    \end{macrocode}
% \end{macro}
% \end{macro}
%
% \subsubsection{Transformaciones elementales}
%
% \iffalse
%%%%%%%%%%%%%%%%%%%%%%%%%%%%%%%%%%%%
%% --- Transformaciones elementales y permutaciones
%%%%%%%%%%%%%%%%%%%%%%%%%%%%%%%%%%%%
% \fi
%
% \paragraph{Transf. elemental aplicada la izquierda o derecha de un objeto}
%
% \iffalse
%%%%%%%%%%%%%%%%%%%%%%%%%%%%%%%%%%%%
%% --- Transf. elemental aplicada la izquierda o derecha de un objeto
%%%%%%%%%%%%%%%%%%%%%%%%%%%%%%%%%%%%
% \fi
%
%%%%%%%%%%%% Tipo I - Fil %%%%%%%%%%%%%%%%%%%%%%%%%%%%%%
%
% \begin{macro}{\TESF}
% \begin{macro}{\TESFp}
% \begin{macro}{\TESFP}
% \begin{macro}{\TESFpE}
% \begin{macro}{\TESFPE}
% Una transformación elemental Tipo I por la izquierda
%    \begin{macrocode}
\html@def\TESF  #1#2#3#4{ \Lidx  {#4}{ \OEsu{#1}{#2}{#3}\!  } }
\html@def\TESFp #1#2#3#4{ \Lidxp {#4}{ \OEsu{#1}{#2}{#3}\!\!} }
\html@def\TESFP #1#2#3#4{ \LidxP {#4}{ \OEsu{#1}{#2}{#3}\!\!} }
\html@def\TESFpE#1#2#3#4{ \LidxpE{#4}{ \OEsu{#1}{#2}{#3}\!\!} }
\html@def\TESFPE#1#2#3#4{ \LidxPE{#4}{ \OEsu{#1}{#2}{#3}\!\!} }
%    \end{macrocode}
% \end{macro}
% \end{macro}
% \end{macro}
% \end{macro}
% \end{macro}
%
%%%%%%%%%%%% Tipo I - Col %%%%%%%%%%%%%%%%%%%%%%%%%%%%%%
%
% \begin{macro}{\TESC}
% \begin{macro}{\TESCp}
% \begin{macro}{\TESCP}
% \begin{macro}{\TESCpE}
% \begin{macro}{\TESCPE}
% Una transformación elemental Tipo I por la derecha
%    \begin{macrocode}
\html@def\TESC  #1#2#3#4{ \Ridx  {#4}{  \!\OEsu{#1}{#2}{#3} } }
\html@def\TESCp #1#2#3#4{ \Ridxp {#4}{\!\!\OEsu{#1}{#2}{#3} } }
\html@def\TESCP #1#2#3#4{ \RidxP {#4}{\!\!\OEsu{#1}{#2}{#3} } }
\html@def\TESCpE#1#2#3#4{ \RidxpE{#4}{  \!\OEsu{#1}{#2}{#3} } }
\html@def\TESCPE#1#2#3#4{ \RidxPE{#4}{  \!\OEsu{#1}{#2}{#3} } }
%    \end{macrocode}
% \end{macro}
% \end{macro}
% \end{macro}
% \end{macro}
% \end{macro}
%
%%%%%%%%%%%% Tipo II - Fil %%%%%%%%%%%%%%%%%%%%%%%%%%%%%%
%
% \begin{macro}{\TEPF}
% \begin{macro}{\TEPFp}
% \begin{macro}{\TEPFP}
% \begin{macro}{\TEPFpE}
% \begin{macro}{\TEPFPE}
% Una transformación elemental Tipo II por la izquierda
%    \begin{macrocode}
\html@def\TEPF  #1#2#3{ \Lidx  {#3}{ \OEpr{#1}{#2}\!  } }
\html@def\TEPFp #1#2#3{ \Lidxp {#3}{ \OEpr{#1}{#2}\!\!} }
\html@def\TEPFP #1#2#3{ \LidxP {#3}{ \OEpr{#1}{#2}\!\!} }
\html@def\TEPFpE#1#2#3{ \LidxpE{#3}{ \OEpr{#1}{#2}\!  } }
\html@def\TEPFPE#1#2#3{ \LidxPE{#3}{ \OEpr{#1}{#2}\!  } }
%    \end{macrocode}
% \end{macro}
% \end{macro}
% \end{macro}
% \end{macro}
% \end{macro}
%
%%%%%%%%%%%% Tipo II - Col %%%%%%%%%%%%%%%%%%%%%%%%%%%%%%
%
% \begin{macro}{\TEPC}
% \begin{macro}{\TEPCp}
% \begin{macro}{\TEPCP}
% \begin{macro}{\TEPCpE}
% \begin{macro}{\TEPCPE}
% Una transformación elemental Tipo II por la derecha
%    \begin{macrocode}
\html@def\TEPC  #1#2#3{ \Ridx  {#3}{  \!\OEpr{#1}{#2} } }
\html@def\TEPCp #1#2#3{ \Ridxp {#3}{\!\!\OEpr{#1}{#2} } }
\html@def\TEPCP #1#2#3{ \RidxP {#3}{\!\!\OEpr{#1}{#2} } }
\html@def\TEPCpE#1#2#3{ \RidxpE{#3}{  \!\OEpr{#1}{#2} } }
\html@def\TEPCPE#1#2#3{ \RidxPE{#3}{  \!\OEpr{#1}{#2} } }
%    \end{macrocode}
% \end{macro}
% \end{macro}
% \end{macro}
% \end{macro}
% \end{macro}
%
%%%%%%%%%%%% Intercambio - Fil %%%%%%%%%%%%%%%%%%%%%%%%%%%%%%
%
% \begin{macro}{\TEIF}
% \begin{macro}{\TEIFp}
% \begin{macro}{\TEIFP}
% \begin{macro}{\TEIFpE}
% \begin{macro}{\TEIFPE}
% Intercambio por la izquierda
%    \begin{macrocode}
\html@def\TEIF  #1#2#3{ \Lidx  {#3}{ \OEin{#1}{#2}\!  } }
\html@def\TEIFp #1#2#3{ \Lidxp {#3}{ \OEin{#1}{#2}\!\!} }
\html@def\TEIFP #1#2#3{ \LidxP {#3}{ \OEin{#1}{#2}\!\!} }
\html@def\TEIFpE#1#2#3{ \LidxpE{#3}{ \OEin{#1}{#2}\!  } }
\html@def\TEIFPE#1#2#3{ \LidxPE{#3}{ \OEin{#1}{#2}\!  } }
%    \end{macrocode}
% \end{macro}
% \end{macro}
% \end{macro}
% \end{macro}
% \end{macro}
%
%
%%%%%%%%%%%% Intercambio - Col %%%%%%%%%%%%%%%%%%%%%%%%%%%%%%
%
% \begin{macro}{\TEIC}
% \begin{macro}{\TEICp}
% \begin{macro}{\TEICP}
% \begin{macro}{\TEICpE}
% \begin{macro}{\TEICPE}
% Intercambio por la derecha
%    \begin{macrocode}
\html@def\TEIC  #1#2#3{ \Ridx  {#3}{  \!\OEin{#1}{#2}  } }
\html@def\TEICp #1#2#3{ \Ridxp {#3}{\!\!\OEin{#1}{#2}  } }
\html@def\TEICP #1#2#3{ \RidxP {#3}{\!\!\OEin{#1}{#2}  } }
\html@def\TEICpE#1#2#3{ \RidxpE{#3}{  \!\OEin{#1}{#2}  } }
\html@def\TEICPE#1#2#3{ \RidxPE{#3}{    \OEin{#1}{#2}  } }
%    \end{macrocode}
% \end{macro}
% \end{macro}
% \end{macro}
% \end{macro}
% \end{macro}
%
%%%%%%%%%%%%%%%%%%%%%%%%%%%%%%%%%%%%%%%%%%
%
% \begin{macro}{\Mint}
% Matriz intercambio
%    \begin{macrocode}
\html@def\Mint#1#2{ \TEIC{#1}{#2}{\Mat{I}} }
%    \end{macrocode}
% \end{macro}
%
%
% \begin{macro}{\MintT}
% Matriz intercambio (filas)
%    \begin{macrocode}
\html@def\MintT#1#2{ \TEIF{#1}{#2}{\Mat{I}} }
%    \end{macrocode}
% \end{macro}
%
%
% \begin{macro}{\PC}
% Permutación por la derecha
%    \begin{macrocode}
\html@def\PC#1{ \Ridx{#1}{\!\OEper} }
%    \end{macrocode}
% \end{macro}
%
%
% \begin{macro}{\PF}
% Permutación por la izquierda
%    \begin{macrocode}
\html@def\PC#1{ \Lidx{#1}{\!\OEper} }
%    \end{macrocode}
% \end{macro}
%
%
% \begin{macro}{\MP}
% Matriz permutación
%    \begin{macrocode}
\html@def\MP{ \PC{\Mat{I}} }
%    \end{macrocode}
% \end{macro}
%
%
% \begin{macro}{\MPT}
% Matriz permutación
%    \begin{macrocode}
\html@def\MPT{ \PF{\Mat{I}} }
%    \end{macrocode}
% \end{macro}
%
%
% \paragraph{Sucesiones indiciadas de Transf. elementales}
%
% \iffalse
%%%%%%%%%%%%%%%%%%%%%%%%%%%%%%%%%%%%
%% --- Sucesiones indiciadas de Transf. elementales
%%%%%%%%%%%%%%%%%%%%%%%%%%%%%%%%%%%%
% \fi
%
%
% \begin{macro}{\SITEF}
% \begin{macro}{\SITEFp}
% \begin{macro}{\SITEFP}
% \begin{macro}{\SITEFpE}
% \begin{macro}{\SITEFPE}
% Sucesión de transformaciones elementales genéricas por la izquierda (filas)
%    \begin{macrocode}
\html@def\SITEF  #1#2#3{\Lidx{#3}{\dSOEg{#1}{#2}}}
\html@def\SITEFp #1#2#3{\SITEF{#1}{#2}{\parentesis{#3}}}
\html@def\SITEFP #1#2#3{\SITEF{#1}{#2}{\Parentesis{#3}}}
\html@def\SITEFpE#1#2#3{\parentesis{\SITEF{#1}{#2}{#3}}}
\html@def\SITEFPE#1#2#3{\Parentesis{\SITEF{#1}{#2}{#3}}}
%    \end{macrocode}
% \end{macro}
% \end{macro}
% \end{macro}
% \end{macro}
% \end{macro}
%
%
% \begin{macro}{\SITEC}
% \begin{macro}{\SITECp}
% \begin{macro}{\SITECP}
% \begin{macro}{\SITECpE}
% \begin{macro}{\SITECPE}
% Sucesión de transformaciones elementales genéricas por la derecha (columnas)
%    \begin{macrocode}
\html@def\SITEC  #1#2#3{\Ridx{#3}{\dSOEg{#1}{#2}}}
\html@def\SITECp #1#2#3{\SITEC{#1}{#2}{\parentesis{#3}}}
\html@def\SITECP #1#2#3{\SITEC{#1}{#2}{\Parentesis{#3}}}
\html@def\SITECpE#1#2#3{\parentesis{\SITEC{#1}{#2}{#3}}}
\html@def\SITECPE#1#2#3{\Parentesis{\SITEC{#1}{#2}{#3}}}
%    \end{macrocode}
% \end{macro}
% \end{macro}
% \end{macro}
% \end{macro}
% \end{macro}
%
% \begin{macro}{\SITEFC}
% \begin{macro}{\SITEFCp}
% \begin{macro}{\SITEFCP}
% \begin{macro}{\SITEFCpE}
% \begin{macro}{\SITEFCPE}
% Sucesión de transformaciones elementales genéricas a izquierda y derecha
%    \begin{macrocode}
\html@def\SITEFC  #1#2#3{\LRidx{#3}{\dSOEg{#2}{#1}}{\dSOEg{#1}{#2}}}
\html@def\SITEFCp #1#2#3{\SITEFC{#1}{#2}{\parentesis{#3}}}
\html@def\SITEFCP #1#2#3{\SITEFC{#1}{#2}{\Parentesis{#3}}}
\html@def\SITEFCpE#1#2#3{\parentesis{\SITEFC{#1}{#2}{#3}}}
\html@def\SITEFCPE#1#2#3{\Parentesis{\SITEFC{#1}{#2}{#3}}}
%    \end{macrocode}
% \end{macro}
% \end{macro}
% \end{macro}
% \end{macro}
% \end{macro}
%
%
% \begin{macro}{\SITEFCR}
% \begin{macro}{\SITEFCRp}
% \begin{macro}{\SITEFCRP}
% \begin{macro}{\SITEFCRpE}
% \begin{macro}{\SITEFCRPE}
% Sucesión de transformaciones elementales genéricas a izquierda y derecha
%    \begin{macrocode}
\html@def\SITEFCR  #1#2#3{\LRidx{#3}{\dSOEg{#1}{#2}}{\dSOEg{#1}{#2}}}
\html@def\SITEFCRp #1#2#3{\SITEFCR{#1}{#2}{\parentesis{#3}}}
\html@def\SITEFCRP #1#2#3{\SITEFCR{#1}{#2}{\Parentesis{#3}}}
\html@def\SITEFCRpE#1#2#3{\parentesis{\SITEFCR{#1}{#2}{#3}}}
\html@def\SITEFCRPE#1#2#3{\Parentesis{\SITEFCR{#1}{#2}{#3}}}
%    \end{macrocode}
% \end{macro}
% \end{macro}
% \end{macro}
% \end{macro}
% \end{macro}
%
%%%%%%%%%%%%%%%%%%%%%%%%%%%%%%%%%%%%%%%%%%
%
% \paragraph{Transf. elemental aplicada la izquierda de un objeto}
%
% \iffalse
%%%%%%%%%%%%%%%%%%%%%%%%%%%%%%%%%%%%
%% --- Transformaciones elementales genéricas aplicadas a la izquierda de un objeto
%%%%%%%%%%%%%%%%%%%%%%%%%%%%%%%%%%%%
% \fi
%
% \begin{macro}{\dTEEF}
% \begin{macro}{\dTEEFp}
% \begin{macro}{\dTEEFP}
% \begin{macro}{\dTEEFpE}
% \begin{macro}{\dTEEFPE}
% Una transformación elemental genérica con exponente por la izquierda
%    \begin{macrocode}
\html@def\dTEEF  #1#2#3{ \Lidx  {#3}{\dOEgE{#1}{#2}} }
\html@def\dTEEFp #1#2#3{ \Lidxp {#3}{\dOEgE{#1}{#2}} }
\html@def\dTEEFP #1#2#3{ \LidxP {#3}{\dOEgE{#1}{#2}} }
\html@def\dTEEFpE#1#2#3{ \LidxpE{#3}{\dOEgE{#1}{#2}} }
\html@def\dTEEFPE#1#2#3{ \LidxpE{#3}{\dOEgE{#1}{#2}} }
%    \end{macrocode}
% \end{macro}
% \end{macro}
% \end{macro}
% \end{macro}
% \end{macro}
%
%
% \begin{macro}{\dTEF}
% \begin{macro}{\dTEFp}
% \begin{macro}{\dTEFP}
% \begin{macro}{\dTEFpE}
% \begin{macro}{\dTEFPE}
% Una transformación elemental genérica por la izquierda
%    \begin{macrocode}
\html@def\dTEF  #1#2{ \Lidx  {#2}{{\dOEg{#1}}} }
\html@def\dTEFp #1#2{ \Lidxp {#2}{{\dOEg{#1}}} }
\html@def\dTEFP #1#2{ \LidxP {#2}{{\dOEg{#1}}} }
\html@def\dTEFpE#1#2{ \LidxpE{#2}{{\dOEg{#1}}} }
\html@def\dTEFPE#1#2{ \LidxPE{#2}{{\dOEg{#1}}} }
%    \end{macrocode}
% \end{macro}
% \end{macro}
% \end{macro}
% \end{macro}
% \end{macro}
%
%
% \begin{macro}{\dETEF}
% \begin{macro}{\dETEFp}
% \begin{macro}{\dETEFP}
% \begin{macro}{\dETEFpE}
% \begin{macro}{\dETEFPE}
% Una transformación elemental espejo genérica por la izquierda
%    \begin{macrocode}
\html@def\dETEF  #1#2{ \Lidx  {#2}{{\dEOEg{#1}}} }
\html@def\dETEFp #1#2{ \Lidxp {#2}{{\dEOEg{#1}}} }
\html@def\dETEFP #1#2{ \LidxP {#2}{{\dEOEg{#1}}} }
\html@def\dETEFpE#1#2{ \LidxpE{#2}{{\dEOEg{#1}}} }
\html@def\dETEFPE#1#2{ \LidxPE{#2}{{\dEOEg{#1}}} }
%    \end{macrocode}
% \end{macro}
% \end{macro}
% \end{macro}
% \end{macro}
% \end{macro}
%
%
% \begin{macro}{\dInvTEF}
% \begin{macro}{\dInvTEFp}
% \begin{macro}{\dInvTEFP}
% \begin{macro}{\dInvTEFpE}
% \begin{macro}{\dInvTEFPE}
% Una transformación elemental inversa genérica por la izquierda
%    \begin{macrocode}
\html@def\dInvTEF  #1#2{ \Lidx  {#2}{{\dInvOEg{#1}}} }
\html@def\dInvTEFp #1#2{ \Lidxp {#2}{{\dInvOEg{#1}}} }
\html@def\dInvTEFP #1#2{ \LidxP {#2}{{\dInvOEg{#1}}} }
\html@def\dInvTEFpE#1#2{ \LidxpE{#2}{{\dInvOEg{#1}}} }
\html@def\dInvTEFPE#1#2{ \LidxPE{#2}{{\dInvOEg{#1}}} }
%    \end{macrocode}
% \end{macro}
% \end{macro}
% \end{macro}
% \end{macro}
% \end{macro}
%
%
% \begin{macro}{\dEInvTEF}
% \begin{macro}{\dEInvTEFp}
% \begin{macro}{\dEInvTEFP}
% \begin{macro}{\dEInvTEFpE}
% \begin{macro}{\dEInvTEFPE}
% Una transformación elemental inversa genérica por la izquierda
%    \begin{macrocode}
\html@def\dEInvTEF  #1#2{ \Lidx  {#2}{{\dEInvOEg{#1}}} }
\html@def\dEInvTEFp #1#2{ \Lidxp {#2}{{\dEInvOEg{#1}}} }
\html@def\dEInvTEFP #1#2{ \LidxP {#2}{{\dEInvOEg{#1}}} }
\html@def\dEInvTEFpE#1#2{ \LidxpE{#2}{{\dEInvOEg{#1}}} }
\html@def\dEInvTEFPE#1#2{ \LidxPE{#2}{{\dEInvOEg{#1}}} }
%    \end{macrocode}
% \end{macro}
% \end{macro}
% \end{macro}
% \end{macro}
% \end{macro}
%
% \paragraph{Transf. elemental aplicada la derecha de un objeto}
%
% \iffalse
%%%%%%%%%%%%%%%%%%%%%%%%%%%%%%%%%%%%
%% --- Transformaciones elementales genéricas aplicadas a la derecha de un objeto
%%%%%%%%%%%%%%%%%%%%%%%%%%%%%%%%%%%%
% \fi
%
% \begin{macro}{\dTEEC}
% \begin{macro}{\dTEECp}
% \begin{macro}{\dTEECP}
% \begin{macro}{\dTEECpE}
% \begin{macro}{\dTEECPE}
% Una transformación elemental genérica con exponente por la derecha
%    \begin{macrocode}
\html@def\dTEEC  #1#2#3{ \Ridx  {#3}{\dOEgE{#1}{#2}} }
\html@def\dTEECp #1#2#3{ \Ridxp {#3}{\dOEgE{#1}{#2}} }
\html@def\dTEECP #1#2#3{ \RidxP {#3}{\dOEgE{#1}{#2}} }
\html@def\dTEECpE#1#2#3{ \RidxpE{#3}{\dOEgE{#1}{#2}} }
\html@def\dTEECPE#1#2#3{ \RidxPE{#3}{\dOEgE{#1}{#2}} }
%    \end{macrocode}
% \end{macro}
% \end{macro}
% \end{macro}
% \end{macro}
% \end{macro}
%
%
% \begin{macro}{\dTEC}
% \begin{macro}{\dTECp}
% \begin{macro}{\dTECP}
% \begin{macro}{\dTECpE}
% \begin{macro}{\dTECPE}
% Una transformación elemental genérica por la derecha
%    \begin{macrocode}
\html@def\dTEC  #1#2{ \Ridx  {#2}{{\dOEg{#1}}} }
\html@def\dTECp #1#2{ \Ridxp {#2}{{\dOEg{#1}}} }
\html@def\dTECP #1#2{ \RidxP {#2}{{\dOEg{#1}}} }
\html@def\dTECpE#1#2{ \RidxpE{#2}{{\dOEg{#1}}} }
\html@def\dTECPE#1#2{ \RidxPE{#2}{{\dOEg{#1}}} }
%    \end{macrocode}
% \end{macro}
% \end{macro}
% \end{macro}
% \end{macro}
% \end{macro}
%
%
% \begin{macro}{\dETEC}
% \begin{macro}{\dETECp}
% \begin{macro}{\dETECP}
% \begin{macro}{\dETECpE}
% \begin{macro}{\dETECPE}
% Una transformación elemental espejo genérica por la derecha
%    \begin{macrocode}
\html@def\dETEC  #1#2{ \Ridx  {#2}{{\dEOEg{#1}}} }
\html@def\dETECp #1#2{ \Ridxp {#2}{{\dEOEg{#1}}} }
\html@def\dETECP #1#2{ \RidxP {#2}{{\dEOEg{#1}}} }
\html@def\dETECpE#1#2{ \RidxpE{#2}{{\dEOEg{#1}}} }
\html@def\dETECPE#1#2{ \RidxPE{#2}{{\dEOEg{#1}}} }
%    \end{macrocode}
% \end{macro}
% \end{macro}
% \end{macro}
% \end{macro}
% \end{macro}
%
%
% \begin{macro}{\dInvTEC}
% \begin{macro}{\dInvTEC}
% \begin{macro}{\dInvTEC}
% \begin{macro}{\dInvTEC}
% \begin{macro}{\dInvTEC}
% Una transformación elemental inversa genérica por la derecha
%    \begin{macrocode}
\html@def\dInvTEC  #1#2{ \Ridx  {#2}{{\dInvOEg{#1}}} }
\html@def\dInvTECp #1#2{ \Ridxp {#2}{{\dInvOEg{#1}}} }
\html@def\dInvTECP #1#2{ \RidxP {#2}{{\dInvOEg{#1}}} }
\html@def\dInvTECpE#1#2{ \RidxpE{#2}{{\dInvOEg{#1}}} }
\html@def\dInvTECPE#1#2{ \RidxPE{#2}{{\dInvOEg{#1}}} }
%    \end{macrocode}
% \end{macro}
% \end{macro}
% \end{macro}
% \end{macro}
% \end{macro}
%
%
% \begin{macro}{\dEInvTEC}
% \begin{macro}{\dEInvTEC}
% \begin{macro}{\dEInvTEC}
% \begin{macro}{\dEInvTEC}
% \begin{macro}{\dEInvTEC}
% Una transformación elemental inversa genérica por la derecha
%    \begin{macrocode}
\html@def\dEInvTEC  #1#2{ \Ridx  {#2}{{\dEInvOEg{#1}}} }
\html@def\dEInvTECp #1#2{ \Ridxp {#2}{{\dEInvOEg{#1}}} }
\html@def\dEInvTECP #1#2{ \RidxP {#2}{{\dEInvOEg{#1}}} }
\html@def\dEInvTECpE#1#2{ \RidxpE{#2}{{\dEInvOEg{#1}}} }
\html@def\dEInvTECPE#1#2{ \RidxPE{#2}{{\dEInvOEg{#1}}} }
%    \end{macrocode}
% \end{macro}
% \end{macro}
% \end{macro}
% \end{macro}
% \end{macro}
%
%%%%%%%%%%%%%%%%%%%%%%%%%%%%%%%%%%%%%%%%%%%%%%%%%%%%%%%%%%%%%%%%%%%%%%
%%%%%%%%%%%%%%%%%%%%%%%%%%%%%%%%%%%%%%%%%%%%%%%%%%%%%%%%%%%%%%%%%%%%%%
%
% \paragraph{Transformaciones elementales particulares}
%
% \iffalse
%%%%%%%%%%%%%%%%%%%%%%%%%%%%%%%%%%%%
%% --- Transformaciones elementales particulares
%%%%%%%%%%%%%%%%%%%%%%%%%%%%%%%%%%%%
% \fi
%
%
% \begin{macro}{\dTrF}
% \begin{macro}{\dTrFp}
% \begin{macro}{\dTrFP}
% \begin{macro}{\dTrFpE}
% \begin{macro}{\dTrFPE}
% Transformación o sucesión de transformaciones elementales por la izquierda
%    \begin{macrocode}
\html@def\dTrF  #1#2{\Lidx{#2}{#1}}
\html@def\dTrFp #1#2{\dTrF{#1}{\parentesis{#2}}}
\html@def\dTrFP #1#2{\dTrF{#1}{\Parentesis{#2}}}
\html@def\dTrFpE#1#2{\parentesis{\dTrF{#1}{#2}}}
\html@def\dTrFPE#1#2{\Parentesis{\dTrF{#1}{#2}}}
%    \end{macrocode}
% \end{macro}
% \end{macro}
% \end{macro}
% \end{macro}
% \end{macro}
%
% \begin{macro}{\dTrC}
% \begin{macro}{\dTrCp}
% \begin{macro}{\dTrCP}
% \begin{macro}{\dTrCpE}
% \begin{macro}{\dTrCPE}
% Transformación o sucesión de transformaciones elementales por la derecha
%    \begin{macrocode}
\html@def\dTrC  #1#2{\Ridx{#2}{#1}}
\html@def\dTrCp #1#2{\dTrC{#1}{\parentesis{#2}}}
\html@def\dTrCP #1#2{\dTrC{#1}{\Parentesis{#2}}}
\html@def\dTrCpE#1#2{\parentesis{\dTrC{#1}{#2}}}
\html@def\dTrCPE#1#2{\Parentesis{\dTrC{#1}{#2}}}
%    \end{macrocode}
% \end{macro}
% \end{macro}
% \end{macro}
% \end{macro}
% \end{macro}
%
% \begin{macro}{\dTrFC}
% \begin{macro}{\dTrFCp}
% \begin{macro}{\dTrFCP}
% \begin{macro}{\dTrFCpE}
% \begin{macro}{\dTrFCPE}
% Transformación o sucesión de transformaciones elementales por ambos lados
%    \begin{macrocode}
\html@def\dTrFC  #1#2#3{\LRidx  {#3}{#2}{#1}}
\html@def\dTrFCp #1#2#3{\LRidxp {#3}{#2}{#1}}
\html@def\dTrFCP #1#2#3{\LRidxP {#3}{#2}{#1}}
\html@def\dTrFCpE#1#2#3{\LRidxpE{#3}{#2}{#1}}
\html@def\dTrFCPE#1#2#3{\LRidxPE{#3}{#2}{#1}}
%    \end{macrocode}
% \end{macro}
% \end{macro}
% \end{macro}
% \end{macro}
% \end{macro}
%
%
% \subsubsection{Operador que quita un elemento}
%
% \iffalse
%%%%%%%%%%%%%%%%%%%%%%%%%%%%%%%%%%%%
%% --- OPERADOR QUE QUITA UN ELEMENTO
%%%%%%%%%%%%%%%%%%%%%%%%%%%%%%%%%%%%
% \fi
%
% \begin{macro}{\fueraitemL}
% Signo de operador que quita un elemento
%    \begin{macrocode}
\html@def\fueraitemL#1{{_{}}{#1}{^{\Lsh}}}
%    \end{macrocode}
% \end{macro}
%
%
% \begin{macro}{\fueraitemR}
% Signo de operador que quita un elemento
%    \begin{macrocode}
\html@def\fueraitemR#1{{^{\Rsh\!}}{#1}{_{}}}
%    \end{macrocode}
% \end{macro}
%
%
% \begin{macro}{\quitaLR}
% Sistema resultante de quitar un elemento por la izquierda y otro por la derecha
%    \begin{macrocode}
\html@def\quitaLR#1#2#3{{^{\fueraitemL{#2}\!}}{{#1}}{^{\fueraitemR{#3}}}}
%    \end{macrocode}
% \end{macro}
%
%
% \begin{macro}{\quitaL}
% Sistema resultante de quitar un elemento por la izquierda
%    \begin{macrocode}
\html@def\quitaL#1#2{{^{\fueraitemL{#2}\!}}{{#1}}{^{}}}
%    \end{macrocode}
% \end{macro}
%
%
% \begin{macro}{\quitaR}
% Sistema resultante de quitar un elemento por la derecha
%    \begin{macrocode}
\html@def\quitaR#1#2{{^{}}{{#1}}{^{\!\fueraitemR{#2}}}}
%    \end{macrocode}
% \end{macro}
%
%
% \subsubsection{Selección de elementos sin emplear el operador selector}
%
% \iffalse
%%%%%%%%%%%%%%%%%%%%%%%%%%%%%%%%%%%%
%% --- Selección de elementos sin emplear el operador selector
%%%%%%%%%%%%%%%%%%%%%%%%%%%%%%%%%%%%
% \fi
%
%
% \begin{macro}{\elemUUU}
% Selección de un elemento de un sistema
%    \begin{macrocode}
\html@def\elemUUU#1#2{\textrm{elem}_{#2}\Parentesis{#1}}
%    \end{macrocode}
% \end{macro}
%
%
% \begin{macro}{\VectCCC}
% \begin{macro}{\VectCCCT}
% Selección de una columna de una matriz
%    \begin{macrocode}
\html@def\VectCCC #1#2{\textrm{col}_{#2}\MatP  {#1}}
\html@def\VectCCCT#1#2{\textrm{col}_{#2}\MatTPE{#1}}
%    \end{macrocode}
% \end{macro}
% \end{macro}
%
%
% \begin{macro}{\VectFFF}
% \begin{macro}{\VectFFFT}
% Selección de una columna de una matriz
%    \begin{macrocode}
\html@def\VectFFF #1#2{\textrm{\eng{fila}{row}}_{#2}\MatP  {#1}}
\html@def\VectFFFT#1#2{\textrm{\eng{fila}{row}}_{#2}\MatTPE{#1}}
%    \end{macrocode}
% \end{macro}
% \end{macro}
%
%
% \begin{macro}{\eleMMM}
% \begin{macro}{\eleMMMT}
% \begin{macro}{\eleMM}
% Selección de un elemento de una matriz
%    \begin{macrocode}
\html@def\eleMMM #1#2#3#{\textrm{elem}_{#2#3}{\MatP  {#1}} }
\html@def\eleMMMT#1#2#3#{\textrm{elem}_{#2#3}{\MatTPE{#1}} }
\html@def\eleMM  #1#2#3#{\MakeLowercase{#1}_{{#2}{#3}}     }
%    \end{macrocode}
% \end{macro}
% \end{macro}
% \end{macro}
%
%
% \iffalse
%%%%%%%%%%%%%%%%%%%%%%%%%%%%%%%%%%%%
%% --- Sistemas genéricos
%%%%%%%%%%%%%%%%%%%%%%%%%%%%%%%%%%%%
% \fi
%
% \subsection{Sistemas genéricos}
%
%%%%%%%%%%%%%%%%%%%%%%%%%%%%%%%%%%%%%%%%%%%%%
%
% \begin{macro}{\SV}
% Sistema de Vectores
%    \begin{macrocode}
\html@def\SV#1{\mathsf{#1}}
%    \end{macrocode}
% \end{macro}
%
%
% \begin{macro}{\concatSV}
% Concatenación de sistemas
%    \begin{macrocode}
\html@def\concatSV#1#2{{#1}\mathbin{\concat}{#2}}
%    \end{macrocode}
% \end{macro}
%
%
% \iffalse
%%%%%%%%%%%%%%%%%%%%%%%%%%%%%%%%%%%%
%% --- Notación matricial
%%%%%%%%%%%%%%%%%%%%%%%%%%%%%%%%%%%%
% \fi
%
% \subsection{Vectores y matrices}
% \subsubsection{Vectores}
%
% \begin{macro}{\vect}
% \begin{macro}{\vectp}
% \begin{macro}{\vectP}
% Vector genérico
%    \begin{macrocode}
\html@def\vect #1{\vec{#1}}
\html@def\vectp#1{\parentesis{\vect{#1}}}
\html@def\vectP#1{\Parentesis{\vect{#1}}}
%    \end{macrocode}
% \end{macro}
% \end{macro}
% \end{macro}
%
%%%%%%%%%%%%%%%%%%%%%%%%%%%%%%%%%%%%%%%%%%%%%
% \subsubsection{Vectores de $\R[n]$}
%
% \begin{macro}{\Vect}
% \begin{macro}{\Vectp}
% \begin{macro}{\VectP}
% Vector de $\mathbb{R}^n$
%    \begin{macrocode}
\html@def\Vect #1{\boldsymbol{#1}}
\html@def\Vectp#1{\parentesis{\Vect{#1}}}
\html@def\VectP#1{\Parentesis{\Vect{#1}}}
%    \end{macrocode}
% \end{macro}
% \end{macro}
% \end{macro}
%
%%%%%%%%%%%%%%%%%%%%%%%%%%%%%%%%%%%%%%%%%%%%%
% \subsubsection{Matrices}
%
% \begin{macro}{\Mat}
% \begin{macro}{\Matp}
% \begin{macro}{\MatP}
% Matriz
%    \begin{macrocode}
\html@def\Mat #1{\boldsymbol{\mathsf{#1}}}
\html@def\Matp#1{\parentesis{\Mat{#1}}}
\html@def\MatP#1{\Parentesis{\Mat{#1}}}
%    \end{macrocode}
% \end{macro}
% \end{macro}
% \end{macro}
%
%%%%%%%%%%%%%%%%%%%%%%%%%%%%%%%%%%%%%%%%%%%%%
% \textbf{Matrices transpuestas}
%
% \begin{macro}{\MatT}
% \begin{macro}{\MatTp}
% \begin{macro}{\MatTP}
% \begin{macro}{\MatTpE}
% \begin{macro}{\MatTPE}
% Matriz transpuesta
%    \begin{macrocode}
\html@def\MatT  #1{\Trans  {\Mat{#1}}}
\html@def\MatTp #1{\Transp {\Mat{#1}}}
\html@def\MatTP #1{\TransP {\Mat{#1}}}
\html@def\MatTpE#1{\TranspE{\Mat{#1}}}
\html@def\MatTPE#1{\TransPE{\Mat{#1}}}
%    \end{macrocode}
% \end{macro}
% \end{macro}
% \end{macro}
% \end{macro}
% \end{macro}
%
%%%%%%%%%%%%%%%%%%%%%%%%%%%%%%%%%%%%%%%%%%%%%
% \textbf{\small\quad Matriz transpuesta de la transpuesta}
%
% \begin{macro}{\MatTT}
% \begin{macro}{\MatTTPE}
% Matriz transpuesta
%    \begin{macrocode}
\html@def\MatTT  #1{\TransP{\MatT{#1}}}
\html@def\MatTTPE#1{\Parentesis{\MatTT{#1}}}
%    \end{macrocode}
% \end{macro}
% \end{macro}
%
%%%%%%%%%%%%%%%%%%%%%%%%%%%%%%%%%%%%%%%%%%%%%
% \textbf{Matrices columna}
%
% \begin{macro}{\MVectC}
% Matriz columna creada con una columna
%    \begin{macrocode}
\html@def\MVectC#1#2{\left[\VectC{#1}{#2}\right]}
%    \end{macrocode}
% \end{macro}
%
%
% \begin{macro}{\MVectF}
% Matriz columna creada con una fila
%    \begin{macrocode}
\html@def\MVectF#1#2{\left[\VectF{#1}{#2}\right]}
%    \end{macrocode}
% \end{macro}
%
%%%%%%%%%%%%%%%%%%%%%%%%%%%%%%%%%%%%%%%%%%%%%
% \textbf{Matrices fila}
%
% \begin{macro}{\MVectCT}
% Matriz fila creada con una columna
%    \begin{macrocode}
\html@def\MVectCT#1#2{\Trans{\left[\VectC{#1}{#2}\right]}}
%    \end{macrocode}
% \end{macro}
%
%
% \begin{macro}{\MVectFT}
% Matriz fila creada con una fila
%    \begin{macrocode}
\html@def\MVectFT#1#2{\Trans{\left[\VectF{#1}{#2}\right]}}
%    \end{macrocode}
% \end{macro}
%
%%%%%%%%%%%%%%%%%%%%%%%%%%%%%%%%%%%%%%%%%%%%%
%
% \iffalse
%%%%%%%%%%%%%%%%%%%%%%%%%%%%%%%%%%%%
%% --- Miscelánea matrices
%%%%%%%%%%%%%%%%%%%%%%%%%%%%%%%%%%%%
% \fi
%
% \subsubsection{Miscelánea matrices}
%
% \textbf{Características de las matrices}
%
% \begin{macro}{\Traza}
% Operador traza
%    \begin{macrocode}
\html@def\Traza{\mathrm{tr}}
%    \end{macrocode}
% \end{macro}
%
%
% \begin{macro}{\rg}
% Operador rango
%    \begin{macrocode}
\html@def\Rango{\mathrm{rg}}
%    \end{macrocode}
% \end{macro}
%
%
% \begin{macro}{\traza}
% Traza
%    \begin{macrocode}
\html@def\traza#1{\Traza{\Parentesis{#1}}}
%    \end{macrocode}
% \end{macro}
%
%
% \begin{macro}{\rango}
% Rango
%    \begin{macrocode}
\html@def\rango#1{\rg{\Parentesis{#1}}}
%    \end{macrocode}
% \end{macro}
%
%
% \textbf{Determinante de una matriz}
%
%
% \begin{macro}{\cof}
% Cofactor
%    \begin{macrocode}
\html@def\cof{\mathop{\mathrm{cof}}}
%    \end{macrocode}
% \end{macro}
%
%
% \begin{macro}{\adj}
% Adjunta
%    \begin{macrocode}
\html@def\adj{\mathrm{Adj}}
%    \end{macrocode}
% \end{macro}
%
%
% \begin{macro}{\determinante}
% Determinante con barras
%    \begin{macrocode}
\html@def\determinante#1{\modulus{#1}}
%    \end{macrocode}
% \end{macro}
%
%
% \begin{macro}{\subMat}
% Determinante con barras
%    \begin{macrocode}
\html@def\subMat#1#2#3{\quitaLR{\Mat{#1}}{#2}{#3}}
%    \end{macrocode}
% \end{macro}
%
%
% \begin{macro}{\Menor}
% \begin{macro}{\MenoR}
% Menor de una matriz
%    \begin{macrocode}
\html@def\Menor#1#2#3{\det\big({\subMat{#1}{#2}{#3}}\big)}
\html@def\MenoR#1#2#3{\big|{\subMat{#1}{#2}{#3}}\big|}
%    \end{macrocode}
% \end{macro}
% \end{macro}
%
%
% \begin{macro}{\Cof}
% Cofactor de una matriz
%    \begin{macrocode}
\html@def\Cof#1#2#3{{\cof}_{{#2}{#3}}\parentesis{\Mat{#1}}}
%    \end{macrocode}
% \end{macro}
%
%
% \textbf{Orden de las matrices}
%
% \begin{macro}{\Dim}
% \begin{macro}{\Dimp}
% \begin{macro}{\DimP}
% \begin{macro}{\DimpE}
% \begin{macro}{\DimPE}
% Orden del objeto
%    \begin{macrocode}
\html@def\Dim  #1#2#3{\mathop{#1}\limits_{\scriptscriptstyle #2\times#3}}
\html@def\Dimp #1#2#3{\Dim{\parentesis{#1}}{#2}{#3}}
\html@def\DimP #1#2#3{\Dim{\Parentesis{#1}}{#2}{#3}}
\html@def\DimpE#1#2#3{\parentesis{\Dim{#1}}{#2}{#3}}
\html@def\DimPE#1#2#3{\Parentesis{\Dim{#1}}{#2}{#3}}
%    \end{macrocode}
% \end{macro}
% \end{macro}
% \end{macro}
% \end{macro}
% \end{macro}
%
%
% \begin{macro}{\Matdim}
% \begin{macro}{\Matdimp}
% \begin{macro}{\MatdimP}
% \begin{macro}{\MatdimpE}
% \begin{macro}{\MatdimPE}
% Matriz con el orden por debajo
%    \begin{macrocode}
\html@def\Matdim  #1#2#3{\Dim  {\Mat{#1}}{#2}{#3}}
\html@def\Matdimp #1#2#3{\Dimp {\Mat{#1}}{#2}{#3}}
\html@def\MatdimP #1#2#3{\DimP {\Mat{#1}}{#2}{#3}}
\html@def\MatdimpE#1#2#3{\DimpE{\Mat{#1}}{#2}{#3}}
\html@def\MatdimPE#1#2#3{\DimPE{\Mat{#1}}{#2}{#3}}
%    \end{macrocode}
% \end{macro}
% \end{macro}
% \end{macro}
% \end{macro}
% \end{macro}
%
%
% \textbf{Matriz de autovalores}
%
% \begin{macro}{\MDaV}
% Matriz de autovalores
%    \begin{macrocode}
\html@def\MDaV{D}
%    \end{macrocode}
% \end{macro}
%
%
% \iffalse
%%%%%%%%%%%%%%%%%%%%%%%%%%%%%%%%%%%%
%% --- Productos entre vectores
%%%%%%%%%%%%%%%%%%%%%%%%%%%%%%%%%%%%
% \fi
%
% \subsection{Productos entre vectores}
%
% \subsubsection{Producto escalar}
%
% \begin{macro}{\eSc}
% Producto escalar
%    \begin{macrocode}
\html@def\eSc#1#2{\left<{#1},{#2}\right>}
%    \end{macrocode}
% \end{macro}
%
%
% \begin{macro}{\esc}
% Producto escalar entre vectores genéricos
%    \begin{macrocode}
\html@def\esc#1#2{\left<{\vect{#1}},{\vect{#2}}\right>}
%    \end{macrocode}
% \end{macro}
%
%
% \subsubsection{Producto punto}
%
% \begin{macro}{\dotProd}
% \begin{macro}{\dotProdp}
% \begin{macro}{\dotProdP}
% Producto punto
%    \begin{macrocode}
\html@def\dotProd #1#2{{#1}\cdot{#2}}
\html@def\dotProdp#1#2{\parentesis{\dotProd{#1}{#2}}}
\html@def\dotProdP#1#2{\Parentesis{\dotProd{#1}{#2}}}
%    \end{macrocode}
% \end{macro}
% \end{macro}
% \end{macro}
%
%
% \begin{macro}{\dotprod}
% \begin{macro}{\dotprodp}
% \begin{macro}{\dotprodP}
% Producto punto entre vectores de $\R[n]$
%    \begin{macrocode}
\html@def\dotprod #1#2{\Vect{#1}\cdot\Vect{#2}}
\html@def\dotprodp#1#2{\parentesis{\dotprod{#1}{#2}}}
\html@def\dotprodP#1#2{\Parentesis{\dotprod{#1}{#2}}}
%    \end{macrocode}
% \end{macro}
% \end{macro}
% \end{macro}
%
%
% \subsubsection{Producto punto a punto o \emph{Hadamard}}
%
% \begin{macro}{\prodH}
% \begin{macro}{\prodHp}
% \begin{macro}{\prodHP}
% Producto punto a punto o \emph{Hadamard}
%    \begin{macrocode}
\html@def\prodH#1#2{{#1}\odot{#2}}
\html@def\prodHp#1#2{\parentesis{\prodH{#1}{#2}}}
\html@def\prodHP#1#2{\Parentesis{\prodH{#1}{#2}}}
%    \end{macrocode}
% \end{macro}
% \end{macro}
% \end{macro}
%
%
% \begin{macro}{\prodh}
% \begin{macro}{\prodhp}
% \begin{macro}{\prodhP}
% Producto punto a punto o \emph{Hadamard}
%    \begin{macrocode}
\html@def\prodh #1#2{\Vect{#1}\odot\Vect{#2}}
\html@def\prodhp#1#2{\parentesis{\prodh{#1}{#2}}}
\html@def\prodhP#1#2{\Parentesis{\prodh{#1}{#2}}}
%    \end{macrocode}
% \end{macro}
% \end{macro}
% \end{macro}
%
%
% \iffalse
%%%%%%%%%%%%%%%%%%%%%%%%%%%%%%%%%%%%
%% --- Matriz por vector y vector por matriz
%%%%%%%%%%%%%%%%%%%%%%%%%%%%%%%%%%%%
% \fi
%
% \subsection{Matriz por vector y vector por matriz}
%
% \begin{macro}{\MV}
% \begin{macro}{\MvpE}
% \begin{macro}{\MVPE}
% Producto de matriz por vector
%    \begin{macrocode}
\html@def\MV  #1#2{\Mat{#1}\Vect{#2}}
\html@def\MVpE#1#2{\parentesis{\Mat{#1}\Vect{#2}}}
\html@def\MVPE#1#2{\Parentesis{\Mat{#1}\Vect{#2}}}
%    \end{macrocode}
% \end{macro}
% \end{macro}
% \end{macro}
%
%
% \begin{macro}{\VM}
% \begin{macro}{\VMpE}
% \begin{macro}{\VMPE}
% Producto de vector por matriz
%    \begin{macrocode}
\html@def\VM  #1#2{\Vect{#1}\Mat{#2}}
\html@def\VMpE#1#2{\parentesis{\Vect{#1}\Mat{#2}}}
\html@def\VMPE#1#2{\Parentesis{\Vect{#1}\Mat{#2}}}
%    \end{macrocode}
% \end{macro}
% \end{macro}
% \end{macro}
%
%
% \begin{macro}{\MTV}
% \begin{macro}{\MTVp}
% \begin{macro}{\MTVP}
% Producto de matriz por vector
%    \begin{macrocode}
\html@def\MTV #1#2{\MatT  {#1}\Vect{#2}}
\html@def\MTVp#1#2{\MatTpE{#1}\Vect{#2}}
\html@def\MTVP#1#2{\MatTPE{#1}\Vect{#2}}
%    \end{macrocode}
% \end{macro}
% \end{macro}
% \end{macro}
%
%
% \begin{macro}{\VMT}
% \begin{macro}{\VMTp}
% \begin{macro}{\VMTP}
% Producto de vector por matriz
%    \begin{macrocode}
\html@def\VMT #1#2{\Vect{#1}\MatT  {#2}}
\html@def\VMTp#1#2{\Vect{#1}\MatTpE{#2}}
\html@def\VMTP#1#2{\Vect{#1}\MatTPE{#2}}
%    \end{macrocode}
% \end{macro}
% \end{macro}
% \end{macro}
%
%
% \iffalse
%%%%%%%%%%%%%%%%%%%%%%%%%%%%%%%%%%%%
%% --- Matriz por matriz
%%%%%%%%%%%%%%%%%%%%%%%%%%%%%%%%%%%%
% \fi
%
% \subsection{Matriz por matriz}
%
% \begin{macro}{\MN}
% Producto de matriz por matriz
%    \begin{macrocode}
\html@def\MN#1#2{\Mat{#1}\Mat{#2}}
%    \end{macrocode}
% \end{macro}
%
%
% \begin{macro}{\MTN}
% \begin{macro}{\MTNp}
% \begin{macro}{\MTNP}
% Producto de matriz transpuesta por matriz
%    \begin{macrocode}
\html@def\MTN #1#2{\MatT  {#1}\Mat{#2}}
\html@def\MTNp#1#2{\MatTpE{#1}\Mat{#2}}
\html@def\MTNP#1#2{\MatTPE{#1}\Mat{#2}}
%    \end{macrocode}
% \end{macro}
% \end{macro}
% \end{macro}
%
%
% \begin{macro}{\MNT}
% \begin{macro}{\MNTp}
% \begin{macro}{\MNTP}
% Producto de matriz por matriz transpuesta
%    \begin{macrocode}
\html@def\MNT #1#2{\Mat{#1}\MatT  {#2}}
\html@def\MNTp#1#2{\Mat{#1}\MatTpE{#2}}
\html@def\MNTP#1#2{\Mat{#1}\MatTPE{#2}}
%    \end{macrocode}
% \end{macro}
% \end{macro}
% \end{macro}
%
%
% \begin{macro}{\MTM}
% \begin{macro}{\MTMp}
% \begin{macro}{\MTMP}
% Producto de matriz transpuesta por matriz
%    \begin{macrocode}
\html@def\MTM #1{\MatT  {#1}\Mat{#1}}
\html@def\MTMp#1{\MatTpE{#1}\Mat{#1}}
\html@def\MTMP#1{\MatTPE{#1}\Mat{#1}}
%    \end{macrocode}
% \end{macro}
% \end{macro}
% \end{macro}
%
%
% \begin{macro}{\MMT}
% \begin{macro}{\MMTp}
% \begin{macro}{\MMTP}
% Producto de matriz por su transpuesta
%    \begin{macrocode}
\html@def\MMT #1{\Mat{#1}\MatT  {#1}}
\html@def\MMTp#1{\Mat{#1}\MatTpE{#1}}
\html@def\MMTP#1{\Mat{#1}\MatTPE{#1}}
%    \end{macrocode}
% \end{macro}
% \end{macro}
% \end{macro}
%
%%%%%%%%%%%%%%%%%%%%%%%%%%%%%%%
%
% \begin{macro}{\MNMT}
% \begin{macro}{\MNMTp}
% \begin{macro}{\MNMTP}
% Producto de matriz por matriz por matriz transpuesta
%    \begin{macrocode}
\html@def\MNMT #1#2{\MN{#1}{#2}\MatT {#1}}
\html@def\MNMTp#1#2{\MN{#1}{#2}\MatTp{#1}}
\html@def\MNMTP#1#2{\MN{#1}{#2}\MatTP{#1}}
%    \end{macrocode}
% \end{macro}
% \end{macro}
% \end{macro}
%
%
% \begin{macro}{\MTNM}
% \begin{macro}{\MTNMp}
% \begin{macro}{\MTNMP}
% Producto de matriz transpuesta por matriz por matriz
%    \begin{macrocode}
\html@def\MTNM #1#2{\MatT {#1}\MN{#2}{#1}}
\html@def\MTNMp#1#2{\MatTp{#1}\MN{#2}{#1}}
\html@def\MTNMP#1#2{\MatTP{#1}\MN{#2}{#1}}
%    \end{macrocode}
% \end{macro}
% \end{macro}
% \end{macro}
%
%%%%%%%%%%%%%%%%%%%%%%%%%%%%%%%%%%%%%%%%%%%%%
%
% \iffalse
%%%%%%%%%%%%%%%%%%%%%%%%%%%%%%%%%%%%
%% --- Matriz inversa
%%%%%%%%%%%%%%%%%%%%%%%%%%%%%%%%%%%%
% \fi
%
% \paragraph{Matriz inversa}
%
% \begin{macro}{\InvMat}
% \begin{macro}{\InvMatp}
% \begin{macro}{\InvMatP}
% \begin{macro}{\InvMatpE}
% \begin{macro}{\InvMatPE}
% Inversa de una matriz
%    \begin{macrocode}
\html@def\InvMat  #1{\Inv  {\Mat{#1}}}
\html@def\InvMatp #1{\Invp {\Mat{#1}}}
\html@def\InvMatP #1{\InvP {\Mat{#1}}}
\html@def\InvMatpE#1{\InvpE{\Mat{#1}}}
\html@def\InvMatPE#1{\InvPE{\Mat{#1}}}
%    \end{macrocode}
% \end{macro}
% \end{macro}
% \end{macro}
% \end{macro}
% \end{macro}
%
%
% \begin{macro}{\InvMatT}
% \begin{macro}{\InvMatTpE}
% \begin{macro}{\InvMatTPE}
% Inversa de una matriz transpuesta
%    \begin{macrocode}
\html@def\InvMatT#1{\InvP{\MatT{#1}}}
\html@def\InvMatTpE#1{\parentesis{\InvMatT{#1}}}
\html@def\InvMatTPE#1{\Parentesis{\InvMatT{#1}}}
%    \end{macrocode}
% \end{macro}
% \end{macro}
% \end{macro}
%
%
% \begin{macro}{\TInvMat}
% \begin{macro}{\TInvMatpE}
% \begin{macro}{\TInvMatPE}
% Transpuesta de la inversa de una matriz 
%    \begin{macrocode}
\html@def\TInvMat#1{\Trans{\InvMatpE{\MatT{#1}}}}
\html@def\TInvMatpE#1{\parentesis{\TInvMat{#1}}}
\html@def\TInvMatPE#1{\Parentesis{\TInvMat{#1}}}
%    \end{macrocode}
% \end{macro}
% \end{macro}
% \end{macro}
%
%
% \iffalse
%%%%%%%%%%%%%%%%%%%%%%%%%%%%%%%%%%%%
%% --- Otros productos entre matrices y vectores
%%%%%%%%%%%%%%%%%%%%%%%%%%%%%%%%%%%%
% \fi
%
% \subsection{Otros productos entre matrices y vectores}
%
%
% \begin{macro}{\MTMV}
% \begin{macro}{\MTMVp}
% \begin{macro}{\MTMVP}
% Producto de matriz transpuesta por matriz por vector
%    \begin{macrocode}
\html@def\MTMV #1#2{\MTM {#1}\Vect{#2}}
\html@def\MTMVp#1#2{\MTMp{#1}\Vect{#2}}
\html@def\MTMVP#1#2{\MTMP{#1}\Vect{#2}}
%    \end{macrocode}
% \end{macro}
% \end{macro}
% \end{macro}
%
%
% \begin{macro}{\VMW}
% Producto de vector por matriz por vector
%    \begin{macrocode}
\html@def\VMW#1#2#3{\VMM{#1}{#2}\Vect{#3}}
%    \end{macrocode}
% \end{macro}
%
%
% \begin{macro}{\VMV}
% Producto de vector por matriz por vector
%    \begin{macrocode}
\html@def\VMV#1#2{\VMW{#1}{#2}{#1}}
%    \end{macrocode}
% \end{macro}
%
%
% \begin{macro}{\VMTW}
% \begin{macro}{\VMTWp}
% \begin{macro}{\VMTWP}
% Producto de vector por matriz transpuesta por vector
%    \begin{macrocode}
\html@def\VMTW #1#2#3{\VMT {#1}{#2}\Vect{#3}}
\html@def\VMTWp#1#2#3{\VMTp{#1}{#2}\Vect{#3}}
\html@def\VMTWP#1#2#3{\VMTP{#1}{#2}\Vect{#3}}
%    \end{macrocode}
% \end{macro}
% \end{macro}
% \end{macro}
%
%
% \begin{macro}{\VMTV}
% \begin{macro}{\VMTVp}
% \begin{macro}{\VMTVP}
% Producto de vector por matriz transpuesta por vector
%    \begin{macrocode}
\html@def\VMTV #1#2{\VMTW {#1}{#2}{#1}}
\html@def\VMTVp#1#2{\VMTWp{#1}{#2}{#1}}
\html@def\VMTVP#1#2{\VMTWP{#1}{#2}{#1}}
%    \end{macrocode}
% \end{macro}
% \end{macro}
% \end{macro}
%
%
% \begin{macro}{\InvMTM}
% Inversa del producto de una matriz transpuesta por ella misma
%    \begin{macrocode}
\html@def\InvMTM#1{\InvP{\MTM{#1}}}
%    \end{macrocode}
% \end{macro}
%
%
%
% \iffalse
%%%%%%%%%%%%%%%%%%%%%%%%%%%%%%%%%%%%
%% --- Sistemas de ecuaciones
%%%%%%%%%%%%%%%%%%%%%%%%%%%%%%%%%%%%
% \fi
%
% \subsection{Sistemas de ecuaciones}
%
% \begin{macro}{\SEL}
% Sistema de ecuaciones lineales con notación matricial
%    \begin{macrocode}
\html@def\SEL#1#2#3{\MV{#1}{#2}=\Vect{#3}}
%    \end{macrocode}
% \end{macro}
%
%
% \begin{macro}{\SELT}
% \begin{macro}{\SELTP}
% Sistema de ecuaciones lineales con notación matricial (matriz de coeficientes transpuesta)
%    \begin{macrocode}
\html@def\SELT#1#2#3{\MTV {#1}{#2}=\Vect{#3}}
\html@def\SELT#1#2#3{\MTVP{#1}{#2}=\Vect{#3}}
%    \end{macrocode}
% \end{macro}
% \end{macro}
%
%
% \begin{macro}{\SELF}
% Sistema de ecuaciones lineales con notación matricial (matriz de coeficientes transpuesta)
%    \begin{macrocode}
\html@def\SELF#1#2#3{\VM{#1}{#2}=\Vect{#3}}
%    \end{macrocode}
% \end{macro}
%
%
% \iffalse
%%%%%%%%%%%%%%%%%%%%%%%%%%%%%%%%%%%%
%% --- Espacios vectoriales
%%%%%%%%%%%%%%%%%%%%%%%%%%%%%%%%%%%%
% \fi
%
% \subsection{Espacios vectoriales}
%
%
% \begin{macro}{\EV}
% Sistema de ecuaciones lineales con notación matricial (matriz de coef. transpuesta)
%    \begin{macrocode}
\html@def\EV#1{\mathcal{#1}}
%    \end{macrocode}
% \end{macro}
%
%
% \begin{macro}{\EspacioNul}
% Letra que denota al Espacio nulo (o núcleo) 
%    \begin{macrocode}
\html@def\EspacioNul{\EV{N}}
%    \end{macrocode}
% \end{macro}
%
%
% \begin{macro}{\EspacioCol}
% Letra que denota al Espacio Columna
%    \begin{macrocode}
\html@def\EspacioCol{\EV{C}}
%    \end{macrocode}
% \end{macro}
%
%
% \begin{macro}{\Nulls}
% Espacio nulo (o núcleo) de un objeto
%    \begin{macrocode}
\html@def\Nulls#1{\EspacioNul\Parentesis{#1}}
%    \end{macrocode}
% \end{macro}
%
%
% \begin{macro}{\nulls}
% Espacio nulo (o núcleo) de una matriz
%    \begin{macrocode}
\html@def\nulls#1{\Nulls{\Mat{#1}}}
%    \end{macrocode}
% \end{macro}
%
%
% \begin{macro}{\Cols}
% Espacio columna de un objeto
%    \begin{macrocode}
\html@def\Cols#1{\EspacioCol\Parentesis{#1}}
%    \end{macrocode}
% \end{macro}
%
%
% \begin{macro}{\cols}
% Espacio columna de una matriz
%    \begin{macrocode}
\html@def\cols#1{\Cols{\Mat{#1}}}
%    \end{macrocode}
% \end{macro}
%
%
% \begin{macro}{\Span}
% Espacio generado por un sistema generador
%    \begin{macrocode}
\html@def\Span#1{\EV{L}\Parentesis{#1}}
%    \end{macrocode}
% \end{macro}
%
%
% \begin{macro}{\PSpan}
% Espacio semi-euclídeo de probabilidad generado por un sistema
%    \begin{macrocode}
\html@def\PSpan#1{\EV{L{\!\!{\scriptscriptstyle{{}^P}}}}\Parentesis{#1}}
%    \end{macrocode}
% \end{macro}
%
%
% \begin{macro}{\coord}
% \begin{macro}{\coordP}
% \begin{macro}{\coordPE}
% Coordenadas respecto de una base
%    \begin{macrocode}
\html@def\coord  #1#2{\Ridx{#1}{\Ridx{\mathbin{/}}{#2}}}
\html@def\coordP #1#2{\coord{\Parentesis{#2}}{#1}}
\html@def\coordPE#1#2{\Parentesis{\coord{#2}{#1}}}
%    \end{macrocode}
% \end{macro}
% \end{macro}
% \end{macro}
%
%
% \iffalse
%%%%%%%%%%%%%%%%%%%%%%%%%%%%%%%%%%%%
%% --- Notación funcional
%%%%%%%%%%%%%%%%%%%%%%%%%%%%%%%%%%%%
% \fi
%
% \subsection{Notación funcional}
%
% \begin{macro}{\dom}
% Dominio de una función
%    \begin{macrocode}
\html@def\dom{\mathop{\mathrm{dom}}}
%    \end{macrocode}
% \end{macro}
%
%
% \begin{macro}{\mifun}
% Breve descripción de una función
%    \begin{macrocode}
\html@def\mifun#1#2#3{#1 \colon #2 \to #3}
%    \end{macrocode}
% \end{macro}
%
%
% \begin{macro}{\deffun}
% Breve descripción de una función
%    \begin{macrocode}
\html@def\deffun#1#2#3#4#5{%
    \begin{array}{r@{\,}ccl}
        #1\colon & #2 & \longrightarrow & #3\cr
            & #4 & \longmapsto & \displaystyle#5
    \end{array}}
%    \end{macrocode}
% \end{macro}
%
%%%%%%%%%%%%%%%%%%%%%%%%%%%%%%%%%%%%%%%%%%%%%
%
% \iffalse
%%%%%%%%%%%%%%%%%%%%%%%%%%%%%%%%%%%%
%% --- Estadística
%%%%%%%%%%%%%%%%%%%%%%%%%%%%%%%%%%%%
% \fi
%
% \subsection{Estadística}
%
%
% \begin{macro}{\Estmc}
% Ajuste por MCO
%    \begin{macrocode}
\html@def\Estmc#1{ \widehat{#1} }
%    \end{macrocode}
% \end{macro}
%
%
% \begin{macro}{\Media}
% Media (proyección ortogonal sobre los vectores contantes)
%    \begin{macrocode}
\html@def\Media#1{ \widebar{#1} }
%    \end{macrocode}
% \end{macro}
%
%
% \begin{macro}{\Smedia}
% Símbolo para el valor medio
%    \begin{macrocode}
\html@def\Smedia{\mu}
%    \end{macrocode}
% \end{macro}
%
%
% \begin{macro}{\media}
% Valor medio
%    \begin{macrocode}
\html@def\media#1{ {\Smedia}_{#1} }
%    \end{macrocode}
% \end{macro}
%
%
% 
% \Finale
\endinput
